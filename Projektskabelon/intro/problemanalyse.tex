\chapter{Problemanalyse}
\textit{I dette kapitel analyses problemstillinger, som opstår i forbindelse med lægemiddelskift. Disse problemstillinger vil sammenfattes i en opsummering og afsluttes med en problemformulering, der danner fremadrettet grundlaget for rapporten.}

\section{Typer af lægemiddelskift}
Lægemiddelskift kan forekomme i forbindelse med kontraktskift eller ved restordre~\citep{Amgros2015}. Et årligt udbud foretages af Amgros, hvis der findes mere end én leverandør af lægemiddelet. På denne måde bringes lægemidlerne i konkurrence, hvilket kan medføre et kontraktskift. I tilfælde af patent på lægemidlet er der ikke analog konkurrence, da prisen på lægemidlet ofte er fastsat. Restordre forekommer når efterspørgslen på et lægemiddel overstiger den tilgængelige mængde.~\citep{Amgros2015}. Udover kontraktskift og restordre kan bagatelkøb medføre præparatskift. Ved bagatelkøb er sygehusapoteket ikke forpligtet til at anvende lægemidlet og leverandøren omfattes ikke af indkøbs- eller forsyningspligt~\fxnote{\url{https://levportal.amgros.dk/Udbudsoversigt/Sider/Bagatelkob.aspx}}

\subsection{Kontraktskift}
Før et udbud finder sted har Amgros en dialog med medicinrådet~\citep{Amgros2017, Amgros2017a}. Lægemidlerne vurderes i forhold til effekt, eksisterende behandling og pris med det formål at stræbe efter laveste priser samt bedst mulig behandling for patienterne. Medicinrådet kategoriserer nye lægemidler og indikationer i forhold til den nuværende standardbehandling i stor, vigtig, lille eller ingen merværdi. Ud fra dette sammenstiller Amgros standardbehandlingen over for en omkostningsanalyse, der er udarbejdet af ansøgeren for lægemiddelskift. Amgros vurderer, hvorvidt de tilsendte oplysninger er relevante og valide. Den kliniske merværdi, omkostningsanalyse og estimeringen af budget konsekvenser danner grundlaget for prisforhandling. Medicinrådet beslutter efter prisforhandlingen, hvorvidt det nye lægemiddel skal anvendes som standardbehandling. Hvis dette er tilfældet foretages et kontraktskift.~\citep{Amgros2017, Amgros2017a}

\subsection{Restordre}
Restordre kan forekomme på forskellige måder som f.eks. leveringesvigt, mangel på råvarer, forhindring i produktion, større forbrug af lægemidlet end beregnet eller at lægemidlet er solgt til en højere pris til et andet land~\citep{Amgros2017}. I tilfælde af restordre er det leverandørens ansvar at dække hospitalsapotekernes udgift ved indkøb af et erstatningslægemiddel~\fxnote{\url{https://levportal.amgros.dk/SiteCollectionDocuments/1.\%20Grundl\%C3\%A6ggende\%20information\%20om\%20l\%C3\%A6gemiddeludbud.pdf}}.


\section{Problematikker ved lægemiddelskift}
Der er både økonomiske og patientsikkerhedsmæssige konsekvenser ved et lægemiddelskift....

\subsection{Kontraktskift}

\subsection{Restordre}
Det er omkostningsfuldt når et lægemiddel går i restordre, da arbejdsgangen på hospitalsafdelingerne skal justeres~\citep{Amgros2015}. Der skal f.eks. foretages vareskift i systemer, på lagre og i medicinrum. Yderligere skal der gives behandlingsinstruktioner til hospitalsafdelingerne. Udover de økonomiske faktorer udgør restordre patientsikkerhedsmæssige risici. Der kan opstå utilsigtede hændelser (UTH'er), hvis præparatet har skiftet navn, udseende og håndteringen af medicin er ændret. Disse risici kan mindskes ved instrukser og ændring af procedure på hospitalerne.~\citep{Amgros2015}


\section{Løsningsstartegier ved lægemiddelskift}

\subsection{Kontraktskift}

\subsection{Restordre}

*** SE 19Præparat ***