\section*{Begreber}
\begin{table}[H]
\begin{tabular}{p{4.5cm} p{10.5cm}}
\textbf{Analoge lægemidler:} & Lægemidler med beslægtet kemi og ensartet klinisk virkning. \vspace{0.5cm} \\
\textbf{Generiske lægemidler:} & Lægemidler der indeholder samme aktive stof med forskellige hjælpestoffer.\vspace{0.5cm} \\
\textbf{Kontraktskift:} & Skift af kontrakt mellem leverandør og Amgros i forbindelse amgros udbud.  \vspace{0.5cm} \\
\textbf{Restordre:} & Efterspørgslen på et lægemiddel overstiger den mængde lægemiddel der er tilgængeligt.  \vspace{0.5cm} \\
Bagatelkøb & Indkøb af lægemidler med en omsætning på under 500.000 kroner årligt. \vspace{0.5cm} \\
Utilsigtede hændelser & En begivenhed, der forekommer i forbindelse med sundhedsfaglig virksomhed eller forsyning af og information om lægemidler. Utilsigtede hændelser omfatter på forhånd kendte og ukendte hændelser og fejl, som ikke skyldes patientens sygdom og som enten er skadevoldende eller ikke-skadevoldende ved forekomst. \vspace{0.5cm} \\
& \vspace{0.5cm} \\
\end{tabular}
\end{table}




\section*{Beskrivelse??}
\begin{table}[H]
\begin{tabular}{p{3cm} p{11cm}}
\textbf{Amgros:} & Regionernes lægemiddelorganisation, hvis formål er at sikre forsyning af lægemidler til offentlige hospitaler i Danmark med henblik på at skærpe konkurrencen mest muligt, samtidigt med at kvalitet og patientsikkerhed sikres. \vspace{0.5cm}
\\ 
\textbf{Medicinrådet:} & Et uafhængigt råd, der udarbejder anbefalinger i forhold til standardbehandlinger og behandlingsvejledninger om lægemidler til de fem danske regioner. \vspace{0.5cm} \\
& \vspace{0.5cm} \\
& \vspace{0.5cm} \\
& \vspace{0.5cm} \\
& \vspace{0.5cm} \\
& \vspace{0.5cm} \\
\end{tabular}
\end{table}