\section{Typer af lægemiddelskift}
Lægemiddelskift kan enten ske i forbindelse med kontraktskift eller ved restordre~\citep{Amgros2015}. Et årligt udbud foretages af Amgros, hvis der findes mere end én leverandør af lægemiddelet. På denne måde bringes lægemidlerne i konkurrence, hvilket kan medføre et kontraktskift. I tilfælde af patent på lægemidlet er der ikke analog konkurrence, da prisen på lægemidlet ofte er fastsat. Restordre forekommer når efterspørgslen på et lægemiddel overstiger den tilgængelige mængde. Dette kan ske på flere forskellige måder som f.eks. mangel på en råvare til at producere medicinen, forhindring i produktion og at leverandøren har solgt lægemidlet til en højere pris til et andet land.~\citep{Amgros2015}

\subsection{Kontraktskift}
Før et udbud finder sted har Amgros en dialog med medicinrådet~\citep{Amgros2017, Amgros2017a}. Lægemidlerne vurderes i forhold til effekt, eksisterende behandling og pris med det formål at stræbe efter laveste priser samt bedst mulig behandling for patienterne. Medicinrådet kategoriserer nye lægemidler og indikationer i forhold til den nuværende standardbehandling i stor, vigtig, lille eller ingen merværdi. Ud fra dette sammenstiller Amgros standardbehandlingen over for en omkostningsanalyse, der er udarbejdet af ansøgeren for lægemiddelskift. Amgros vurderer, hvorvidt de tilsendte oplysninger er relevante og valide. Den kliniske merværdi, omkostningsanalyse og estimeringen af budget konsekvenser danner grundlaget for prisforhandling. Medicinrådet beslutter efter prisforhandlingen, hvorvidt det nye lægemiddel skal anvendes som standardbehandling. Hvis dette er tilfældet foretages et kontraktskift.~\citep{Amgros2017, Amgros2017a}

\subsection{Restordre}
