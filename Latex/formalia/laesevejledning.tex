\section*{Læsevejledning}
\textit{I dette afsnit beskrives opbygningen af rapporten, samt hvordan referencer til figurer og tabeller er angivet. Ligeledes beskrives anvendelse af forkortelser og begreber samt angivelse af referencer til litteratur.}

Rapporten påbegyndes i kapitel 1 med en indledning, hvor problemstillinger ved lægemiddelskift tydeliggøres. Disse problemstillinger analyseres i kapitel 2 ved problemanalysen, hvor overordnede problemstillinger identificeres og sammenfattes i en opsummering. Problemanalysen danner grundlag for udformningen af problemformuleringen. Ud fra problemformuleringen udformes kapitel 3, som indeholder metode.  Resultater opnået ud fra metoden fremlægges af kapitel 4. Hvorefter metoden og resultater diskuteres i kapitel 5. For til sidst at konkludere på problemformuleringen i kapitel 6. Efterfølgende er litteratur og appendiks angivet.

Figurer og tabeller er i rapporten angivet efter det pågældende kapitel. Dette vil sige, at den første figur i kapitel 2 er angivet Figur 2.1 og den første tabel i kapitel 2 er angivet Tabel 2.1. Figurtekst er angivet under den tilhørende figur og tabeltekst er angivet over den tilhørende tabel. 

Forkortelser er i rapporten angivet det førstnævnte sted med ordet med efterfølgende forkortelse angivet i parentes, hvorefter forkortelsen er anvendt i rapporten efterfølgende. Forkortelser og begreber, som ikke er yderligere beskrevet i rapporten, er listet nedenfor i alfabetisk rækkefølge.

Kilder er i rapporten angivet efter Vancouver som kildehenvisning, hvilket betyder, at kilderne nummereres fortløbende og angives i firkantet parentes. Hvis en kilde er angivet før et punktum i en sætningen gælder denne for den pågældende sætningen, hvorimod en kilde efter punktum er gældende for hele sektionen.

\begin{table}[H]
%\caption{Forkortelser}
\label{table:forkortelser}
\begin{tabular}{p{4.5cm} p{9.8cm}}
\multicolumn{2}{c}{\cellcolor[HTML]{C0C0C0}\textbf{ FORKORTELSER}} \vspace{0.2cm} \\
\textbf{ATC} &  Anatomisk Terapeutisk Kemisk \vspace{0.2cm} \\
\textbf{ROC} & Receiver Operating Characteristic \vspace{0.2cm} \\
\textbf{SRN} & Sygehusapoteket Region Nordjylland \vspace{0.2cm} \\
%%\textbf{UTH} 
%%& Utilsigtet hændelse \vspace{0.5cm} \\
%%\textbf{RS} & Registreret specialitet \vspace{0.5cm} \\
%%\textbf{IRS} & Ikke-registreret specialist  \vspace{0.5cm} \\
%%\textbf{RADS} & Rådet for anvendelse af dyr sygehusmedicin \vspace{0.5cm} \\
%\textbf{ATC} & Anatomical Therapeutic Chemical \vspace{0.5cm} \\
%  & \vspace{0.5cm} \\
\end{tabular}
\end{table}
\vspace{-0.8cm}

\begin{table}[H]
%\caption{Begreber}
\label{table:begreber}
\begin{tabular}{p{4.5cm} p{9.8cm}}
\multicolumn{2}{c}{\cellcolor[HTML]{C0C0C0}\textbf{BEGREBER}} \vspace{0.2cm}\\
\textbf{Amgros} & Står for indkøb af lægemidler til landets otte sygehusapoteker og opnår besparelser på lægemidler ved at sende disse i udbud~\citep{Amgros2018b}. \vspace{0.2cm} \\
\textbf{Lægemiddelkomitéen} & Tager stilling til godkendelse af lægemidler på hospitalerne og udarbejder relevante retningsgivende dokumenter, som gælder for lægemiddelområdet~\citep{RegionNordjylland2018}. \vspace{0.2cm} \\
\textbf{Medicinrådet} & Udarbejder anbefalinger og vejledninger om lægemidler til de fem regioner~\citep{Medicinradet2018}.\vspace{0.2cm} \\
\textbf{Utilsigtede hændelse} & En begivenhed, som omfatter kendte og ukendte hændelser og fejl inden for sundhedsvæsnet, som ikke skyldes patientens sygdom, og er skadevoldende eller kunne have været skadevoldende~\citep{AalborgKommune2017}. \\
%\textbf{Analoge substitution:} & Lægemidler med forskelligt aktive stof med nogenlunde ensartet klinisk virkning~\citep{DanskSelskabforPatientsikkerhed2009} . \vspace{0.5cm} \\
%\textbf{Generisk substitution:} & Lægemidler med samme aktive stof \vspace{0.5cm} \\
%\textbf{Kontraktskift:} & Kontraktskift mellem leverandør og Amgros ved Amgrosudbud~\citep{DanskSelskabforPatientsikkerhed2009}.  \vspace{0.5cm} \\
%%\textbf{Restordre:} & Efterspørgslen på et lægemiddel overstiger den tilgængelige mængde af lægemiddel.  \vspace{0.5cm} \\
%%\textbf{Bagatelkøb} & Indkøb af lægemidler med en omsætning på under 500.000 kroner årligt. \vspace{0.5cm} \\
%%\textbf{Utilsigtede hændelser} &  Begivenhed, der forekommer i forbindelse med sundhedsfaglig virksomhed, herunder præhospital indsats, eller i forbindelse med forsyning af og information om lægemidler. Omfatter på forhånd kendte og ukendte hændelser og fejl, som ikke skyldes patientens sygdom, og som enten er skadevoldende eller kunne have været skadevoldende, men forinden blev afværget eller i øvrigt ikke indtraf på grund af omstændighederne. %En begivenhed, der forekommer i forbindelse med sundhedsfaglig virksomhed eller forsyning af og information om lægemidler. Utilsigtede hændelser omfatter på forhånd kendte og ukendte hændelser og fejl, som ikke skyldes patientens sygdom og som enten er skadevoldende eller ikke-skadevoldende ved forekomst.  (Sundhedsloven, Kapitel 61 \§ 198)
% \vspace{0.5cm} \\
%%\textbf{Registreret specialitet:} & Lægemiddel registreret og godkendt af lægemiddelstyrelsen~\citep{Laegemiddelinformaion2017}. \vspace{0.5cm} \\
%%\textbf{Ikke-registreret specialist} & Lægemiddel, der aldrig har været godkendt eller afregistreret i Danmark~\citep{Laegemiddelinformaion2017}. \vspace{0.5cm} \\
%%\textbf{Magistrelt lægemiddel} & Lægemiddel fremstillet på et apotek og ikke vurderet af myndighederne i forhold til kvalitet, sikkerhed og effekt~\citep{Laegemiddelinformaion2017}. \vspace{0.5cm} \\
%%\textbf{Merværdi:} & Den ekstra værdi et lægemiddel kan tilbyde sammenlignet med nuværende standardbehandling vurderet ud fra patientrelaterede kriterier som livsforlængelse, alvorlige symptomer og bivirkninger, helbredsrelateret livskvalitet samt ikke-alvorlige symptomer og bivirkninger  \vspace{0.5cm} \\
%\textbf{Standardbehandling} & Lægemidlet indføres som et alment anvendt behandlingstilbud til en patientgruppe\vspace{0.5cm} \\
%& \vspace{0.5cm} \\
%& \vspace{0.5cm} \\
\end{tabular}
\end{table}


%\begin{table}[H]
%%\caption{Beskrivelse??}
%\label{table:beskrivelser}
%\vspace{0.5cm}
%\begin{tabular}{p{4.5cm} p{9.8cm}}
%\multicolumn{2}{c}{\cellcolor[HTML]{C0C0C0}\textbf{TABEL 3 - BESKRIVELSER}} \vspace{0.5cm} \\
%\textbf{Amgros:} & Regionernes lægemiddelorganisation, hvis formål er at sikre forsyning af lægemidler til offentlige hospitaler i Danmark med henblik på at skærpe konkurrencen mest muligt, samtidigt med at kvalitet og patientsikkerhed sikres. \vspace{0.5cm}
%\\ 
%\textbf{Medicinrådet:} & Et uafhængigt råd, der udarbejder anbefalinger i forhold til standardbehandlinger og behandlingsvejledninger om lægemidler til de fem danske regioner. \vspace{0.5cm} \\
%\textbf{Sygehusapoteket:} & Sikre forsyning af lægemidler,
%fremstilling af sygehusspecifikke lægemidler og leverance af klinisk farmaceutiske serviceydelser. \vspace{0.5cm} \\
%\textbf{Lægemiddelstyrelsen} & Kontrollere og godkender lægemiddelvirksomheder og lægemidler på det danske marked samt overvåger bivirkninger ved lægemidler og godkender kliniske forsøg. Beslutter tilskud til lægemidler og fører tilsyn med medicinsk udstyr. Overvåger utilsigtede hændelser med medicinsk udstyr samt udpeger apotekere, tilrettelægger apoteksstrukturen og fører tilsyn med apoteker og detailforhandlere.\vspace{0.5cm} \\
%\textbf{RADS} & Sikrer ensartet anvendelse af dyr medicin på landets sygehus. Fra år 2017 har Medicinrådet overtaget RADS' opgaver og dens fagudvalg. \vspace{0.5cm} \\
%& \vspace{0.5cm} \\
%& \vspace{0.5cm} \\
%& \vspace{0.5cm} \\
%\end{tabular}
%\end{table}




