\chapter*{Forord}
Denne rapport er et 3. semesters kandidatprojekt på uddannelsen Sundhedsteknologi (M.sc. Biomedical Engineering and Informatics) på Aalborg Universitet. Projektet er udarbejdet i perioden september 2018 til december 2018 af Maria Kaalund Kroustrup. 

Projektet er udarbejdet med udgangspunkt i det overordnede tema for semesteret "Anvendt sundhedsteknologi og informatik". I studieordningen for uddannelsen fremgår det, at fokus er at være i stand til selvstændigt at initiere eller udføre samarbejde inden for disciplinen samt tage ansvar for egen faglige udvikling~\citep{Studieordning2011}. 

Dette projekt omhandler udviklingen af et regelbaseret system til risikovurdering af lægemiddelskift med henblik på at forbedre den nuværende vurdering af lægemiddelskift og derved opnå en mere effektiv implementering af lægemiddelskift, hvilket vil kunne bidrage til at forebygge medicineringsfejl og derved forbedre patientsikkerheden. 

Der rettes en stor tak til vejleder Kirstine Rosenbeck Gøeg for vejledningen i projektperioden. Yderligere rettes der tak til eksterne vejleder Hanne Plet, Emilie Middelbo Outzen og Lina Klitgaard Larsen for sparring og bidrag til viden inden for lægemiddelskift og sygehusapoteket. Sidst men ikke mindst rettes der tak til Sygehusapoteket Region Nordjylland og deres medarbejdere for samarbejdet. 

\vspace{1.5cm}
\begin{center}
\rule{6cm}{0.4pt} \\
Maria Kaalund Kroustrup \\
\textit{mkrous14@student.aau.dk}
\end{center}