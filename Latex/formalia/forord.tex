\chapter*{Forord}
Denne rapport er et 3. semesters kandidatprojekt på kandidatuddannelsen Sundhedsteknologi (M.sc. Biomedical Engineering and Informatics) på Aalborg Universitet. Projektet er udarbejdet i perioden september 2018 til december 2018 af Maria Kaalund Kroustrup. 

Projektet er udarbejdet med udgangspunkt i det overordnede tema for semesteret "Anvendt sundhedsteknologi og informatik". I studieordningen for uddannelsen fremgår det at fokus er at være i stand til selvstændigt at initiere eller udføre samarbejde inden for disciplinen samt tage ansvar for deres egen faglige udvikling~\citep{Studieordning2011}. 

Dette projekt omhandler udviklingen af en algoritme til kategorisering ved lægemiddelskift. Algoritmen udvikles med henblik på at opstille retningslinjer for behandlingsinstrukser i forhold til kategoriseringen.......


Der rettes stor tak til vejleder Kirstine Rosenbeck Gøeg for vejledningen i projektperioden. Yderligere rettes der tak til eksterne vejleder Hanne Plet for sparring og bidrag til viden inden for sygehusapoteket. Sidst men ikke mindst rettes der tak til samarbejdet med Sygehusapoteket Region Nordjylland. 

\vspace{1.5cm}
\begin{center}
\rule{6cm}{0.4pt} \\
Maria Kaalund Kroustrup \\
\textit{mkrous14@student.aau.dk}
\end{center}