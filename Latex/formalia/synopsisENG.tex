\vspace{-0.3cm}
Drug substitution occur in order to save expenditures on medicine. Substitution of drugs contribute to medications error in the clinic, which can affect the patient safety. The implementation of drug substitution is important in proportion to reduce medication errors. 

Risk assessment of drug substitution before implementation in the clinic is performed by employee from the hospital pharmacy in region Northern Jutland. The assessment occurs manually and is based on experience, which make it vulnerable and person-dependent.

The aim of this project is to develop a rule-based system for risk assessment of drug substitution and evaluate the applicability of the system. Risk assessment is based on risk factors, which is described in the literature and is used in the current process. On the basis of this, a risk score is calculated as an assisting tool for the employee in the hospital pharmacy to assess and differentiate between drugs, that need particularly attention when implementing in the clinic. 

It is concluded that a rule-based system has the applicability for risk assessment of drug substitution, but it is necessary to modify the system in order to improve the current assessment of drug.
\vspace{-0.2cm}