\vspace{-0.3cm}
Substitution af lægemidler sker for at opnå økonomiske besparelser på sygehusmedicin, hvilket kan medvirke til fejlmedicinering i klinikken og derved have patientsikkerhedsmæssige konsekvenser som følge. Implementering af substituerede lægemidler har derfor betydning for forebyggelsen af medicineringsfejl. 

Risikovurderingen af lægemidler inden implementering i klinikken foretages i Region Nordjylland af medarbejdere fra Sygehusapoteket i regionen. Denne vurderingen sker manuelt og er baseret på erfaringer, hvilket gør den sårbar og personafhængig. 

%Ingen videnskabelig litteratur har undersøgt anvendelsen af informationssystemer til forbedring af den nuværende vurdering af lægemiddelskift, men risikovurdering som beslutningsstøtte system er anvendt inden for samme domæne.

Bidraget i dette projekt er at udvikle et regelbaseret system til risikovurdering af lægemiddelskift og evaluere anvendeligheden af systemet med henblik på at løse problemstillingen. Risikovurderingen foretages ud fra risikofaktorer, som er beskrevet i litteraturen samt ud fra den nuværende vurdering af lægemiddelskift. På baggrund af disse er der udregnet en risikoscore, der skal anvendes, som et hjælpeværktøj til medarbejderne fra Sygehusapoteket i forhold til at skelne mellem, hvornår et lægemiddelskift kræver særligt opmærksomhed ved implementering.

Det konkluderes at et regelbaseret system er anvendeligt til risikovurdering af lægemiddelskift, men at systemet kræver tilpasning for at kunne forbedre den nuværende vurdering af lægemiddelskift.
\vspace{-0.2cm}