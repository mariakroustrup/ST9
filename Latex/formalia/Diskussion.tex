
\chapter{Diskussion}
\textit{I dette kapitel diskuteres metode og resultater i forhold til at besvare problemformuleringen.}

%Hvorfor virker det anvendeligt for dem? Og hvorfor gør det ikke? Hvordan kan det videreudvikles i forhold til nu? Medarbejdernes erfaring er vigtigt

\section{Opsummering af resultater}
Flere lægemiddelskift er vurderet af medarbejderne til at skulle rangeres lavere end systemet. En tendens forekommer for lægemiddelskift, hvor risikofaktorer, såsom look-a-like, Medicinrådet og ATC-kritiske, indgår i risikoscoren. Systemets risikofaktorer og vægtningen af disse skal derfor overvejes i forhold til anvendeligheden. Ligeledes blev det belyst, at erfaringer har en stor indflydelse på risikovurderingen af lægemiddelskift. %hvormed det kan diskuteres, hvorvidt inkorporering af andre risikofaktorer kan forbedre systemets anvendelighed.

Systemets performance blev vurderet til at være god, med et areal under kurven på 0,840, i forhold til at forudsige hvornår et lægemiddelskift kræver uddybende information. På baggrund af dette er grænseværdier fastsat, hvormed det er muligt at inddele lægemiddelskift ud fra risikoscoren i forhold til, hvornår et lægemiddelskift kræver opmærksomhed. %begrænset opmærksomhed, opmærksomhed og særlig opmærksomhed. Systemets performance skal diskuteres i forhold til andre systemer. 

%Da systemet blev anvendt til at undersøge anvendeligheden af et regelbaseret system til risikovurdering af lægemiddelskift,

% skal det overvejes, hvad der kræves for implementeringen af et endeligt system samt hvilken vedligeholdelse systemet kræver og hvilke udfordringer der kan være relateret til dette. 

\section{Risikofaktorer og vægtning}
Look-a-like vurderes ikke af medarbejderne i den nuværende vurdering og er derfor ikke vægtet høj ved risikovurderingen. Det kan derfor diskuteres, hvorvidt funktionen er anvendelig. I klinikken kan forvirring over lignende navn opstå og lede til medicineringsfejl, jævnfør Afsnit \ref{sec:ProblemLaeg}, hvorfor opmærksomhed på look-a-like er anvendeligt for klinikken. Look-a-like anses ikke for at være vigtig i risikovurderingen af lægemiddelskift og bør derfor ikke vægtes ved beregning af risikoscoren, men tilgængeligheden af denne i forhold til at informere klinikken om look-a-likes er stadig relevant.
Derudover kan koblingen af look-a-like med f.eks. dispenseringsform eller terapiområde gøre funktionen mere anvendelig ved at look-a-like ikke kun er defineret ud fra lægemidlets navn, men for lægemiddelnavne med samme dispenseringsform eller terapiområde. Ligeledes kan det overvejes, hvorvidt look-a-likes skal defineres af eksperter på området, hvormed sammenligningen mellem lægemiddelnavne er sat i relation og derfor vil det kunne forventes at være mere relevant og anvendelig.

Lægemiddelskift, som indgår i Medicinrådets behandlingsvejledning, blev vurderet til ikke at have en betydning medmindre, at der var foretaget ændringer i behandlingsvejledningen. Ligeledes blev det påvist, at antallet af ændringer i behandlingsvejledningen har betydning for kompleksiteten, hvormed det skal diskuteres, hvorvidt der skal differentieres mellem antallet af ændringer for at optimere anvendeligheden af risikofaktoren. Lægemidler, som indgår i Medicinrådets behandlingsvejledning anses derfor ikke som vigtig i risikovurderingen, hvormed denne ligeledes ikke bør vægtes ved beregningen af risikoscoren, men stadig være tilgængelighed for at gøre medarbejderne opmærksomhed på dette og derved vurdere, hvorvidt det har relevans for det enkelte lægemiddelskift.

Graden af differentiering inden for hver risikofaktorer skal vurderes i forhold til at opnå en bedre anvendelighed af systemet. Dette gælder både for ændringer i lægemidlets navn, f.eks. ved skift fra originalt til generisk, dispenseringsform og styrke. Ligeledes skal dispenseringsformer ligestilles, da f.eks. skift fra dispergible tabletter og frysetørrede tabletter, ikke har betydning for klinikken samt at ændringer i styrke kan variere alt efter om det er den reelle styrke eller styrkeangivelsen der er ændret. Dette vil kræve at tilfælde, hvor ændringer kan ligestilles identificeres samt at gradueringsgraden for hver risikofaktor defineres af én ekspert eller flere eksperter før det implementeres i systemet. Dertil skal det vurderes, hvorvidt nogle ændringer skal prioriteres højere end andre. 

\section{Inkorporering af risikofaktorer}
Risikovurdering af lægemiddelskift er meget erfaringsbaseret og flere faktorer vurderes af medarbejdere end der indgår i systemets risikovurdering såsom, hvilke afdelinger der anvender det, hvor mange patienter samt hvor meget lægemiddelskiftet koster. For at inkorporere dette i systemet, vil det kræve at systemet kobles med data om forbrug i forhold til afdelinger og historisk data om prisniveauer på medicin. Ligeledes kræves det at der defineres en grænse for, hvornår disse faktorer vil medvirke til at lægemiddelskiftet er kritisk. Dette kan være vanskeligt at definere, da det afhænger af mange faktorer som f.eks., hvilken afdeling der er involveret, hvorvidt denne afdeling har erfaring med lægemiddelskift og derfor er mere eller mindre opmærksom på ændringer. Dette skal ligeledes vurderes i forhold til, hvor meget lægemidlet koster ud fra hvor mange afdelinger og patientgrupper som anvender lægemidlet. Dertil skal det vurderes, hvorvidt risikofaktorer, som er erfaringsbaseret, skal inkorporeres i systemet eller om vurderingen af disse håndteres bedre af én ekspert eller flere eksperter inden for området.

\section{Systemets performance}
Ud fra sammenligningen med andre studier som foretager risikovurdering af diagnostiske tests performer systemet til risikovurdering af lægemiddelskift ligeværdigt~\citep{Chan2009,VanStraten2010}. For et multistudie, der undersøger performancen af systemer til forudsigelse af risikoen for kardiovaskulære sygdomme for patienter med diabetes, varierede arealet under kurven mellem 0,64 og 0,49~\citep{Chan2009}, hvilket henholdsvis indikerer en mindre nøjagtig og en ikke informativ test~\citep{Greiner2000}. Dette vil sige, at systemet til risikovurderingen af lægemiddelskift preformer bedre, med et areal under kurven på 0,84. Modsat er det i et andet multistudie, som undersøger performance af systemer til risikovurdering af patienter som får foretaget en koronararterie bypass, påvist et areal under kurven på over 0,80 for begge systemer~\citep{VanStraten2010}, hvilket er ligeværdigt med systemet til risikovurdering af lægemiddelskift.

Det kan diskuteres, hvorvidt studierne er sammenlignelige med systemet til risikovurdering af lægemiddelskift, da antallet af samples i studierne er 
højere og derved kan medvirke til en større variation i systemernes performance. Derudover er der anvendt forskellige metoder til vægtningen af risikofaktorerne, hvilket ligeledes kan have en indflydelse. Samtidig er risikovurderingen foretaget ud fra risikofaktorer, som er kendetegnet ved patienten, hvormed der kan være mange uforudsete faktorer, som kan influere i forhold til systemets output og derved afspejles af systemets performance. 

\section{Implementering og vedligeholdelse}
%Da systemet blev anvendt til at undersøge anvendeligheden af et regelbaseret system til risikovurdering af lægemiddelskift, skal det overvejes, hvad der kræves for implementeringen af et endeligt system samt hvilken vedligeholdelse systemet kræver og hvilke udfordringer der kan være relateret til dette. 

Implementering af systemet vil kræve at der bestemmes, hvilke risikofaktorer der skal anvendes samt hvordan vægtningen og gradueringen af disse fastsættes. Denne proces kan varetages ved inputs fra eksperter på området for at opnå en fælles konsensus. Derudover skal det undersøges, hvordan opmærksomheden på lægemiddelskiftene skal visualiseres for brugervenligheden ved f.eks. farveindikationer eller ved en bestemt rangorden. 

Det vil ligeledes kræve en optimering af præprocessering af data eller at data fremadrettet bliver indsamlet struktureret for at systemet kan tage højde for variationen i data, som f.eks. forkortelser ved dispenseringsform. Det kan diskuteres, hvorvidt det er muligt at indsamle data struktureret, da det ikke er SRN som står for dette. Derudover kan det overvejes om Levenshtein Distancen kan anvendes til sammenligning af ord, hvormed mindre variationer i data kan tillades. Dette kan f.eks. gøre ved at definere et antal af operationer som må være forskellige ved sammenligning.

Samtidig kræver det, at der udarbejdes en brugergrænseflade for medarbejderne før anvendelse af systemet, hvilket skal give medarbejderne mulighed for at indlæse et skifteliste for et nyt år, for derefter at gemme skiftelisten med systemets output. Derudover vil det kræve at brugeren af systemet, herunder medarbejderne fra SRN, har kendskab til system og kan se fordelene ved at anvende systemet. Samtidig skal der ved implementering af systemet tages højde for forandringer i risikovurdering af lægemiddelskift. Da systemet er et regelbaseret system kan justeringer foretages uden at dette påvirker andre dele af systemet. Dette kan imidlertid medvirke til, at systemet bliver for kompleks, hvormed det skal vurderes, hvorvidt andre metoder, som f.eks. statistiske metoder er mere hensigtsmæssige at benytte.


%Overensstemmelsen mellem systemet og medarbejdernes vurdering af lægemiddelskift er begrænset. Systemet har en højere specificitet end sensitivitet, hvilket betyder, at systemet er bedre til at identificere lægemiddelskift, som ikke kræver uddybende information sammenlignet med lægemiddelskift, som kræver uddybende information. Ligeledes har systemet en høj type 2 fejl, hvilket betyder, at systemet vurderer lægemiddelskift, hvor der kræves uddybende information til ikke at kræve uddybende information. 

%I forhold til risikoscoren blev det påvist, at medarbejdernes vurdering var mere nøjagtig sammenlignet med systemet, hvilket vil sige at medarbejderne vurdering har en højere sensitivitet og lavere specificitet end systemet. \textcolor{red}{skriv til her i forhold til diskriminationsgrænse.}

%Systemets er vurderet af medarbejdere fra SRN til at være et godt udgangspunkt for et hjælpeværktøj i forbindelse med risikovurdering af lægemiddelskift, men at risikofaktorerne og vægtningen af disse skal granuleres i forhold til at forbedre risikoscoren. Derudover blev funktioner såsom look-a-like og Medicinrådet vurderet til ikke at blive vægtet særligt højt ved lægemiddelskiftet, men at det i kombination med antallet af ændrede faktorer kunne have en betydning for lægemiddelskiftet.

%\section{Risikovurdering}
%Risikovurderingen afspejler ikke den nuværende vurdering af lægemiddelskift, hvilket kan skyldes, at der var variation mellem medarbejderne vurdering af lægemiddelskift. Dette kan være på grund af, at medarbejderne var repræsenteret fra forskellige afdelinger og derved havde forskellige vurderinger af lægemiddelskiftene set ud fra den afdeling de repræsenterede. Samtidig kan medarbejdernes vurdering være præget af erfaringer som systemet ikke tager højde for i vurderingen, hvilket kan medvirke til, at nogle lægemidler er vurderet til ikke at kræve uddybende information af medarbejderne men at systemet vurderer dette. Dette understøttes af en medarbejder, hvis generelle  kommentar var, at det var svært ikke at tage deres erfaring med i vurderingen, hvilket fremgår af Tabel \ref{table:resultat2} i Appendiks \ref{App:Resultat}. Til trods for, at dette blev gjort opmærksom på via introduktionen, der blev givet inden risikovurderingen, som fremgår af Appendiks \ref{App:Intro}. Derudover kan der være andre ændrede faktorer som skyldes, at lægemiddelskiftet, er uddybet af Lægemiddel Nyt, såsom ændret form, størrelse og farve på lægemidlet, hvilket er beskrevet i Afsnit \ref{sec:ProblemLaeg}, eller f.eks. ændring af pakningsstørrelse, som der i nogle tilfælde er uddybende kommentarer om i Lægemiddel Nyt, hvilket fremgår af Tabel \ref{table:Proces} i Afsnit \ref{sec:ImpLaeg}. I kraft af, at systemet ikke tager højde for disse ændringer kan det medvirke til, at nogle lægemiddelskift er uddybet af  Lægemiddel Nyt, men at disse ikke er vurderet af medarbejderne til at skulle uddybes, da deres vurderingen udelukkende er baseret på systemets output, herunder risikoscore og begrundelse for denne.
%
%\section{Risikoscore}
%%0-5~\%
%Medarbejderne kommenterede, at systemet i flere tilfælde vurderede et lægemiddelskift risikoscore for højt og derved skulle rangeres lavere. Dette kan skyldes, at medarbejdernes erfaring er medtaget i vurderingen i forhold til at kende skiftet og derved vil rangere dette lavere, hvilket understøttes af en medarbejder i kommentaren som fremgår af Tabel \ref{table:resultat2} i Appendiks \ref{App:Resultat}. Ligeledes blev det af flere medarbejdere pointeret at funktionerne look-a-like og Medicinråd var vægtet for højt. Årsagen til dette kan være, at  risikofaktorerne ikke tages med i den nuværende vurderingen, hvormed disse ikke prioriteres særligt højt.
%
%Systemet skelner ikke i forhold til om der er ændring i styrken eller  styrkeangivelsen, hvormed et lægemiddelskift med samme risikoscore kan være vurderet forskelligt ud fra, hvorvidt lægemiddelskiftet kræver uddybende information. Dette understøttes af kommentarer om tvivl i forhold til styrke fra flere medarbejdere, jævnfør Tabel \ref{table:resultat2} i Appendiks \ref{App:Resultat}. Dette vil kræve, at systemet ved sammenligning af ændring i styrke kombineres med, hvorvidt der er sket ændringer i pakningsstørrelse.\textcolor{red}{Tilføj i forhold til diskriminationsgrænse.}
%
%\section{Risikofaktorer og deres vægtning}
%Risikofaktorer, som look-a-like, blev vurderet af medarbejderne til ikke at have den store betydning for lægemiddelskift, hvormed vægtning af denne er for høj. Dette kan skyldes, at look-a-like funktion, udelukkende kigger i forhold til sammenligning af lægemidlets navn og ikke tager højde for om andre faktorer som, f.eks. dispenseringsform eller tepariområde, er ens for de lægemidler som er hinandens look-a-like. Derudover kan det være at look-a-like skal vægtes lavere eller udelades fra risikovurderingen, men stadig være tilgængelige i klinikken i forhold til at mindske fejlmedicinering opstået ved forvirring over sammenlignelige lægemiddelnavne, jævnfør Afsnit \ref{sec:ProblemLaeg}.
%
%Ligeledes blev Medicinrådet vurderet til ikke at skulle vægtes høj, da det at et lægemiddel indgår i behandlingsvejledningen i sig selv ikke har betydning, men at ændringer af faktorer og hvor mange der ændres har betydning. Dette kan skyldes, at Medicinrådet har en stor betydning første gang det implementeres i klinikken i forhold til at opnå besparelser eller hvis der sker ændringer, som kan forårsage fejlmedicinering, hvor betydningen efterfølgende er begrænset, hvorfor det er nødvendigt at systemet differentiere mellem dette.
%
%Det blev for navn, dispenseringsform og styrke vurderet af medarbejderne, at der ligeledes krævede granulering af ændringer i disse i forhold til vægtningen, da der var variationer i sværhedsgraden inden for hvert type af skift. Dette skyldes, at nogle skift anses som værende ligeværdige som f.eks. dispenseringsformer, der anvendes på samme måde i klinikken er af mindre betydning, sammenlignet med dispenseringsformer, som anvendes forskelligt i klinikken og derved har større betydning, hvis denne er ændret ved lægemiddelskift. Dette kræver, at der foretages en mere detaljeret præprocessering af data for at identificere ligeværdige typer af skift og samtidig have en større granuleringsgrad inden for hvert skift i forhold til vægtning af risikofaktorerne.

\chapter{Konklusion}
\textit{I dette kapitel konkluderes der på rapporten problemformulering i forhold til, hvorvidt et regelbaseret system er anvendeligt til risikovurdering af lægemiddelskift med henblik på at forbedre den nuværende vurdering af lægemiddelskift.}

Et regelbaseret system til risikovurdering af lægemiddelskift er anvendeligt i forhold til at forudsige, hvornår et lægemiddelskift kræver uddybende information ved implementering. Derudover er det påvist, at risikocoren kan anvendes til at inddele lægemiddelskift ud fra hvor meget opmærksomhed de kræver, hvilket medvirker til at effektivisere den nuværende vurdering af lægemiddelskift.
Alligevel afspejler risikovurderingen på nogle områder ikke den nuværende vurdering af lægemiddelskift, hvormed systemet kræver yderligere tilpasning.

Ved tilpasning af systemet formodes det at risikovurderingen af lægemiddelskift kan foretages på samme vilkår som den nuværende vurdering og derfor være anvendeligt. Dertil, skal systemet stadig tiltænkes som et hjælpeværktøj, hvormed medarbejdernes erfaring skal tages med i den endelige vurderingen. 

På baggrund af dette skal yderligere studier undersøge, hvordan risikofaktorer skal gradueres og vægtes for at optimere anvendeligheden af systemet og derved
forbedre den nuværende vurdering af lægemiddelskift.
