\chapter*{Abstract}
Drug substitution occur in order to save expenditures on medicine. Substitution of drugs contribute to medication errors in the clinic, which can affect the patient safety. The implementation of drug substitutions are important for reducing medication errors. Currently, risk assessment of drug substitution is performed by employees from the hospital pharmacy in region Northern Jutland. This process occurs manually and the assessment is based on experience, which makes it vulnerable and person-dependent. Therefore, the study aimed to develop a rule-based system for risk assessment of drug substitution. The applicability of the system was evaluated by 11 employees from the hospital pharmacy. A Receiver Operating Characteristic curve was used to investigate the performance of the risk assessment. Based on the result, the Area Under the Curve indicated a good test for predicting drugs in need for particular attention when implementing substitutions in the clinic. A limitation of the study was found in the use of risk factors and weights. In conclusion, the present study demonstrates, that a rule-based system is applicable for risk assessment of drug substitution, but there is still need for further investigation to clarify risk factors and weights in order to modify the system and improve the current assessment of drug substitution.  
