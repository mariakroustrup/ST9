\chapter*{Abstract}
Drug substitution occur in order to save expenditures on medicine. Substitution of drugs contribute to medications error in the clinic, which can affect the patient safety. The implementation of drug substitution is important for reducing medication errors. Current risk assessment of drug substitution is performed by employee from the hospital pharmacy in region Northern Jutland. This process occurs manually and is based on experience, which makes it vulnerable and person-dependent. Therefore, the study aimed to develop a rule-based system for risk assessment of drug substitution. The applicability of the system was evaluated by eleven employees from the hospital pharmacy and the performance of the risk assessment was investigated using Receiver Operating Characteristic. The Area Under the Curve indicated a good test for predicting drugs in need for particularly attention when implementing in the clinic. A limitation of the study was found in the used of risk factors and weights. In conclusion, the present study demonstrates, that a rule-based system is applicability for risk assessment of drug substitution, but there is need for further investigation for clarifying risk factors and weights in order to modify the system and improve the current assessment of drug substitution.  
