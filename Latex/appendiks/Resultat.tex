\chapter{Resultat af evaluering}
\vspace{-1.7cm}
\hspace{-1cm}
\begin{table}[H]
\caption{Resultat af vurderingen om lægemiddelskift krævede uddybende information via Lægemiddel Nyt. Fælles enighed er fremhævet med blå eller grøn. Uenig blandt én af  testpersonerne er fremhævet med gul. Lægemiddelskift, hvor der tilføjet ekstra kommentarer er angivet med asterisk(*) og yderligere beskrevet af Tabel \ref{table:resultat2}.}
\vspace{2mm}
\label{table:resultat}
\centering
\begin{tabular}{l|c|c|c|c|c|c|c|c|c|c|c|c|c}
\rowcolor[HTML]{C0C0C0}{\textbf{Lægemiddel}}& \multicolumn{11}{c}{\textbf{Testperson}} & \multicolumn{2}{c}{\textbf{Antal}}\\
\rowcolor[HTML]{C0C0C0}{\textbf{nummer}} & \textbf{1} &\textbf{2} & \textbf{3} & \textbf{4} & \textbf{5} & \textbf{6} & \textbf{7} & \textbf{8} & \textbf{9} & \textbf{10} & \textbf{11} & \textbf{ja} & \textbf{nej}\\ \hline
\cellcolor[HTML]{C0C0C0}{\textbf{1}}   & \cellcolor[HTML]{ECF4FF} nej	& \cellcolor[HTML]{ECF4FF}nej &	\cellcolor[HTML]{ECF4FF}nej &\cellcolor[HTML]{ECF4FF} nej	& \cellcolor[HTML]{ECF4FF}nej &	\cellcolor[HTML]{ECF4FF}nej	& \cellcolor[HTML]{ECF4FF}nej & \cellcolor[HTML]{ECF4FF}nej	& \cellcolor[HTML]{ECF4FF}nej	& \cellcolor[HTML]{ECF4FF}nej	& \cellcolor[HTML]{ECF4FF}nej & \cellcolor[HTML]{EFEFEF}0 & \cellcolor[HTML]{EFEFEF}11\\ \hline 
\cellcolor[HTML]{C0C0C0}\textbf{2}	 & \cellcolor[HTML]{ECF4FF} nej	& \cellcolor[HTML]{ECF4FF}nej &	\cellcolor[HTML]{ECF4FF}nej &\cellcolor[HTML]{ECF4FF} nej	& \cellcolor[HTML]{ECF4FF}nej &	\cellcolor[HTML]{ECF4FF}nej	& \cellcolor[HTML]{ECF4FF}nej & \cellcolor[HTML]{ECF4FF}nej	& \cellcolor[HTML]{ECF4FF}nej	& \cellcolor[HTML]{ECF4FF}nej	& \cellcolor[HTML]{ECF4FF}nej & \cellcolor[HTML]{EFEFEF}0 & \cellcolor[HTML]{EFEFEF}11\\ \hline 
\cellcolor[HTML]{C0C0C0}\textbf{3}	 & \cellcolor[HTML]{ECF4FF} nej	& \cellcolor[HTML]{ECF4FF}nej &	\cellcolor[HTML]{ECF4FF}nej &\cellcolor[HTML]{ECF4FF} nej	& \cellcolor[HTML]{ECF4FF}nej &	\cellcolor[HTML]{ECF4FF}nej	& \cellcolor[HTML]{ECF4FF}nej & \cellcolor[HTML]{ECF4FF}nej	& \cellcolor[HTML]{ECF4FF}nej	& \cellcolor[HTML]{ECF4FF}nej	& \cellcolor[HTML]{ECF4FF}nej & \cellcolor[HTML]{EFEFEF}0 & \cellcolor[HTML]{EFEFEF}11\\ \hline 
\cellcolor[HTML]{C0C0C0}\textbf{4}	 & \cellcolor[HTML]{ECF4FF} nej	& \cellcolor[HTML]{ECF4FF}nej &	\cellcolor[HTML]{ECF4FF}nej &\cellcolor[HTML]{ECF4FF} nej	& \cellcolor[HTML]{ECF4FF}nej &	\cellcolor[HTML]{ECF4FF}nej	& \cellcolor[HTML]{ECF4FF}nej & \cellcolor[HTML]{ECF4FF}nej	& \cellcolor[HTML]{ECF4FF}nej	& \cellcolor[HTML]{ECF4FF}nej	& \cellcolor[HTML]{ECF4FF}nej & \cellcolor[HTML]{EFEFEF}0 & \cellcolor[HTML]{EFEFEF}11\\ \hline 
\cellcolor[HTML]{C0C0C0}\textbf{5}	 & \cellcolor[HTML]{ECF4FF} nej	& \cellcolor[HTML]{ECF4FF}nej &	\cellcolor[HTML]{ECF4FF}nej &\cellcolor[HTML]{ECF4FF} nej	& \cellcolor[HTML]{ECF4FF}nej* &	\cellcolor[HTML]{ECF4FF}nej	& \cellcolor[HTML]{ECF4FF}nej & \cellcolor[HTML]{ECF4FF}nej	& \cellcolor[HTML]{ECF4FF}nej	& \cellcolor[HTML]{ECF4FF}nej	& \cellcolor[HTML]{ECF4FF}nej & \cellcolor[HTML]{EFEFEF}0 & \cellcolor[HTML]{EFEFEF}11\\ \hline 
\cellcolor[HTML]{C0C0C0}\textbf{6}	 & \cellcolor[HTML]{ECF4FF} nej	& \cellcolor[HTML]{ECF4FF}nej &	\cellcolor[HTML]{ECF4FF}nej &\cellcolor[HTML]{ECF4FF} nej	& \cellcolor[HTML]{ECF4FF}nej &	\cellcolor[HTML]{ECF4FF}nej	& \cellcolor[HTML]{ECF4FF}nej & \cellcolor[HTML]{ECF4FF}nej	& \cellcolor[HTML]{ECF4FF}nej	& \cellcolor[HTML]{ECF4FF}nej	& \cellcolor[HTML]{ECF4FF}nej & \cellcolor[HTML]{EFEFEF}0 & \cellcolor[HTML]{EFEFEF}11\\ \hline 
\cellcolor[HTML]{C0C0C0}\textbf{7}	 & \cellcolor[HTML]{ECF4FF} nej	& \cellcolor[HTML]{ECF4FF}nej &	\cellcolor[HTML]{ECF4FF}nej &\cellcolor[HTML]{ECF4FF} nej	& \cellcolor[HTML]{ECF4FF}nej &	\cellcolor[HTML]{ECF4FF}nej	& \cellcolor[HTML]{ECF4FF}nej & \cellcolor[HTML]{ECF4FF}nej	& \cellcolor[HTML]{ECF4FF}nej	& \cellcolor[HTML]{ECF4FF}nej	& \cellcolor[HTML]{ECF4FF}nej & \cellcolor[HTML]{EFEFEF}0 & \cellcolor[HTML]{EFEFEF}11\\ \hline 
\cellcolor[HTML]{C0C0C0}\textbf{8}	 & \cellcolor[HTML]{ECF4FF} nej	& \cellcolor[HTML]{ECF4FF}nej &	\cellcolor[HTML]{ECF4FF}nej &\cellcolor[HTML]{ECF4FF} nej	& \cellcolor[HTML]{ECF4FF}nej &	\cellcolor[HTML]{ECF4FF}nej	& \cellcolor[HTML]{ECF4FF}nej & \cellcolor[HTML]{ECF4FF}nej	& \cellcolor[HTML]{ECF4FF}nej	& \cellcolor[HTML]{ECF4FF}nej	& \cellcolor[HTML]{ECF4FF}nej & \cellcolor[HTML]{EFEFEF}0 & \cellcolor[HTML]{EFEFEF}11\\ \hline 
\cellcolor[HTML]{C0C0C0}\textbf{9}	& ja* 	& ja  &	ja*  & \cellcolor[HTML]{FFFC9E}nej   & ja  & ja  &ja*   &ja &ja &ja &ja & \cellcolor[HTML]{EFEFEF}10 & \cellcolor[HTML]{EFEFEF}1 \\ \hline
\cellcolor[HTML]{C0C0C0}\textbf{10}	&nej &nej &nej&	nej	&ja	&nej&	ja&	ja&	nej&	nej&nej & \cellcolor[HTML]{EFEFEF}3 & \cellcolor[HTML]{EFEFEF}8 \\ \hline
\cellcolor[HTML]{C0C0C0}\textbf{11}	 & \cellcolor[HTML]{ECF4FF} nej	& \cellcolor[HTML]{ECF4FF}nej &	\cellcolor[HTML]{ECF4FF}nej &\cellcolor[HTML]{ECF4FF} nej	& \cellcolor[HTML]{ECF4FF}nej &	\cellcolor[HTML]{ECF4FF}nej	& \cellcolor[HTML]{ECF4FF}nej & \cellcolor[HTML]{ECF4FF}nej	& \cellcolor[HTML]{ECF4FF}nej	& \cellcolor[HTML]{ECF4FF}nej	& \cellcolor[HTML]{ECF4FF}nej & \cellcolor[HTML]{EFEFEF}0 & \cellcolor[HTML]{EFEFEF}11\\ \hline 
\cellcolor[HTML]{C0C0C0}\textbf{12}	&ja*&	ja&	ja&	nej*&	ja&	nej&	nej&	ja&	ja&	nej&	nej & \cellcolor[HTML]{EFEFEF}6 & \cellcolor[HTML]{EFEFEF}5 \\ \hline
\cellcolor[HTML]{C0C0C0}\textbf{13}	&nej&	nej&	nej&	nej&	nej&	nej& 	ja&	ja&	nej&	nej&	nej & \cellcolor[HTML]{EFEFEF}2 & \cellcolor[HTML]{EFEFEF}9\\ \hline
\cellcolor[HTML]{C0C0C0}\textbf{14}&ja*&	ja&	ja*&	\cellcolor[HTML]{FFFC9E}nej*&	ja&	ja&	ja*&	ja&	ja&	ja&	ja &\cellcolor[HTML]{EFEFEF}10 &\cellcolor[HTML]{EFEFEF}1
\\ \hline
\cellcolor[HTML]{C0C0C0}\textbf{15}	& nej & nej & \cellcolor[HTML]{FFFC9E}ja & nej & nej & nej &	nej & nej* & nej & nej & nej & \cellcolor[HTML]{EFEFEF}1 & \cellcolor[HTML]{EFEFEF}10 \\ \hline
\cellcolor[HTML]{C0C0C0}\textbf{16}	& nej & nej & nej & nej & ja & nej & ja & ja & nej & nej & nej & \cellcolor[HTML]{EFEFEF}3 & \cellcolor[HTML]{EFEFEF}8 \\ \hline
\cellcolor[HTML]{C0C0C0}\textbf{17} & \cellcolor[HTML]{ECF4FF} nej	& \cellcolor[HTML]{ECF4FF}nej &	\cellcolor[HTML]{ECF4FF}nej &\cellcolor[HTML]{ECF4FF} nej	& \cellcolor[HTML]{ECF4FF}nej &	\cellcolor[HTML]{ECF4FF}nej	& \cellcolor[HTML]{ECF4FF}nej & \cellcolor[HTML]{ECF4FF}nej*	& \cellcolor[HTML]{ECF4FF}nej	& \cellcolor[HTML]{ECF4FF}nej	& \cellcolor[HTML]{ECF4FF}nej & \cellcolor[HTML]{EFEFEF}0 & \cellcolor[HTML]{EFEFEF}11\\ \hline 
\cellcolor[HTML]{C0C0C0}\textbf{18}	 & \cellcolor[HTML]{ECF4FF} nej	& \cellcolor[HTML]{ECF4FF}nej &	\cellcolor[HTML]{ECF4FF}nej &\cellcolor[HTML]{ECF4FF} nej	& \cellcolor[HTML]{ECF4FF}nej &	\cellcolor[HTML]{ECF4FF}nej	& \cellcolor[HTML]{ECF4FF}nej & \cellcolor[HTML]{ECF4FF}nej	& \cellcolor[HTML]{ECF4FF}nej	& \cellcolor[HTML]{ECF4FF}nej	& \cellcolor[HTML]{ECF4FF}nej & \cellcolor[HTML]{EFEFEF}0 & \cellcolor[HTML]{EFEFEF}11\\ \hline 
\cellcolor[HTML]{C0C0C0}\textbf{19}	&nej	 & ja & ja &	nej&	ja &	ja&ja*&	ja&	ja&	ja&	ja  & \cellcolor[HTML]{EFEFEF}9 & \cellcolor[HTML]{EFEFEF}2 \\ \hline
\cellcolor[HTML]{C0C0C0}\textbf{20}	&nej	&nej&nej	&nej&	nej&	nej&	nej&\cellcolor[HTML]{FFFC9E}	ja&nej&	nej&	nej & \cellcolor[HTML]{EFEFEF}1 & \cellcolor[HTML]{EFEFEF}10 \\ \hline
\cellcolor[HTML]{C0C0C0}\textbf{21}	&nej&	ja	&ja	&nej	&ja	&ja	&ja*	&ja	&ja	&nej&	ja  & \cellcolor[HTML]{EFEFEF}9 & \cellcolor[HTML]{EFEFEF}2 \\ \hline
\cellcolor[HTML]{C0C0C0}\textbf{22}	&nej*	&ja	&ja*	&nej&	ja&	ja&	ja&	ja&	ja&	nej&	ja & \cellcolor[HTML]{EFEFEF}9 & \cellcolor[HTML]{EFEFEF}2\\ \hline
\cellcolor[HTML]{C0C0C0}\textbf{23}	&nej&	nej&nej*&	nej	&nej&	nej	&nej&	\cellcolor[HTML]{FFFC9E}ja	&nej&	nej*&	nej & \cellcolor[HTML]{EFEFEF}1 & \cellcolor[HTML]{EFEFEF}10\\ \hline
\cellcolor[HTML]{C0C0C0}\textbf{24}	&ja*	&nej&nej&nej	&ja& 	nej&	nej&	ja&	nej&	nej&	nej  & \cellcolor[HTML]{EFEFEF}3 & \cellcolor[HTML]{EFEFEF}8 \\ \hline
\cellcolor[HTML]{C0C0C0}\textbf{25}	&nej&	nej&	nej&	nej&ja&	nej&	ja&	ja*&	nej&	nej&	nej & \cellcolor[HTML]{EFEFEF}3 & \cellcolor[HTML]{EFEFEF}8 \\ \hline
\cellcolor[HTML]{C0C0C0}\textbf{26}	&nej	&ja	&nej	&nej&ja	&ja	&nej	&ja*	&nej&	nej&	nej & \cellcolor[HTML]{EFEFEF}4 & \cellcolor[HTML]{EFEFEF}7 \\ \hline
\cellcolor[HTML]{C0C0C0}\textbf{27}	&ja	&nej	&nej	&nej	&ja	&nej	&ja	&ja*	&nej	&nej	&nej & \cellcolor[HTML]{EFEFEF}4 & \cellcolor[HTML]{EFEFEF}7 \\ \hline
\cellcolor[HTML]{C0C0C0}\textbf{28}	&\cellcolor[HTML]{FFFC9E}nej	&ja*	&ja*	&ja*	&ja	&ja	&ja	&ja	&ja	&ja	&ja & \cellcolor[HTML]{EFEFEF}10 & \cellcolor[HTML]{EFEFEF}1  \\ \hline
\cellcolor[HTML]{C0C0C0}\textbf{29}	&ja	&ja	&ja	&\cellcolor[HTML]{FFFC9E}nej	&ja	&ja	&ja	&ja	&ja	&ja	&ja  & \cellcolor[HTML]{EFEFEF}10 & \cellcolor[HTML]{EFEFEF}1\\ \hline
\cellcolor[HTML]{C0C0C0}\textbf{30}	&\cellcolor[HTML]{D4EED3}ja	&\cellcolor[HTML]{D4EED3}ja	&\cellcolor[HTML]{D4EED3}ja	&\cellcolor[HTML]{D4EED3}ja	&\cellcolor[HTML]{D4EED3}ja	&\cellcolor[HTML]{D4EED3}ja	&\cellcolor[HTML]{D4EED3}ja	&\cellcolor[HTML]{D4EED3}ja	&\cellcolor[HTML]{D4EED3}ja	&\cellcolor[HTML]{D4EED3}ja	&\cellcolor[HTML]{D4EED3}ja  & \cellcolor[HTML]{EFEFEF}11 & \cellcolor[HTML]{EFEFEF}0\\ \hline
\cellcolor[HTML]{C0C0C0}\textbf{31}	&ja	&ja	&nej*&ja*	&ja	&ja	&ja*	&ja	&ja	&nej&	ja  & \cellcolor[HTML]{EFEFEF}9 & \cellcolor[HTML]{EFEFEF}2  \\ \hline
\cellcolor[HTML]{C0C0C0}\textbf{32}	&\cellcolor[HTML]{D4EED3}ja	&\cellcolor[HTML]{D4EED3}ja	&\cellcolor[HTML]{D4EED3}ja	&\cellcolor[HTML]{D4EED3}ja*	&\cellcolor[HTML]{D4EED3}ja	&\cellcolor[HTML]{D4EED3}ja	&\cellcolor[HTML]{D4EED3}ja	&\cellcolor[HTML]{D4EED3}ja	&\cellcolor[HTML]{D4EED3}ja	&\cellcolor[HTML]{D4EED3}ja	&\cellcolor[HTML]{D4EED3}ja  & \cellcolor[HTML]{EFEFEF}11 & \cellcolor[HTML]{EFEFEF}0\\ \hline
\cellcolor[HTML]{C0C0C0}\textbf{33}	&nej*&	ja	&ja	&nej	&ja	&ja	&ja	&ja	&ja	&ja	&ja & \cellcolor[HTML]{EFEFEF}9 & \cellcolor[HTML]{EFEFEF}2\\ \hline%\specialrule{.2em}{.1em}{.1em} 
\rowcolor[HTML]{EFEFEF}\multicolumn{14}{r}{\textbf{Gennemsnit}}\\
\rowcolor[HTML]{EFEFEF}\textbf{Antal ja:} & 9 & 13 &	12&	4	&18&	12&	16&	21&	12&	7&	11 & \multicolumn{2}{c}{12,27}\\ \hline
\rowcolor[HTML]{EFEFEF}\textbf{Antal nej:} &24 &	20&	21&	28&	15&	21&	17&	12&	21&	26&	22 &\multicolumn{2}{c}{20,63} \\
\end{tabular}
\end{table}

\begin{table}[H]
\caption{Tilføjede kommentarer til lægemiddelskift.}
\vspace{2mm}
\label{table:resultat2}
\centering
\begin{tabular}{p{2.2cm}|p{12cm}}
\rowcolor[HTML]{C0C0C0}{\textbf{Lægemiddel nummer}} & \textbf{Kommentar} \\\hline
\cellcolor[HTML]{C0C0C0}\textbf{5} & Testperson 3: Holdbarhed \\ \hline
\cellcolor[HTML]{C0C0C0}\textbf{9}\multirow{3}{*}{} & Testperson 1: Styrken er ikke ændret, men det er angivelsen.  \\\cline{2-2}
 \cellcolor[HTML]{C0C0C0}       & Testperson 3: Styrkeændring. \\ \cline{2-2}
     \cellcolor[HTML]{C0C0C0}             &Testperson 7: Obs på om der er tale om styrke eller styrkeangivelsen \\ \hline
\cellcolor[HTML]{C0C0C0}\textbf{12}\multirow{2}{*}{} & Testperson 1: Testperson 1: Bemærkningen er ikke helt dækkende.  \\ \cline{2-2}
\cellcolor[HTML]{C0C0C0}  & Testperson 4: Hvis doseringen er ændret så ja.  \\ \hline
\cellcolor[HTML]{C0C0C0} \textbf{14} \multirow{4}{*}{} & Testperson 1: Styrken er ikke ændret, men det er angivelsen.  \\ \cline{2-2}
\cellcolor[HTML]{C0C0C0}			& Testperson 3: Ændring i styrkeangivelsen.  \\ \cline{2-2}
\cellcolor[HTML]{C0C0C0}                  & Testperson 4: Hvis pakningsstørrelse er den samme, er de ens, men blot forskelligt angivet. \\ \cline{2-2} \cellcolor[HTML]{C0C0C0} & Testperson 7: Obs på om der er tale om styrke eller styrkeangivelsen  \\ \hline        
\cellcolor[HTML]{C0C0C0}\textbf{15} & Testperson 8:  Afhænger af problemstilling  \\ \hline
\cellcolor[HTML]{C0C0C0}\textbf{17} & Testperson 8:  Afhænger af problemstilling  \\ \hline
\cellcolor[HTML]{C0C0C0}\textbf{19} & Testperson 7: Obs på om der er tale om styrke eller styrkeangivelsen \\ \hline
\cellcolor[HTML]{C0C0C0}\textbf{21} & Testperson 7: Obs på om der er tale om styrke eller styrkeangivelsen \\ \hline    
\cellcolor[HTML]{C0C0C0} \textbf{22}\multirow{2}{*}{} & Testperson 1: ATC-koden L er mere kritisk end B.  \\ \cline{2-2}
\cellcolor[HTML]{C0C0C0}          & Testperson 3: En ikke-registreret specialist. \\ \hline
\cellcolor[HTML]{C0C0C0} \textbf{23}\multirow{2}{*}{} & Testperson 3: Konserveringsemballage. \\ \cline{2-2}
\cellcolor[HTML]{C0C0C0}           & Testperson 10: Har kendskab til skiftet. \\ \hline
\cellcolor[HTML]{C0C0C0} \textbf{24} & Testperson 1: Dette lægemiddel har ofte mange uforudset problemer \\ \hline
\cellcolor[HTML]{C0C0C0} \textbf{25} & Testperson 8: Ændring i Medicinrådet \\ \hline
\cellcolor[HTML]{C0C0C0} \textbf{26} & Testperson 8: Ændring i Medicinrådet \\ \hline
\cellcolor[HTML]{C0C0C0}   \textbf{27} & Testperson 8: Ændring i Medicinrådet \\ \hline
\cellcolor[HTML]{C0C0C0} \textbf{28}\multirow{3}{*}{} & Testperson 2: Ikke Synonym. \\ \cline{2-2}
\cellcolor[HTML]{C0C0C0}      & Testperson 3: Ikke synonymskift. \\ \cline{2-2}
 \cellcolor[HTML]{C0C0C0}                    & Testperson 4: Er der forskel i dosering? \\ \hline
\cellcolor[HTML]{C0C0C0} \textbf{31}\multirow{3}{*}{} & Testperson 3: Medicinrådet er ikke nødvendigvis kritisk. \\ \cline{2-2}
\cellcolor[HTML]{C0C0C0}     & Testperson 4: Er der forskel i dosering? \\ \cline{2-2}
\cellcolor[HTML]{C0C0C0}   & Testperson 7: Obs på om der er tale om styrke eller styrkeangivelsen. \\ \hline
\cellcolor[HTML]{C0C0C0}  \textbf{32} & Testperson 4: Medmindre de skal doseres forskelligt \\ \hline
\cellcolor[HTML]{C0C0C0} \textbf{33} & Testperson 1: Ikke enig med at dette er det mest kritiske lægemiddel. \\ \hline
\cellcolor[HTML]{C0C0C0} \textbf{Generelt}\multirow{3}{*} & Testperson 2: Navneændring har ikke særlig stor betydning. Klinikken er vant til det hedder noget forskelligt. At det er i medicinrådet betyder nødvendigvis ikke at der er problematisk. Dem med ændring i styrke og dispenseringsform skal have højere score og dem med reel styrkeændring skal højere rangeres end de styrkeændringer der bare er skrevet på en anden måde f.eks. 250 mg/5ml og 50 mg/ml. Look-a-like anser jeg som udgangspunkt ikke et problem. Eneste er oxycontin og alle dens variationer, da her også er udfordringer ved dispenseringsform.  \\ \cline{2-2}
\cellcolor[HTML]{C0C0C0}    & Testperson 6: Svært ikke at tage erfaring om f.eks. afdeling med ind i vurderingen. \\ \cline{2-2}
\cellcolor[HTML]{C0C0C0}                & Testperson 9: Informere om ændret styrkeangivelse og ændret dispenseringsform, hvis det har betydning for administration. \\ 
\end{tabular}
\end{table}

\begin{table}[H]
\caption{}
\vspace{2mm}
\label{table:resultat3}
\centering
\begin{tabular}{l|c|c|c|c|c|c|p{2cm}}
\rowcolor[HTML]{C0C0C0}{\textbf{Lægemiddel}}& \multicolumn{6}{c}{\textbf{Testperson}} &  \\
\rowcolor[HTML]{C0C0C0}\textbf{nummer}& \textbf{1} & \textbf{2} & \textbf{4} & \textbf{5} & \textbf{8} & \textbf{10}  & \textbf{I alt:}\\ \hline
\cellcolor[HTML]{C0C0C0}\textbf{9} & Højere & & & & &  & \cellcolor[HTML]{EFEFEF} 1 \\ \hline
\cellcolor[HTML]{C0C0C0}\textbf{11} & Lavere & Højere & & & Lavere &  & \cellcolor[HTML]{EFEFEF}3 \\\hline
\cellcolor[HTML]{C0C0C0}\textbf{12} & & & Lavere & & & & \cellcolor[HTML]{EFEFEF}1 \\\hline
\cellcolor[HTML]{C0C0C0}\textbf{13}& Lavere  & & & & & Lavere  & \cellcolor[HTML]{EFEFEF}2  \\ \hline
\cellcolor[HTML]{C0C0C0}\textbf{14} &  & & Lavere  & & &  & \cellcolor[HTML]{EFEFEF}1 \\ \hline
\cellcolor[HTML]{C0C0C0}\textbf{15} & & & & Lavere & & Lavere  & \cellcolor[HTML]{EFEFEF}2 \\\hline
\cellcolor[HTML]{C0C0C0}\textbf{16} & & & &  & & Lavere & \cellcolor[HTML]{EFEFEF}1 \\\hline
\cellcolor[HTML]{C0C0C0}\textbf{17}& & & Lavere & Lavere & & Lavere & \cellcolor[HTML]{EFEFEF}2 \\\hline
\cellcolor[HTML]{C0C0C0}\textbf{18} & Lavere & & Lavere & Lavere & & Lavere & \cellcolor[HTML]{EFEFEF}4 \\\hline
\cellcolor[HTML]{C0C0C0}\textbf{20} & & & Lavere &  & & Lavere & \cellcolor[HTML]{EFEFEF}2 \\\hline
\cellcolor[HTML]{C0C0C0}\textbf{22} & Lavere & &  &  & & Lavere & \cellcolor[HTML]{EFEFEF} 2\\\hline
\cellcolor[HTML]{C0C0C0}\textbf{23} & Lavere & Lavere & & Lavere & & Lavere  & \cellcolor[HTML]{EFEFEF} 4\\ \hline
\cellcolor[HTML]{C0C0C0}\textbf{24} & & Lavere & & & & Lavere & \cellcolor[HTML]{EFEFEF} 2 \\\hline
\cellcolor[HTML]{C0C0C0}\textbf{25} & & & Lavere & & & Lavere & \cellcolor[HTML]{EFEFEF} 2 \\\hline
\cellcolor[HTML]{C0C0C0}\textbf{26} & Lavere & & Lavere & & & Lavere  & \cellcolor[HTML]{EFEFEF} 3\\\hline
\cellcolor[HTML]{C0C0C0}\textbf{27} & & &  &  & & Lavere  & \cellcolor[HTML]{EFEFEF} 1\\\hline
\cellcolor[HTML]{C0C0C0}\textbf{31} & & &  &  & & Lavere & \cellcolor[HTML]{EFEFEF} 1 \\\hline
\cellcolor[HTML]{C0C0C0}\textbf{32} & & & & & & Lavere & \cellcolor[HTML]{EFEFEF}1 \\\hline
\cellcolor[HTML]{C0C0C0}\textbf{33} & Lavere & & Lavere & & & & \cellcolor[HTML]{EFEFEF} 2\\\hline
\rowcolor[HTML]{EFEFEF}\multicolumn{8}{r}{\textbf{Gennemsnit}}\\
\rowcolor[HTML]{EFEFEF} \textbf{Antal}: & 8 & 3 & 8 & 4 & 1 & 14  & 6, 33\\
\end{tabular}
\end{table}

\newpage
\section{Referat af feedback og forbedringer til systemet}
Testpersonerne blev inden diskussion om feedback og forbedringer til systemet inddelt i grupper af henholdsvis 2 og 3 i forhold til at diskutere anvendeligheden af systemet, funktioner som look-a-like og Medicinrådet samt forbedringer til videreudvikling af systemet. Grupperne er illustreret af Tabel \ref{table:grupper} og den efterfølgende test angivet i kursiv er et referat af diskussion.

\begin{table}[H]
\caption{Grupper til diskussion}
\vspace{2mm}
\label{table:grupper}
\centering
\begin{tabular}{l|l}
\rowcolor[HTML]{C0C0C0} \textbf{Gruppe} & \textbf{Afdeling} \\
\textbf{1} & 2 fra Lægemiddelinformation \\ \hline
\textbf{2} \multirow{2}{*}{} & 1 fra Lægemiddelinformation \\  & 2 fra medicinservice\\ \hline
\textbf{3} & 2 fra Lægemiddelinformation \\ \hline
\textbf{4}\multirow{2}{*}{} & 1 fra Lægemiddelinformation \\  & 1 fra Klinisk Farmaci \\ \hline
\textbf{5} & 2 fra Lægemiddelinformation \\ 
\end{tabular}
\end{table}

\textit{Den første gruppe mente ikke at look-a-like skulle vægtes særligt højt, hvilket de også har fået af vide af Amgros, hvilket flere var enige i. Dertil blev det tilføjet at klinikken efterhånden er vant til at lægemidler skifter navne og det der gør lægemiddelskift komplekse mere afhænger af om der var andre faktorer, som har betydning. I forhold til Medicinrådet blev der sagt at det i sig selv nødvendigvis ikke er årsag til at det bliver kompliceret, men at det er antallet af egenskaber der skifter i denne forbindelse som holdbarhed, emballage, konserveringsmiddel som har betydning for kompleksiteten. }

\textit{En person fra den anden gruppe tilføjede til look-a-like at det kunne være en idé at afprøve med flere bogstaver, da funktionen på nuværende tidspunkt fandt sammenhænge som var for nemme. Denne person synes yderligere at det var rart, at oplysningerne kan komme af sig selv og er begejstret for værktøjet. De andre fra gruppen fra informerede om at medicinservice altid vil have forbrugende afdelinger i tankerne, når de skal vurdere kompleksiteten i lægemiddelskiftet. Det kunne derfor være en ide at koble data på forbrug. Da det kommer an på prisen i forhold til hvad det koster at skifte, hvor meget forbruges der samt hvor mange patienter der anvender lægemidlet. }

\textit{Den tredje gruppe synes at hjælpeværktøjet var fint i forhold til at kunne bruge mindre tid på helt simple skift. Dertil blev der tilføjet at hjernen selvfølgelig ikke skulle slås helt fra. I forhold til Medicinrådet gav de den tidligere gruppe ret i at, hvis der ikke er sket en ændring i Medicinrådets behandlingsvejledninger, har det måske ikke den store betydning. }

\textit{I den fjerde gruppe var de enige i at look-a-like ikke har den store værdi. I forhold til navneændring kan det være svært fra skift til skift, hvor det måske kræver at der gås mere i dybden med at graduere scoren afhængig af om det f.eks. er skift fra original til generisk. Dertil blev der tilføjet at dispenseringsform også kan differentiere eller graduere scoren, da f.eks. dispergible tabelletter og frysetørrede tabeller ikke betyder noget for klinikken. }

\textit{
Den sidste gruppe mente at systemet var et godt udgangspunkt og kunne f.eks. anvendes inden udbuddet, ved at gøre opmærksom på, hvor der er kritiske områder. Hvis f.eks. at værktøjet får inputs fra flere problemstillinger, vil dette være rigtig godt. Derudover blev der i forhold til styrke tilføjet at hvis der skiftet i styrkeangivelse, men ikke pakningsstørrelse vil dette ikke have den store betydning. Der skal derfor skelnes mellem om der er tale om styrke eller styrkeangivelse.}