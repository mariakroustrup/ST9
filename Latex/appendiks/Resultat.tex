\chapter{Resultat af evaluering} 

\section{Vurdering af lægemiddelskift} \label{App:Resultat}
Ud fra Tabel \ref{table:resultat} fremgår det at ud af alle lægemiddelskiftene var der for 13 lægemidler enighed mellem medarbejderne fra SRN. Ud af disse blev 11 lægemidler vurderet til ikke at krævede uddybende samt at 2 lægemidler krævede uddybende information ved implementering af disse i klinikken. Størstedelen af lægemiddelskift blev vurderet af 9 ud af 11 til ikke at kræve uddybende information, mens 2 medarbejdere vurderede at størstedelen af lægemidlerne krævede uddybende information, hvilket er markeret med rødt i Tabel \ref{table:resultat}. Derudover blev 12,27 lægemidler i gennemsnit, svarende til 37,19~\%, vurderet til at kræve uddybende information, mens 20,63 lægemidler, svarende til 62,5~\%, ikke krævede uddybende information, hvilket ligeledes fremgår af Tabel \ref{table:resultat}. 

\begin{longtable} {l|c|c|c|c|c|c|c|c|c|c|c|c|c}\caption{Vurderingen af lægemiddelskift i forhold til uddybende information via Lægemiddel Nyt. Fælles enighed er fremhævet med blå eller grøn. Uenig blandt én af  testpersonerne er fremhævet med gul. Lægemiddelskift,  angivet med asterisk(*), er der uddybende kommentarer til, hvilket fremgår af Tabel \ref{table:resultat2}.}
	\vspace{2mm}
	\label{table:resultat} \\
\rowcolor[HTML]{C0C0C0}{\textbf{Lægemiddel}}& \multicolumn{11}{c}{\textbf{Testperson}} & \multicolumn{2}{c}{\textbf{Antal}}\\ 
\rowcolor[HTML]{C0C0C0}{\textbf{nummer}} & \textbf{1} &\textbf{2} & \textbf{3} & \textbf{4} & \textbf{5} & \textbf{6} & \textbf{7} & \textbf{8} & \textbf{9} & \textbf{10} & \textbf{11} & \textbf{ja} & \textbf{nej}\\ \hline
\cellcolor[HTML]{C0C0C0}{\textbf{1}}   & \cellcolor[HTML]{ECF4FF} nej	& \cellcolor[HTML]{ECF4FF}nej &	\cellcolor[HTML]{ECF4FF}nej &\cellcolor[HTML]{ECF4FF} nej	& \cellcolor[HTML]{ECF4FF}nej &	\cellcolor[HTML]{ECF4FF}nej	& \cellcolor[HTML]{ECF4FF}nej & \cellcolor[HTML]{ECF4FF}nej	& \cellcolor[HTML]{ECF4FF}nej	& \cellcolor[HTML]{ECF4FF}nej	& \cellcolor[HTML]{ECF4FF}nej & \cellcolor[HTML]{EFEFEF}0 & \cellcolor[HTML]{EFEFEF}11\\ \hline 
\cellcolor[HTML]{C0C0C0}\textbf{2}	 & \cellcolor[HTML]{ECF4FF} nej	& \cellcolor[HTML]{ECF4FF}nej &	\cellcolor[HTML]{ECF4FF}nej &\cellcolor[HTML]{ECF4FF} nej	& \cellcolor[HTML]{ECF4FF}nej &	\cellcolor[HTML]{ECF4FF}nej	& \cellcolor[HTML]{ECF4FF}nej & \cellcolor[HTML]{ECF4FF}nej	& \cellcolor[HTML]{ECF4FF}nej	& \cellcolor[HTML]{ECF4FF}nej	& \cellcolor[HTML]{ECF4FF}nej & \cellcolor[HTML]{EFEFEF}0 & \cellcolor[HTML]{EFEFEF}11\\ \hline 
\cellcolor[HTML]{C0C0C0}\textbf{3}	 & \cellcolor[HTML]{ECF4FF} nej	& \cellcolor[HTML]{ECF4FF}nej &	\cellcolor[HTML]{ECF4FF}nej &\cellcolor[HTML]{ECF4FF} nej	& \cellcolor[HTML]{ECF4FF}nej &	\cellcolor[HTML]{ECF4FF}nej	& \cellcolor[HTML]{ECF4FF}nej & \cellcolor[HTML]{ECF4FF}nej	& \cellcolor[HTML]{ECF4FF}nej	& \cellcolor[HTML]{ECF4FF}nej	& \cellcolor[HTML]{ECF4FF}nej & \cellcolor[HTML]{EFEFEF}0 & \cellcolor[HTML]{EFEFEF}11\\ \hline 
\cellcolor[HTML]{C0C0C0}\textbf{4}	 & \cellcolor[HTML]{ECF4FF} nej	& \cellcolor[HTML]{ECF4FF}nej &	\cellcolor[HTML]{ECF4FF}nej &\cellcolor[HTML]{ECF4FF} nej	& \cellcolor[HTML]{ECF4FF}nej &	\cellcolor[HTML]{ECF4FF}nej	& \cellcolor[HTML]{ECF4FF}nej & \cellcolor[HTML]{ECF4FF}nej	& \cellcolor[HTML]{ECF4FF}nej	& \cellcolor[HTML]{ECF4FF}nej	& \cellcolor[HTML]{ECF4FF}nej & \cellcolor[HTML]{EFEFEF}0 & \cellcolor[HTML]{EFEFEF}11\\ \hline 
\cellcolor[HTML]{C0C0C0}\textbf{5}	 & \cellcolor[HTML]{ECF4FF} nej	& \cellcolor[HTML]{ECF4FF}nej &	\cellcolor[HTML]{ECF4FF}nej &\cellcolor[HTML]{ECF4FF} nej	& \cellcolor[HTML]{ECF4FF}nej* &	\cellcolor[HTML]{ECF4FF}nej	& \cellcolor[HTML]{ECF4FF}nej & \cellcolor[HTML]{ECF4FF}nej	& \cellcolor[HTML]{ECF4FF}nej	& \cellcolor[HTML]{ECF4FF}nej	& \cellcolor[HTML]{ECF4FF}nej & \cellcolor[HTML]{EFEFEF}0 & \cellcolor[HTML]{EFEFEF}11\\ \hline 
\cellcolor[HTML]{C0C0C0}\textbf{6}	 & \cellcolor[HTML]{ECF4FF} nej	& \cellcolor[HTML]{ECF4FF}nej &	\cellcolor[HTML]{ECF4FF}nej &\cellcolor[HTML]{ECF4FF} nej	& \cellcolor[HTML]{ECF4FF}nej &	\cellcolor[HTML]{ECF4FF}nej	& \cellcolor[HTML]{ECF4FF}nej & \cellcolor[HTML]{ECF4FF}nej	& \cellcolor[HTML]{ECF4FF}nej	& \cellcolor[HTML]{ECF4FF}nej	& \cellcolor[HTML]{ECF4FF}nej & \cellcolor[HTML]{EFEFEF}0 & \cellcolor[HTML]{EFEFEF}11\\ \hline 
\cellcolor[HTML]{C0C0C0}\textbf{7}	 & \cellcolor[HTML]{ECF4FF} nej	& \cellcolor[HTML]{ECF4FF}nej &	\cellcolor[HTML]{ECF4FF}nej &\cellcolor[HTML]{ECF4FF} nej	& \cellcolor[HTML]{ECF4FF}nej &	\cellcolor[HTML]{ECF4FF}nej	& \cellcolor[HTML]{ECF4FF}nej & \cellcolor[HTML]{ECF4FF}nej	& \cellcolor[HTML]{ECF4FF}nej	& \cellcolor[HTML]{ECF4FF}nej	& \cellcolor[HTML]{ECF4FF}nej & \cellcolor[HTML]{EFEFEF}0 & \cellcolor[HTML]{EFEFEF}11\\ \hline 
\cellcolor[HTML]{C0C0C0}\textbf{8}	 & \cellcolor[HTML]{ECF4FF} nej	& \cellcolor[HTML]{ECF4FF}nej &	\cellcolor[HTML]{ECF4FF}nej &\cellcolor[HTML]{ECF4FF} nej	& \cellcolor[HTML]{ECF4FF}nej &	\cellcolor[HTML]{ECF4FF}nej	& \cellcolor[HTML]{ECF4FF}nej & \cellcolor[HTML]{ECF4FF}nej	& \cellcolor[HTML]{ECF4FF}nej	& \cellcolor[HTML]{ECF4FF}nej	& \cellcolor[HTML]{ECF4FF}nej & \cellcolor[HTML]{EFEFEF}0 & \cellcolor[HTML]{EFEFEF}11\\ \hline 
\cellcolor[HTML]{C0C0C0}\textbf{9}	& ja* 	& ja  &	ja*  & \cellcolor[HTML]{FFFC9E}nej   & ja  & ja  &ja*   &ja &ja &ja &ja & \cellcolor[HTML]{EFEFEF}10 & \cellcolor[HTML]{EFEFEF}1 \\ \hline
\cellcolor[HTML]{C0C0C0}\textbf{10}	&nej &nej &nej&	nej	&ja	&nej&	ja&	ja&	nej&	nej&nej & \cellcolor[HTML]{EFEFEF}3 & \cellcolor[HTML]{EFEFEF}8 \\ \hline
\cellcolor[HTML]{C0C0C0}\textbf{11}	 & \cellcolor[HTML]{ECF4FF} nej	& \cellcolor[HTML]{ECF4FF}nej &	\cellcolor[HTML]{ECF4FF}nej &\cellcolor[HTML]{ECF4FF} nej	& \cellcolor[HTML]{ECF4FF}nej &	\cellcolor[HTML]{ECF4FF}nej	& \cellcolor[HTML]{ECF4FF}nej & \cellcolor[HTML]{ECF4FF}nej	& \cellcolor[HTML]{ECF4FF}nej	& \cellcolor[HTML]{ECF4FF}nej	& \cellcolor[HTML]{ECF4FF}nej & \cellcolor[HTML]{EFEFEF}0 & \cellcolor[HTML]{EFEFEF}11\\ \hline 
\cellcolor[HTML]{C0C0C0}\textbf{12}	&ja*&	ja&	ja&	nej*&	ja&	nej&	nej&	ja&	ja&	nej&	nej & \cellcolor[HTML]{EFEFEF}6 & \cellcolor[HTML]{EFEFEF}5 \\ \hline
\cellcolor[HTML]{C0C0C0}\textbf{13}	&nej&	nej&	nej&	nej&	nej&	nej& 	ja&	ja&	nej&	nej&	nej & \cellcolor[HTML]{EFEFEF}2 & \cellcolor[HTML]{EFEFEF}9\\ \hline
\cellcolor[HTML]{C0C0C0}\textbf{14}&ja*&	ja&	ja*&	\cellcolor[HTML]{FFFC9E}nej*&	ja&	ja&	ja*&	ja&	ja&	ja&	ja &\cellcolor[HTML]{EFEFEF}10 &\cellcolor[HTML]{EFEFEF}1
\\ \hline
\cellcolor[HTML]{C0C0C0}\textbf{15}	& nej & nej & \cellcolor[HTML]{FFFC9E}ja & nej & nej & nej &	nej & nej* & nej & nej & nej & \cellcolor[HTML]{EFEFEF}1 & \cellcolor[HTML]{EFEFEF}10 \\ \hline
\cellcolor[HTML]{C0C0C0}\textbf{16}	& nej & nej & nej & nej & ja & nej & ja & ja & nej & nej & nej & \cellcolor[HTML]{EFEFEF}3 & \cellcolor[HTML]{EFEFEF}8 \\ \hline
\cellcolor[HTML]{C0C0C0}\textbf{17} & \cellcolor[HTML]{ECF4FF} nej	& \cellcolor[HTML]{ECF4FF}nej &	\cellcolor[HTML]{ECF4FF}nej &\cellcolor[HTML]{ECF4FF} nej	& \cellcolor[HTML]{ECF4FF}nej &	\cellcolor[HTML]{ECF4FF}nej	& \cellcolor[HTML]{ECF4FF}nej & \cellcolor[HTML]{ECF4FF}nej*	& \cellcolor[HTML]{ECF4FF}nej	& \cellcolor[HTML]{ECF4FF}nej	& \cellcolor[HTML]{ECF4FF}nej & \cellcolor[HTML]{EFEFEF}0 & \cellcolor[HTML]{EFEFEF}11\\ \hline 
\cellcolor[HTML]{C0C0C0}\textbf{18}	 & \cellcolor[HTML]{ECF4FF} nej	& \cellcolor[HTML]{ECF4FF}nej &	\cellcolor[HTML]{ECF4FF}nej &\cellcolor[HTML]{ECF4FF} nej	& \cellcolor[HTML]{ECF4FF}nej &	\cellcolor[HTML]{ECF4FF}nej	& \cellcolor[HTML]{ECF4FF}nej & \cellcolor[HTML]{ECF4FF}nej	& \cellcolor[HTML]{ECF4FF}nej	& \cellcolor[HTML]{ECF4FF}nej	& \cellcolor[HTML]{ECF4FF}nej & \cellcolor[HTML]{EFEFEF}0 & \cellcolor[HTML]{EFEFEF}11\\ \hline 
\cellcolor[HTML]{C0C0C0}\textbf{19}	&nej	 & ja & ja &	nej&	ja &	ja&ja*&	ja&	ja&	ja&	ja  & \cellcolor[HTML]{EFEFEF}9 & \cellcolor[HTML]{EFEFEF}2 \\ \hline
\cellcolor[HTML]{C0C0C0}\textbf{20}	&nej	&nej&nej	&nej&	nej&	nej&	nej&\cellcolor[HTML]{FFFC9E}	ja&nej&	nej&	nej & \cellcolor[HTML]{EFEFEF}1 & \cellcolor[HTML]{EFEFEF}10 \\ \hline
\cellcolor[HTML]{C0C0C0}\textbf{21}	&nej&	ja	&ja	&nej	&ja	&ja	&ja*	&ja	&ja	&nej&	ja  & \cellcolor[HTML]{EFEFEF}9 & \cellcolor[HTML]{EFEFEF}2 \\ \hline
\cellcolor[HTML]{C0C0C0}\textbf{22}	&nej*	&ja	&ja*	&nej&	ja&	ja&	ja&	ja&	ja&	nej&	ja & \cellcolor[HTML]{EFEFEF}9 & \cellcolor[HTML]{EFEFEF}2\\ \hline
\cellcolor[HTML]{C0C0C0}\textbf{23}	&nej&	nej&nej*&	nej	&nej&	nej	&nej&	\cellcolor[HTML]{FFFC9E}ja	&nej&	nej*&	nej & \cellcolor[HTML]{EFEFEF}1 & \cellcolor[HTML]{EFEFEF}10\\ \hline
\cellcolor[HTML]{C0C0C0}\textbf{24}	&ja*	&nej&nej&nej	&ja& 	nej&	nej&	ja&	nej&	nej&	nej  & \cellcolor[HTML]{EFEFEF}3 & \cellcolor[HTML]{EFEFEF}8 \\ \hline
\cellcolor[HTML]{C0C0C0}\textbf{25}	&nej&	nej&	nej&	nej&ja&	nej&	ja&	ja*&	nej&	nej&	nej & \cellcolor[HTML]{EFEFEF}3 & \cellcolor[HTML]{EFEFEF}8 \\ \hline
\cellcolor[HTML]{C0C0C0}\textbf{26}	&nej	&ja	&nej	&nej&ja	&ja	&nej	&ja*	&nej&	nej&	nej & \cellcolor[HTML]{EFEFEF}4 & \cellcolor[HTML]{EFEFEF}7 \\ \hline
\cellcolor[HTML]{C0C0C0}\textbf{27}	&ja	&nej	&nej	&nej	&ja	&nej	&ja	&ja*	&nej	&nej	&nej & \cellcolor[HTML]{EFEFEF}4 & \cellcolor[HTML]{EFEFEF}7 \\ \hline
\cellcolor[HTML]{C0C0C0}\textbf{28}	&\cellcolor[HTML]{FFFC9E}nej	&ja*	&ja*	&ja*	&ja	&ja	&ja	&ja	&ja	&ja	&ja & \cellcolor[HTML]{EFEFEF}10 & \cellcolor[HTML]{EFEFEF}1  \\ \hline
\cellcolor[HTML]{C0C0C0}\textbf{29}	&ja	&ja	&ja	&\cellcolor[HTML]{FFFC9E}nej	&ja	&ja	&ja	&ja	&ja	&ja	&ja  & \cellcolor[HTML]{EFEFEF}10 & \cellcolor[HTML]{EFEFEF}1\\ \hline
\cellcolor[HTML]{C0C0C0}\textbf{30}	&\cellcolor[HTML]{D4EED3}ja	&\cellcolor[HTML]{D4EED3}ja	&\cellcolor[HTML]{D4EED3}ja	&\cellcolor[HTML]{D4EED3}ja	&\cellcolor[HTML]{D4EED3}ja	&\cellcolor[HTML]{D4EED3}ja	&\cellcolor[HTML]{D4EED3}ja	&\cellcolor[HTML]{D4EED3}ja	&\cellcolor[HTML]{D4EED3}ja	&\cellcolor[HTML]{D4EED3}ja	&\cellcolor[HTML]{D4EED3}ja  & \cellcolor[HTML]{EFEFEF}11 & \cellcolor[HTML]{EFEFEF}0\\ \hline
\cellcolor[HTML]{C0C0C0}\textbf{31}	&ja	&ja	&nej*&ja*	&ja	&ja	&ja*	&ja	&ja	&nej&	ja  & \cellcolor[HTML]{EFEFEF}9 & \cellcolor[HTML]{EFEFEF}2  \\ \hline
\cellcolor[HTML]{C0C0C0}\textbf{32}	&\cellcolor[HTML]{D4EED3}ja	&\cellcolor[HTML]{D4EED3}ja	&\cellcolor[HTML]{D4EED3}ja	&\cellcolor[HTML]{D4EED3}ja*	&\cellcolor[HTML]{D4EED3}ja	&\cellcolor[HTML]{D4EED3}ja	&\cellcolor[HTML]{D4EED3}ja	&\cellcolor[HTML]{D4EED3}ja	&\cellcolor[HTML]{D4EED3}ja	&\cellcolor[HTML]{D4EED3}ja	&\cellcolor[HTML]{D4EED3}ja  & \cellcolor[HTML]{EFEFEF}11 & \cellcolor[HTML]{EFEFEF}0\\ \hline
\cellcolor[HTML]{C0C0C0}\textbf{33}	&nej*&	ja	&ja	&nej	&ja	&ja	&ja	&ja	&ja	&ja	&ja & \cellcolor[HTML]{EFEFEF}9 & \cellcolor[HTML]{EFEFEF}2\\ \hline
\rowcolor[HTML]{EFEFEF}\multicolumn{14}{r}{\textbf{Gennemsnit}}\\
\rowcolor[HTML]{EFEFEF}\textbf{Antal ja:} & 9 & 13 &	12 &	 4 &\cellcolor[HTML]{F6E6E5}18&	12&	16&	\cellcolor[HTML]{F6E6E5}21&	12&	7&	11 & \multicolumn{2}{c}{12,27}\\ \hline
\rowcolor[HTML]{EFEFEF}\textbf{Antal nej:} &24 &	20&	21&	28&	\cellcolor[HTML]{F6E6E5}15&	21&	17&	\cellcolor[HTML]{F6E6E5}12&	21&	26&	22 &\multicolumn{2}{c}{20,63} \\
\end{longtable}

Ud af 11 medarbejdere havde 9 medarbejder kommentarer til lægemiddelskift og generelle kommentarer i forhold til risikovurderingen, hvilket fremgår af Tabel \ref{table:resultat2}. I forhold til de medarbejdere, som var uenig med de resterende var der angivet yderligere kommentarer i forhold til lægemiddelskift 14. 

\begin{longtable} {p{2.2cm}|p{12cm}}\caption{Tilføjede kommentarer til lægemiddelskift.}
	\vspace{2mm}
	\label{table:resultat2} \\
	\rowcolor[HTML]{C0C0C0}{\textbf{Lægemiddel nummer}} & \textbf{Kommentar} \\\hline
	\cellcolor[HTML]{C0C0C0}\textbf{5} & Testperson 3: Holdbarhed \\ \hline
\cellcolor[HTML]{C0C0C0}\textbf{9}\multirow{3}{*}{} & Testperson 1: Styrken er ikke ændret, men det er angivelsen.  \\\cline{2-2}
 \cellcolor[HTML]{C0C0C0}       & Testperson 3: Styrkeændring. \\ \cline{2-2}
     \cellcolor[HTML]{C0C0C0}             &Testperson 7: Obs på om der er tale om styrke eller styrkeangivelsen \\ \hline
\cellcolor[HTML]{C0C0C0}\textbf{12}\multirow{2}{*}{} & Testperson 1: Bemærkningen er ikke helt dækkende.  \\ \cline{2-2}
\cellcolor[HTML]{C0C0C0}  & Testperson 4: Hvis doseringen er ændret så ja.  \\ \hline
\cellcolor[HTML]{C0C0C0} \textbf{14} \multirow{4}{*}{} &  Testperson 1: Styrken er ikke ændret, men det er angivelsen.  \\ \cline{2-2}
\cellcolor[HTML]{C0C0C0}			& Testperson 3: Ændring i styrkeangivelsen.  \\ \cline{2-2}
\cellcolor[HTML]{C0C0C0}                  & Testperson 4: Hvis pakningsstørrelse er den samme, er de ens, men blot forskelligt angivet. \\ \cline{2-2} \cellcolor[HTML]{C0C0C0} & Testperson 7: Obs på om der er tale om styrke eller styrkeangivelsen  \\ \hline        
\cellcolor[HTML]{C0C0C0}\textbf{15} & Testperson 8:  Afhænger af problemstilling  \\ \hline
\cellcolor[HTML]{C0C0C0}\textbf{17} & Testperson 8:  Afhænger af problemstilling  \\ \hline
\cellcolor[HTML]{C0C0C0}\textbf{19} & Testperson 7: Obs på om der er tale om styrke eller styrkeangivelsen \\ \hline
\cellcolor[HTML]{C0C0C0}\textbf{21} & Testperson 7: Obs på om der er tale om styrke eller styrkeangivelsen \\ \hline    
\cellcolor[HTML]{C0C0C0} \textbf{22}\multirow{2}{*}{} & Testperson 1: ATC-koden L er mere kritisk end B.  \\ \cline{2-2}
\cellcolor[HTML]{C0C0C0}          & Testperson 3: En ikke-registreret Specialitet. \\ \hline
\cellcolor[HTML]{C0C0C0} \textbf{23}\multirow{2}{*}{} & Testperson 3: Konserveringsemballage. \\ \cline{2-2}
\cellcolor[HTML]{C0C0C0}           & Testperson 10: Har kendskab til skiftet. \\ \hline
\cellcolor[HTML]{C0C0C0} \textbf{24} & Testperson 1: Dette lægemiddel har ofte mange uforudset problemer \\ \hline
\cellcolor[HTML]{C0C0C0} \textbf{25} & Testperson 8: Ændring i Medicinrådet \\ \hline
\cellcolor[HTML]{C0C0C0} \textbf{26} & Testperson 8: Ændring i Medicinrådet \\ \hline
\cellcolor[HTML]{C0C0C0}   \textbf{27} & Testperson 8: Ændring i Medicinrådet \\ \hline
\cellcolor[HTML]{C0C0C0} \textbf{28}\multirow{3}{*}{} & Testperson 2: Ikke Synonym. \\ \cline{2-2}
\cellcolor[HTML]{C0C0C0}      & Testperson 3: Ikke synonymskift. \\ \cline{2-2}
 \cellcolor[HTML]{C0C0C0}                    & Testperson 4: Er der forskel i dosering? \\ \hline
\cellcolor[HTML]{C0C0C0} \textbf{31}\multirow{3}{*}{} & Testperson 3: Medicinrådet er ikke nødvendigvis kritisk. \\ \cline{2-2}
\cellcolor[HTML]{C0C0C0}     & Testperson 4: Er der forskel i dosering? \\ \cline{2-2}
\cellcolor[HTML]{C0C0C0}   & Testperson 7: Obs på om der er tale om styrke eller styrkeangivelsen. \\ \hline
\cellcolor[HTML]{C0C0C0}  \textbf{32} & Testperson 4: Medmindre de skal doseres forskelligt \\ \hline
\cellcolor[HTML]{C0C0C0} \textbf{33} & Testperson 1: Ikke enig med at dette er det mest kritiske lægemiddel. \\ \hline
\cellcolor[HTML]{C0C0C0} \textbf{Generelt}\multirow{3}{*} & Testperson 2: Navneændring har ikke særlig stor betydning. Klinikken er vant til det hedder noget forskelligt. At det er i medicinrådet betyder nødvendigvis ikke at der er problematisk. Dem med ændring i styrke og dispenseringsform skal have højere score og dem med reel styrkeændring skal højere rangeres end de styrkeændringer der bare er skrevet på en anden måde f.eks. 250 mg/5ml og 50 mg/ml. Look-a-like anser jeg som udgangspunkt ikke et problem. Eneste er oxycontin og alle dens variationer, da her også er udfordringer ved dispenseringsform.  \\ \cline{2-2}
\cellcolor[HTML]{C0C0C0}    & Testperson 6: Svært ikke at tage erfaring om f.eks. afdeling med ind i vurderingen. \\ \cline{2-2}
\cellcolor[HTML]{C0C0C0}                & Testperson 9: Informere om ændret styrkeangivelse og ændret dispenseringsform, hvis det har betydning for administration. \\ \hline
	\end{longtable}

\newpage

\section{Vurdering af risikoscore} \label{App:ROC}
Til visualisering af ROC-kurven udregnes koordinater for sensitivitet og 1-specificitet, hvilket fremgår af Tabel\ref{table:app_ROC}.
\begin{table}[H]
\caption{Koordinater til ROC-kurve. *Den mindste cut-off værdi er den minimale observerede test værdi minus 1, og den største cutoff værdi er den maksimale observerede test værdi plus 1. Alle de andre cut-off værdier er gennemsnittet af to sammenhængende observerede test værdier.}
\vspace{2mm}
\label{table:app_ROC}
\centering
\begin{tabular}{p{3cm}|p{2.5cm}|p{2.5cm}|p{2.5cm}|p{2.5cm}}
\rowcolor[HTML]{C0C0C0}\textbf{Positiv, hvis} & \multicolumn{2}{|c}{\textbf{Lægemiddel Nyt}} & \multicolumn{2}{|c}{\textbf{Golden Standard}} \\
\rowcolor[HTML]{C0C0C0}{\textbf{større end eller lig med*}}& \textbf{Sensitivitet} & \textbf{1-Specificitet}  & \textbf{Sensitivitet} & \textbf{1-Specificitet}  \\ \hline
-1,00 & 1,000 & 1,000 & 1,000 & 1,000 \\ \hline
2,50 & 1,000 & 0,909 & 1,000 & 0,905 \\ \hline
7,50 & 1,000 & 0,636 & 1,000 & 0,619 \\ \hline
12,50 & 0,800 & 0,591 & 0,909 & 0,524 \\ \hline
17,50 & 0,800 & 0,455 & 0,818 & 0,429 \\ \hline
22,50 & 0,700 & 0,364 & 0,818 & 0,286 \\ \hline
27,50 & 0,600 & 0,273 & 0,636 & 0,238 \\ \hline
32,50 & 0,500 & 0,182 & 0,545 & 0,143 \\ \hline
37,50 & 0,500 & 0,045 & 0,545 & 0,000 \\ \hline
42,50 & 0,300 & 0,045 & 0,364 & 0,000 \\ \hline
47,50 & 0,200 & 0,045 & 0,273 & 0,000 \\ \hline
52,50 & 0,100 & 0,000 & 0,091 & 0,000 \\ \hline
56,00 & 0,000 & 0,000 & 0,000 & 0,000  \\ \hline
\end{tabular}
\end{table}


\section{Vurdering af rangeringen af lægemiddelskift} \label{App:Rang}
Ud af 11 medarbejdere kommenterede 6 på rangeringen af lægemiddelskift. 19 lægemiddelskift blev af medarbejderne vurderet til enten at skulle have en højere eller lavere score end systemet havde beregnet, hvilket fremgår af Tabel \ref{table:resultat3}.

\begin{table}[H]
\caption{Angivet højere eller lavere rangeringen af lægemiddelskift. Medarbejdere som ikke kommenterede på rangeringen og lægemiddelskift, som ikke blev rangeret er udeladt af tabellen.}
\vspace{2mm}
\label{table:resultat3}
\centering
\begin{tabular}{l|c|c|c|c|c|c|p{2cm}}
\rowcolor[HTML]{C0C0C0}{\textbf{Lægemiddel}}& \multicolumn{6}{c}{\textbf{Testperson}} &  \\
\rowcolor[HTML]{C0C0C0}\textbf{nummer}& \textbf{1} & \textbf{2} & \textbf{4} & \textbf{5} & \textbf{8} & \textbf{10}  & \textbf{I alt:}\\ \hline
\cellcolor[HTML]{C0C0C0}\textbf{9} & Højere & & & & &  & \cellcolor[HTML]{EFEFEF} 1 \\ \hline
\cellcolor[HTML]{C0C0C0}\textbf{11} & Lavere & Højere & & & Lavere &  & \cellcolor[HTML]{EFEFEF}3 \\\hline
\cellcolor[HTML]{C0C0C0}\textbf{12} & & & Lavere & & & & \cellcolor[HTML]{EFEFEF}1 \\\hline
\cellcolor[HTML]{C0C0C0}\textbf{13}& Lavere  & & & & & Lavere  & \cellcolor[HTML]{EFEFEF}2  \\ \hline
\cellcolor[HTML]{C0C0C0}\textbf{14} &  & & Lavere  & & &  & \cellcolor[HTML]{EFEFEF}1 \\ \hline
\cellcolor[HTML]{C0C0C0}\textbf{15} & & & & Lavere & & Lavere  & \cellcolor[HTML]{EFEFEF}2 \\\hline
\cellcolor[HTML]{C0C0C0}\textbf{16} & & & &  & & Lavere & \cellcolor[HTML]{EFEFEF}1 \\\hline
\cellcolor[HTML]{C0C0C0}\textbf{17}& & & Lavere & Lavere & & Lavere & \cellcolor[HTML]{EFEFEF}2 \\\hline
\cellcolor[HTML]{C0C0C0}\textbf{18} & Lavere & & Lavere & Lavere & & Lavere & \cellcolor[HTML]{EFEFEF}4 \\\hline
\cellcolor[HTML]{C0C0C0}\textbf{20} & & & Lavere &  & & Lavere & \cellcolor[HTML]{EFEFEF}2 \\\hline
\cellcolor[HTML]{C0C0C0}\textbf{22} & Lavere & &  &  & & Lavere & \cellcolor[HTML]{EFEFEF} 2\\\hline
\cellcolor[HTML]{C0C0C0}\textbf{23} & Lavere & Lavere & & Lavere & & Lavere  & \cellcolor[HTML]{EFEFEF} 4\\ \hline
\cellcolor[HTML]{C0C0C0}\textbf{24} & & Lavere & & & & Lavere & \cellcolor[HTML]{EFEFEF} 2 \\\hline
\cellcolor[HTML]{C0C0C0}\textbf{25} & & & Lavere & & & Lavere & \cellcolor[HTML]{EFEFEF} 2 \\\hline
\cellcolor[HTML]{C0C0C0}\textbf{26} & Lavere & & Lavere & & & Lavere  & \cellcolor[HTML]{EFEFEF} 3\\\hline
\cellcolor[HTML]{C0C0C0}\textbf{27} & & &  &  & & Lavere  & \cellcolor[HTML]{EFEFEF} 1\\\hline
\cellcolor[HTML]{C0C0C0}\textbf{31} & & &  &  & & Lavere & \cellcolor[HTML]{EFEFEF} 1 \\\hline
\cellcolor[HTML]{C0C0C0}\textbf{32} & & & & & & Lavere & \cellcolor[HTML]{EFEFEF}1 \\\hline
\cellcolor[HTML]{C0C0C0}\textbf{33} & Lavere & & Lavere & & & & \cellcolor[HTML]{EFEFEF} 2\\\hline
\rowcolor[HTML]{EFEFEF}\multicolumn{8}{r}{\textbf{Gennemsnit}}\\
\rowcolor[HTML]{EFEFEF} \textbf{Antal}: & 8 & 3 & 8 & 4 & 1 & 14  & 6, 33\\
\end{tabular}
\end{table}

\newpage
\section{Referat af feedback og forbedringer til systemet} \label{App:Referat}
Medarbejderne blev inden diskussion inddelt i grupper af henholdsvis 2 og 3 i forhold til at diskutere anvendeligheden af systemet, funktioner såsom look-a-like og Medicinrådet samt forbedringer til videreudvikling af systemet. Grupperne er illustreret af Tabel \ref{table:grupper} og den efterfølgende test angivet i kursiv er et referat af en diskussion som blev foretaget efterfølgende.

\begin{table}[H]
\caption{Grupper til diskussion}
\vspace{2mm}
\label{table:grupper}
\centering
\begin{tabular}{l|l}
\rowcolor[HTML]{C0C0C0} \textbf{Gruppe} & \textbf{Afdeling} \\
\textbf{1} & 2 fra Lægemiddelinformation \\ \hline
\textbf{2} \multirow{2}{*}{} & 1 fra Lægemiddelinformation \\  & 2 fra medicinservice\\ \hline
\textbf{3} & 2 fra Lægemiddelinformation \\ \hline
\textbf{4}\multirow{2}{*}{} & 1 fra Lægemiddelinformation \\  & 1 fra Klinisk Farmaci \\ \hline
\textbf{5} & 2 fra Lægemiddelinformation \\ 
\end{tabular}
\end{table}

\textit{Den første gruppe mente ikke at look-a-like skulle vægtes særligt højt, hvilket de også har fået af vide af Amgros, hvilket flere var enige i. Dertil blev det tilføjet, at klinikken efterhånden er vant til, at lægemidler skifter navne og det der gør lægemiddelskift komplekse mere afhænger af om der er andre faktorer, som har betydning. I forhold til Medicinrådet blev der sagt at det i sig selv nødvendigvis ikke er årsag til at det bliver kompliceret, men at det er hvilke og hvor mange faktorer, der ændres i denne forbindelse såsom holdbarhed, emballage, konserveringsmiddel, som har betydning for kompleksiteten.}

\textit{En person fra den anden gruppe tilføjede til look-a-like, at det kunne være en idé at afprøve med flere ændrede bogstaver i vurderingen af look-a-like, da sammenhængene på nuværende tidspunkt er for lette. Denne person synes yderligere, at det var rart, at oplysningerne kunne komme af sig selv og var begejstret for værktøjet. De andre fra gruppen informerede om at Medicinservice altid vil have forbrugende afdelinger i tankerne, når de skal vurdere kompleksiteten i lægemiddelskiftet. Da det kommer an på prisen i forhold til, hvad det koster, at skifte, hvor meget der forbruges samt hvor mange patienter der anvender lægemidlet. Det kunne derfor være en ide at koble data om forbrug i risikovurderingen.}

\textit{Den tredje gruppe synes, at hjælpeværktøjet var fint i forhold til at kunne bruge mindre tid på helt simple skift. Dertil blev der tilføjet, at hjernen selvfølgelig ikke skulle slås helt fra. I forhold til Medicinrådet gav de den tidligere gruppe ret i, at hvis der ikke var sket en ændring i Medicinrådets behandlingsvejledninger, har det måske ikke den store betydning.}

\textit{I den fjerde gruppe var de enige i at look-a-like ikke har den store værdi i forhold til systemet. I forhold til navneændring mente denne gruppe at det kunne variere i sværhedsgrad fra skift til skift, hvorved det måske kræver at der gås mere i dybden med at graduere scoren efter typen af navneskift, for eksempel, hvis det er skift fra original til generisk. Dertil blev der tilføjet at dispenseringsform også kan differentieres eller gradueres i forhold til risikoscoren, da f.eks. skift mellem dispergible tabeletter og frysetørrede tabeletter ikke betyder noget for klinikken.}

\textit{
Den sidste gruppe mente, at systemet var et godt udgangspunkt og kunne f.eks. anvendes inden udbuddet, ved at gøre opmærksom på, hvor der er kritiske områder. Hvis f.eks. at værktøjet får inputs fra flere problemstillinger, vil dette være rigtig godt. Derudover blev der i forhold til styrke tilføjet at, hvis der skiftes i styrkeangivelse, men ikke pakningsstørrelse vil dette ikke have den store betydning. Der skal derfor skelnes mellem om der er tale om styrke eller styrkeangivelse.}