\chapter{Resultat af evaluering} 
\vspace{-0.5cm}
\section{Medarbejdernes vurdering af lægemiddelskift} \label{App:Resultat}
Medarbejdernes individuelle vurderinger af lægemiddelskift i forhold til, hvornår lægemiddelskift krævede uddybende information fremgår af Tabel \ref{table:resultat}.
\begin{longtable} {|l|c|c|c|c|c|c|c|c|c|c|c|c|c|}
\caption{Vurderingen af lægemiddelskift i forhold til at kræve uddybende information.  En vurdering til nej betyder, at et lægemiddelskift ikke kræver uddybende information, og en vurdering til ja betyder, at et lægemiddelskift kræver uddybende information. Fælles enighed er fremhævet med blå. Uenig blandt én af testpersonerne er fremhævet med grøn. Lægemiddelskift,  angivet med asterisk(*), er der kommentarer til, hvilket fremgår af Tabel \ref{table:resultat2}.}
	\label{table:resultat} \\ \hline 
\rowcolor[HTML]{C0C0C0}{\textbf{Lægemiddel}}& \multicolumn{11}{c}{\textbf{Testperson}} & \multicolumn{2}{|c|}{\textbf{Antal}}\\ 
\rowcolor[HTML]{C0C0C0}{\textbf{nummer og }}& \multicolumn{11}{c}{\textbf{}} & \multicolumn{2}{|c|}{\textbf{}}\\
\rowcolor[HTML]{C0C0C0}{\textbf{risikoscore}} & \textbf{1} &\textbf{2} & \textbf{3} & \textbf{4} & \textbf{5} & \textbf{6} & \textbf{7} & \textbf{8} & \textbf{9} & \textbf{10} & \textbf{11} & \textbf{ja} & \textbf{nej}\\ \hline
\cellcolor[HTML]{C0C0C0}{\textbf{1    (0~\%)}}   & \cellcolor[HTML]{34CDF9} nej	& \cellcolor[HTML]{34CDF9}nej &	\cellcolor[HTML]{34CDF9}nej &\cellcolor[HTML]{34CDF9} nej	& \cellcolor[HTML]{34CDF9}nej &	\cellcolor[HTML]{34CDF9}nej	& \cellcolor[HTML]{34CDF9}nej & \cellcolor[HTML]{34CDF9}nej	& \cellcolor[HTML]{34CDF9}nej	& \cellcolor[HTML]{34CDF9}nej	& \cellcolor[HTML]{34CDF9}nej & \cellcolor[HTML]{EFEFEF}0 & \cellcolor[HTML]{EFEFEF}11\\ \hline 
\cellcolor[HTML]{C0C0C0}\textbf{2    (0~\%)}	& \cellcolor[HTML]{34CDF9} nej	& \cellcolor[HTML]{34CDF9}nej &	\cellcolor[HTML]{34CDF9}nej &\cellcolor[HTML]{34CDF9} nej	& \cellcolor[HTML]{34CDF9}nej &	\cellcolor[HTML]{34CDF9}nej	& \cellcolor[HTML]{34CDF9}nej & \cellcolor[HTML]{34CDF9}nej	& \cellcolor[HTML]{34CDF9}nej	& \cellcolor[HTML]{34CDF9}nej	& \cellcolor[HTML]{34CDF9}nej & \cellcolor[HTML]{EFEFEF}0 & \cellcolor[HTML]{EFEFEF}11\\ \hline 
\cellcolor[HTML]{C0C0C0}\textbf{3   (5~\%)}	 & \cellcolor[HTML]{34CDF9} nej	& \cellcolor[HTML]{34CDF9}nej &	\cellcolor[HTML]{34CDF9}nej &\cellcolor[HTML]{34CDF9} nej	& \cellcolor[HTML]{34CDF9}nej &	\cellcolor[HTML]{34CDF9}nej	& \cellcolor[HTML]{34CDF9}nej & \cellcolor[HTML]{34CDF9}nej	& \cellcolor[HTML]{34CDF9}nej	& \cellcolor[HTML]{34CDF9}nej	& \cellcolor[HTML]{34CDF9}nej & \cellcolor[HTML]{EFEFEF}0 & \cellcolor[HTML]{EFEFEF}11\\ \hline 
\cellcolor[HTML]{C0C0C0}\textbf{4    (5~\%)}	 & \cellcolor[HTML]{34CDF9} nej	& \cellcolor[HTML]{34CDF9}nej &	\cellcolor[HTML]{34CDF9}nej &\cellcolor[HTML]{34CDF9} nej	& \cellcolor[HTML]{34CDF9}nej &	\cellcolor[HTML]{34CDF9}nej	& \cellcolor[HTML]{34CDF9}nej & \cellcolor[HTML]{34CDF9}nej	& \cellcolor[HTML]{34CDF9}nej	& \cellcolor[HTML]{34CDF9}nej	& \cellcolor[HTML]{34CDF9}nej & \cellcolor[HTML]{EFEFEF}0 & \cellcolor[HTML]{EFEFEF}11\\ \hline 
\cellcolor[HTML]{C0C0C0}\textbf{5    (5~\%)}	 & \cellcolor[HTML]{34CDF9} nej	& \cellcolor[HTML]{34CDF9}nej &	\cellcolor[HTML]{34CDF9}nej &\cellcolor[HTML]{34CDF9} nej	& \cellcolor[HTML]{34CDF9}nej &	\cellcolor[HTML]{34CDF9}nej	& \cellcolor[HTML]{34CDF9}nej & \cellcolor[HTML]{34CDF9}nej	& \cellcolor[HTML]{34CDF9}nej	& \cellcolor[HTML]{34CDF9}nej	& \cellcolor[HTML]{34CDF9}nej & \cellcolor[HTML]{EFEFEF}0 & \cellcolor[HTML]{EFEFEF}11\\ \hline 
\cellcolor[HTML]{C0C0C0}\textbf{6    (5~\%)}	 & \cellcolor[HTML]{34CDF9} nej	& \cellcolor[HTML]{34CDF9}nej &	\cellcolor[HTML]{34CDF9}nej &\cellcolor[HTML]{34CDF9} nej	& \cellcolor[HTML]{34CDF9}nej &	\cellcolor[HTML]{34CDF9}nej	& \cellcolor[HTML]{34CDF9}nej & \cellcolor[HTML]{34CDF9}nej	& \cellcolor[HTML]{34CDF9}nej	& \cellcolor[HTML]{34CDF9}nej	& \cellcolor[HTML]{34CDF9}nej & \cellcolor[HTML]{EFEFEF}0 & \cellcolor[HTML]{EFEFEF}11\\ \hline 
\cellcolor[HTML]{C0C0C0}\textbf{7    (5~\%)}	 & \cellcolor[HTML]{34CDF9} nej	& \cellcolor[HTML]{34CDF9}nej &	\cellcolor[HTML]{34CDF9}nej &\cellcolor[HTML]{34CDF9} nej	& \cellcolor[HTML]{34CDF9}nej &	\cellcolor[HTML]{34CDF9}nej	& \cellcolor[HTML]{34CDF9}nej & \cellcolor[HTML]{34CDF9}nej	& \cellcolor[HTML]{34CDF9}nej	& \cellcolor[HTML]{34CDF9}nej	& \cellcolor[HTML]{34CDF9}nej & \cellcolor[HTML]{EFEFEF}0 & \cellcolor[HTML]{EFEFEF}11\\ \hline 
\cellcolor[HTML]{C0C0C0}\textbf{8    (5~\%)}	& \cellcolor[HTML]{34CDF9} nej	& \cellcolor[HTML]{34CDF9}nej &	\cellcolor[HTML]{34CDF9}nej &\cellcolor[HTML]{34CDF9} nej	& \cellcolor[HTML]{34CDF9}nej &	\cellcolor[HTML]{34CDF9}nej	& \cellcolor[HTML]{34CDF9}nej & \cellcolor[HTML]{34CDF9}nej	& \cellcolor[HTML]{34CDF9}nej	& \cellcolor[HTML]{34CDF9}nej	& \cellcolor[HTML]{34CDF9}nej & \cellcolor[HTML]{EFEFEF}0 & \cellcolor[HTML]{EFEFEF}11\\ \hline 
\cellcolor[HTML]{C0C0C0}\textbf{9    (10~\%)}	& ja* 	& ja  &	ja*  & \cellcolor[HTML]{32CB00}nej   & ja  & ja  &ja*   &ja &ja &ja &ja & \cellcolor[HTML]{EFEFEF}10 & \cellcolor[HTML]{EFEFEF}1 \\ \hline
\cellcolor[HTML]{C0C0C0}\textbf{10 (10~\%)}	&nej &nej &nej&	nej	&ja	&nej&	ja&	ja&	nej&	nej&nej & \cellcolor[HTML]{EFEFEF}3 & \cellcolor[HTML]{EFEFEF}8 \\ \hline
\cellcolor[HTML]{C0C0C0}\textbf{11   (10~\%)}	 & \cellcolor[HTML]{34CDF9} nej	& \cellcolor[HTML]{34CDF9}nej &	\cellcolor[HTML]{34CDF9}nej &\cellcolor[HTML]{34CDF9} nej	& \cellcolor[HTML]{34CDF9}nej &	\cellcolor[HTML]{34CDF9}nej	& \cellcolor[HTML]{34CDF9}nej & \cellcolor[HTML]{34CDF9}nej	& \cellcolor[HTML]{34CDF9}nej	& \cellcolor[HTML]{34CDF9}nej	& \cellcolor[HTML]{34CDF9}nej & \cellcolor[HTML]{EFEFEF}0 & \cellcolor[HTML]{EFEFEF}11\\ \hline 
\cellcolor[HTML]{C0C0C0}\textbf{12 (15~\%)}	&ja*&	ja&	ja&	nej*&	ja&	nej&	nej&	ja&	ja&	nej&	nej & \cellcolor[HTML]{EFEFEF}6 & \cellcolor[HTML]{EFEFEF}5 \\ \hline
\cellcolor[HTML]{C0C0C0}\textbf{13 (15~\%)}	&nej&	nej&	nej&	nej&	nej&	nej& 	ja&	ja&	nej&	nej&	nej & \cellcolor[HTML]{EFEFEF}2 & \cellcolor[HTML]{EFEFEF}9\\ \hline
\cellcolor[HTML]{C0C0C0}\textbf{14 (15~\%)}&ja*&	ja&	ja*& \cellcolor[HTML]{32CB00}nej*&	ja&	ja&	ja*&	ja&	ja&	ja&	ja &\cellcolor[HTML]{EFEFEF}10 &\cellcolor[HTML]{EFEFEF}1
\\ \hline
\cellcolor[HTML]{C0C0C0}\textbf{15 (15~\%)}	& nej* & nej & \cellcolor[HTML]{32CB00}ja & nej & nej & nej &	nej & nej* & nej & nej & nej & \cellcolor[HTML]{EFEFEF}1 & \cellcolor[HTML]{EFEFEF}10 \\ \hline
\cellcolor[HTML]{C0C0C0}\textbf{16 (20~\%)}	& nej & nej & nej & nej & ja & nej & ja & ja & nej & nej & nej & \cellcolor[HTML]{EFEFEF}3 & \cellcolor[HTML]{EFEFEF}8 \\ \hline
\cellcolor[HTML]{C0C0C0}\textbf{17 (20~\%)} & \cellcolor[HTML]{34CDF9} nej	& \cellcolor[HTML]{34CDF9}nej &	\cellcolor[HTML]{34CDF9}nej &\cellcolor[HTML]{34CDF9} nej	& \cellcolor[HTML]{34CDF9}nej &	\cellcolor[HTML]{34CDF9}nej	& \cellcolor[HTML]{34CDF9}nej & \cellcolor[HTML]{34CDF9}nej	& \cellcolor[HTML]{34CDF9}nej	& \cellcolor[HTML]{34CDF9}nej	& \cellcolor[HTML]{34CDF9}nej & \cellcolor[HTML]{EFEFEF}0 & \cellcolor[HTML]{EFEFEF}11\\ \hline 
\cellcolor[HTML]{C0C0C0}\textbf{18 (20~\%)}	& \cellcolor[HTML]{34CDF9} nej	& \cellcolor[HTML]{34CDF9}nej &	\cellcolor[HTML]{34CDF9}nej &\cellcolor[HTML]{34CDF9} nej	& \cellcolor[HTML]{34CDF9}nej &	\cellcolor[HTML]{34CDF9}nej	& \cellcolor[HTML]{34CDF9}nej & \cellcolor[HTML]{34CDF9}nej	& \cellcolor[HTML]{34CDF9}nej	& \cellcolor[HTML]{34CDF9}nej	& \cellcolor[HTML]{34CDF9}nej & \cellcolor[HTML]{EFEFEF}0 & \cellcolor[HTML]{EFEFEF}11\\ \hline 
\cellcolor[HTML]{C0C0C0}\textbf{19 (25~\%)}	&nej	 & ja & ja &	nej&	ja &	ja&ja*&	ja&	ja&	ja&	ja  & \cellcolor[HTML]{EFEFEF}9 & \cellcolor[HTML]{EFEFEF}2 \\ \hline
\cellcolor[HTML]{C0C0C0}\textbf{20 (25~\%)}	&nej	&nej&nej	&nej&	nej&	nej&	nej&\cellcolor[HTML]{32CB00}	ja&nej&	nej&	nej & \cellcolor[HTML]{EFEFEF}1 & \cellcolor[HTML]{EFEFEF}10 \\ \hline
\cellcolor[HTML]{C0C0C0}\textbf{21 (25~\%)}	&nej&	ja	&ja	&nej	&ja	&ja	&ja*	&ja	&ja	&nej&	ja  & \cellcolor[HTML]{EFEFEF}9 & \cellcolor[HTML]{EFEFEF}2 \\ \hline
\cellcolor[HTML]{C0C0C0}\textbf{22 (30~\%)}	&nej*	&ja	&ja*	&nej&	ja&	ja&	ja&	ja&	ja&	nej&	ja & \cellcolor[HTML]{EFEFEF}9 & \cellcolor[HTML]{EFEFEF}2\\ \hline
\cellcolor[HTML]{C0C0C0}\textbf{23 (30~\%)}	&nej&	nej&nej*&	nej	&nej&	nej	&nej&	\cellcolor[HTML]{32CB00}ja	&nej&	nej*&	nej & \cellcolor[HTML]{EFEFEF}1 & \cellcolor[HTML]{EFEFEF}10\\ \hline
\cellcolor[HTML]{C0C0C0}\textbf{24 (30~\%)}	&ja*	&nej&nej&nej	&ja& 	nej&	nej&	ja&	nej&	nej&	nej  & \cellcolor[HTML]{EFEFEF}3 & \cellcolor[HTML]{EFEFEF}8 \\ \hline
\cellcolor[HTML]{C0C0C0}\textbf{25 (35~\%)}	&nej&	nej&	nej&	nej&ja&	nej&	ja&	ja*&	nej&	nej&	nej & \cellcolor[HTML]{EFEFEF}3 & \cellcolor[HTML]{EFEFEF}8 \\ \hline
\cellcolor[HTML]{C0C0C0}\textbf{26 (35~\%)}	&nej	&ja	&nej	&nej&ja	&ja	&nej	&ja*	&nej&	nej&	nej & \cellcolor[HTML]{EFEFEF}4 & \cellcolor[HTML]{EFEFEF}7 \\ \hline
\cellcolor[HTML]{C0C0C0}\textbf{27 (35~\%)}	&ja	&nej	&nej	&nej	&ja	&nej	&ja	&ja*	&nej	&nej	&nej & \cellcolor[HTML]{EFEFEF}4 & \cellcolor[HTML]{EFEFEF}7 \\ \hline
\cellcolor[HTML]{C0C0C0}\textbf{28 (40~\%)}	&\cellcolor[HTML]{32CB00}nej	&ja*	&ja*	&ja*	&ja	&ja	&ja	&ja	&ja	&ja	&ja & \cellcolor[HTML]{EFEFEF}10 & \cellcolor[HTML]{EFEFEF}1  \\ \hline
\cellcolor[HTML]{C0C0C0}\textbf{29 (40~\%)}	&ja	&ja	&ja	&\cellcolor[HTML]{32CB00}nej	&ja	&ja	&ja	&ja	&ja	&ja	&ja  & \cellcolor[HTML]{EFEFEF}10 & \cellcolor[HTML]{EFEFEF}1\\ \hline
\cellcolor[HTML]{C0C0C0}\textbf{30 (45~\%)}	&\cellcolor[HTML]{34CDF9}ja	&\cellcolor[HTML]{34CDF9}ja	&\cellcolor[HTML]{34CDF9}ja	&\cellcolor[HTML]{34CDF9}ja	&\cellcolor[HTML]{34CDF9}ja	&\cellcolor[HTML]{34CDF9}ja	&\cellcolor[HTML]{34CDF9}ja	&\cellcolor[HTML]{34CDF9}ja	&\cellcolor[HTML]{34CDF9}ja	&\cellcolor[HTML]{34CDF9}ja	&\cellcolor[HTML]{34CDF9}ja  & \cellcolor[HTML]{EFEFEF}11 & \cellcolor[HTML]{EFEFEF}0\\ \hline
\cellcolor[HTML]{C0C0C0}\textbf{31 (50~\%)}	&ja	&ja	&nej*&ja*	&ja	&ja	&ja*	&ja	&ja	&nej&	ja  & \cellcolor[HTML]{EFEFEF}9 & \cellcolor[HTML]{EFEFEF}2  \\ \hline
\cellcolor[HTML]{C0C0C0}\textbf{32 (50~\%)}	&\cellcolor[HTML]{34CDF9}ja	&\cellcolor[HTML]{34CDF9}ja	&\cellcolor[HTML]{34CDF9}ja	&\cellcolor[HTML]{34CDF9}ja	&\cellcolor[HTML]{34CDF9}ja	&\cellcolor[HTML]{34CDF9}ja	&\cellcolor[HTML]{34CDF9}ja	&\cellcolor[HTML]{34CDF9}ja	&\cellcolor[HTML]{34CDF9}ja	&\cellcolor[HTML]{34CDF9}ja	&\cellcolor[HTML]{34CDF9}ja  & \cellcolor[HTML]{EFEFEF}11 & \cellcolor[HTML]{EFEFEF}0\\ \hline
\cellcolor[HTML]{C0C0C0}\textbf{33 (55~\%)}	&nej*&	ja	&ja	&nej	&ja	&ja	&ja	&ja	&ja	&ja	&ja & \cellcolor[HTML]{EFEFEF}9 & \cellcolor[HTML]{EFEFEF}2\\ \hline
\rowcolor[HTML]{EFEFEF}\multicolumn{14}{|r|}{\textbf{Gennemsnit:}}\\
\rowcolor[HTML]{EFEFEF}\textbf{Antal ja:} & 9 & 13 &	12 &	 4 &18&	12&	16&	21&	12&	7&	11 & \multicolumn{2}{|c|}{12,27}\\ \hline
\rowcolor[HTML]{EFEFEF}\textbf{Antal nej:} &24 &	20&	21&	28&	15&	21&	17&	12&	21&	26&	22 &\multicolumn{2}{|c|}{20,63} \\ \hline
\end{longtable}

Ud fra Tabel \ref{table:resultat} fremgår det, at ud af alle lægemiddelskiftene var der for 13 lægemidler enighed mellem medarbejderne. Ud af disse blev 11 lægemiddelskift vurderet til ikke at krævede uddybende samt at to lægemidler krævede uddybende information ved
implementering af disse i klinikken. Størstedelen af lægemiddelskift blev vurderet af 9 ud af 11 medarbejdere til ikke at kræve uddybende information, mens to medarbejdere vurderede at størstedelen af lægemidlerne krævede uddybende information.Derudover blev 12,27 lægemidler i gennemsnit, svarende til 37,19 \%, vurderet til at kræve uddybende information, mens 20,63 lægemidler, svarende til 62,5 \%, ikke krævede uddybende information, hvilket ligeledes fremgår af Tabel D.1.

Ud af 11 medarbejdere havde 9 medarbejder kommentarer til lægemiddelskift og generelle kommentarer i forhold til risikovurderingen, hvilket fremgår af Tabel \ref{table:resultat2}. I forhold til de medarbejdere, som var uenig med de resterende var der angivet yderligere kommentarer i forhold til lægemiddelskift 14. 

\begin{longtable} {|p{2.2cm}|p{12cm}|}\caption{Kommentarer fra medarbejderne ved vurdering af lægemiddelskift. Lægemiddelskift, hvor der ikke var tilføjet kommentarer er udeladt af tabellen.}
	\label{table:resultat2} \\ \hline
	\rowcolor[HTML]{C0C0C0}{\textbf{Lægemiddel nummer}} & \textbf{Kommentar} \\\hline
	\cellcolor[HTML]{C0C0C0}\textbf{5} & Testperson 3: Holdbarhed. \\ \hline
\cellcolor[HTML]{C0C0C0}\textbf{9}\multirow{3}{*}{} & Testperson 1: Styrken er ikke ændret, men det er angivelsen.  \\\cline{2-2}
 \cellcolor[HTML]{C0C0C0}       & Testperson 3: Styrkeændring. \\ \cline{2-2}
     \cellcolor[HTML]{C0C0C0}             &Testperson 7: Obs på om der er tale om styrke eller styrkeangivelsen. \\ \hline
\cellcolor[HTML]{C0C0C0}\textbf{12}\multirow{2}{*}{} & Testperson 1: Bemærkningen er ikke helt dækkende.  \\ \cline{2-2}
\cellcolor[HTML]{C0C0C0}  & Testperson 4: Hvis doseringen er ændret så ja.  \\ \hline
\cellcolor[HTML]{C0C0C0}\textbf{14} \multirow{4}{*}{} &  Testperson 1: Styrken er ikke ændret, men det er angivelsen.  \\ \cline{2-2}
\cellcolor[HTML]{C0C0C0}			& Testperson 3: Ændring i styrkeangivelsen.  \\ \cline{2-2}
\cellcolor[HTML]{C0C0C0}                  & Testperson 4: Hvis pakningsstørrelse er den samme, er de ens, men blot forskelligt angivet. \\ \cline{2-2} \cellcolor[HTML]{C0C0C0} & Testperson 7: Obs på om der er tale om styrke eller styrkeangivelsen.  \\ \hline        
\cellcolor[HTML]{C0C0C0}\textbf{15}   & Testperson 8:  Afhænger af problemstilling.  \\ \hline
\cellcolor[HTML]{C0C0C0}\textbf{17} & Testperson 8:  Afhænger af problemstilling.  \\ \hline
\cellcolor[HTML]{C0C0C0}\textbf{18} & Testperson 1: ATC-koden L er mere kritisk end B. \\ \hline 
\cellcolor[HTML]{C0C0C0}\textbf{19}\multirow{2}{*}{} & Testperson 7: Obs på om der er tale om styrke eller styrkeangivelsen. \\ \cline{2-2}
\cellcolor[HTML]{C0C0C0}  & Testperson 9: Blot information om ændret styrkeangivelse. \\ \hline
\cellcolor[HTML]{C0C0C0}\textbf{21}\multirow{2}{*}{} & Testperson 7: Obs på om der er tale om styrke eller styrkeangivelsen. \\ \cline{2-2}
\cellcolor[HTML]{C0C0C0}  & Testperson 9: Blot information om ændret styrkeangivelse. \\ \hline
\cellcolor[HTML]{C0C0C0}\textbf{22}\multirow{2}{*}{} & Testperson 3: En ikke-registreret Specialitet. \\ \cline{2-2}
\cellcolor[HTML]{C0C0C0}  & Testperson 9: Blot information om ændret styrkeangivelse. \\ \hline
\cellcolor[HTML]{C0C0C0}\textbf{23}\multirow{2}{*}{} & Testperson 3: Konserveringsemballage. \\ \cline{2-2}
\cellcolor[HTML]{C0C0C0}           & Testperson 10: Har kendskab til skiftet. \\ \hline
\cellcolor[HTML]{C0C0C0}\textbf{24} & Testperson 1: Dette lægemiddel har ofte mange uforudsete problemer. \\ \hline
\cellcolor[HTML]{C0C0C0}\textbf{25} & Testperson 8: Ændring i Medicinrådet. \\ \hline
\cellcolor[HTML]{C0C0C0}\textbf{26} & Testperson 8: Ændring i Medicinrådet. \\ \hline
\cellcolor[HTML]{C0C0C0}\textbf{27} & Testperson 8: Ændring i Medicinrådet. \\ \hline
\cellcolor[HTML]{C0C0C0}\textbf{28}\multirow{3}{*}{} & Testperson 2: Ikke Synonym. \\ \cline{2-2}
\cellcolor[HTML]{C0C0C0}      & Testperson 3: Ikke synonymskift. \\ \cline{2-2}
 \cellcolor[HTML]{C0C0C0}                    & Testperson 4: Er der forskel i dosering? \\ \hline
\cellcolor[HTML]{C0C0C0}\textbf{31}\multirow{3}{*}{} & Testperson 3: Medicinrådet er ikke nødvendigvis kritisk. \\ \cline{2-2}
\cellcolor[HTML]{C0C0C0}     & Testperson 4: Er der forskel i dosering? \\ \cline{2-2}
\cellcolor[HTML]{C0C0C0}   & Testperson 7: Obs på om der er tale om styrke eller styrkeangivelsen. \\ \hline
\cellcolor[HTML]{C0C0C0}\textbf{32}\multirow{2}{*}{} & Testperson 4: Medmindre de skal doseres forskelligt. \\ \cline{2-2}
\cellcolor[HTML]{C0C0C0}  & Testperson 9: Synes det er relevant at informere om ændret styrkeangivelse og ændret dispenseringsform, hvis det har betydning for administration. \\ \hline
\cellcolor[HTML]{C0C0C0}\textbf{33} & Testperson 1: Ikke enig med at dette er det mest kritiske lægemiddel. \\ \hline
\cellcolor[HTML]{C0C0C0}\textbf{Generelt}\multirow{3}{*} & Testperson 2: Navneændring har ikke særlig stor betydning. Klinikken er vant til det hedder noget forskelligt. At det er i medicinrådet betyder nødvendigvis ikke at der er problematisk. Dem med ændring i styrke og dispenseringsform skal have højere score f.eks. lægemiddelskit nummer 33.  Dem med reel styrkeændring som f.eks. lægemiddelskift nummer 29 skal rangeres højere end de styrkeændringer der bare er skrevet på en anden måde f.eks. 250 mg/5ml og 50 mg/ml. Look-a-like anser jeg som udgangspunkt ikke et problem. Eneste er oxycontin og alle dens variationer, da her også er udfordringer ved dispenseringsform.  \\ \cline{2-2}
\cellcolor[HTML]{C0C0C0}    & Testperson 6: Svært ikke at tage erfaring om f.eks. afdeling med ind i vurderingen. \\ \cline{2-2}
\cellcolor[HTML]{C0C0C0}                & Testperson 9: Informere om ændret styrkeangivelse og ændret dispenseringsform, hvis det har betydning for administration. \\ \hline
	\end{longtable}

\newpage
\section{Risikoscore} \label{App:Rang}
Ud af 11 medarbejdere kommenterede 6 på rangeringen af lægemiddelskift. 19 lægemiddelskift blev af medarbejderne vurderet til enten at skulle have en højere eller lavere score end systemet, hvilket fremgår af Tabel \ref{table:resultat3}.

\begin{longtable}{|l|c|c|c|c|c|c|c|}\caption{Lægemiddelskift, hvor medarbejderne vurderede en højere eller lavere rangeringen. Medarbejdere, som ikke kommenterede på rangeringen, og lægemiddelskift, som ikke blev rangeret, er udeladt af tabellen}
\label{table:resultat3} \\ \hline
\rowcolor[HTML]{C0C0C0}{\textbf{Lægemiddel}}& \multicolumn{6}{c}{\textbf{Testperson}} &  \\
\rowcolor[HTML]{C0C0C0}\textbf{nummer}& \textbf{1} & \textbf{2} & \textbf{4} & \textbf{5} & \textbf{8} & \textbf{10}  & \textbf{I alt:}\\ \hline
\cellcolor[HTML]{C0C0C0}\textbf{9} & Højere & & & & &  & \cellcolor[HTML]{EFEFEF} 1 \\ \hline
\cellcolor[HTML]{C0C0C0}\textbf{11} & Lavere & Højere & & & Lavere &  & \cellcolor[HTML]{EFEFEF}3 \\\hline
\cellcolor[HTML]{C0C0C0}\textbf{12} & & & Lavere & & & & \cellcolor[HTML]{EFEFEF}1 \\\hline
\cellcolor[HTML]{C0C0C0}\textbf{13}& Lavere  & & & & & Lavere  & \cellcolor[HTML]{EFEFEF}2  \\ \hline
\cellcolor[HTML]{C0C0C0}\textbf{14} &  & & Lavere  & & &  & \cellcolor[HTML]{EFEFEF}1 \\ \hline
\cellcolor[HTML]{C0C0C0}\textbf{15} & & & & Lavere & & Lavere  & \cellcolor[HTML]{EFEFEF}2 \\\hline
\cellcolor[HTML]{C0C0C0}\textbf{16} & & & &  & & Lavere & \cellcolor[HTML]{EFEFEF}1 \\\hline
\cellcolor[HTML]{C0C0C0}\textbf{17}& & & Lavere & Lavere & & Lavere & \cellcolor[HTML]{EFEFEF}2 \\\hline
\cellcolor[HTML]{C0C0C0}\textbf{18} & Lavere & & Lavere & Lavere & & Lavere & \cellcolor[HTML]{EFEFEF}4 \\\hline
\cellcolor[HTML]{C0C0C0}\textbf{20} & & & Lavere &  & & Lavere & \cellcolor[HTML]{EFEFEF}2 \\\hline
\cellcolor[HTML]{C0C0C0}\textbf{22} & Lavere & &  &  & & Lavere & \cellcolor[HTML]{EFEFEF} 2\\\hline
\cellcolor[HTML]{C0C0C0}\textbf{23} & Lavere & Lavere & & Lavere & & Lavere  & \cellcolor[HTML]{EFEFEF} 4\\ \hline
\cellcolor[HTML]{C0C0C0}\textbf{24} & & Lavere & & & & Lavere & \cellcolor[HTML]{EFEFEF} 2 \\\hline
\cellcolor[HTML]{C0C0C0}\textbf{25} & & & Lavere & & & Lavere & \cellcolor[HTML]{EFEFEF} 2 \\\hline
\cellcolor[HTML]{C0C0C0}\textbf{26} & Lavere & & Lavere & & & Lavere  & \cellcolor[HTML]{EFEFEF} 3\\\hline
\cellcolor[HTML]{C0C0C0}\textbf{27} & & &  &  & & Lavere  & \cellcolor[HTML]{EFEFEF} 1\\\hline
\cellcolor[HTML]{C0C0C0}\textbf{31} & & &  &  & & Lavere & \cellcolor[HTML]{EFEFEF} 1 \\\hline
\cellcolor[HTML]{C0C0C0}\textbf{32} & & & & & & Lavere & \cellcolor[HTML]{EFEFEF}1 \\\hline
\cellcolor[HTML]{C0C0C0}\textbf{33} & Lavere & & Lavere & & & & \cellcolor[HTML]{EFEFEF} 2\\\hline
\rowcolor[HTML]{EFEFEF}\multicolumn{8}{|r|}{\textbf{Gennemsnit:}}\\ 
\rowcolor[HTML]{EFEFEF} \textbf{I alt:}: & 8 & 3 & 8 & 4 & 1 & 14  & 6, 33\\ \hline
\end{longtable}


%\begin{table}[H]
%\caption{Lægemiddelskift, hvor der er angivet højere eller lavere rangeringen af lægemiddelskift. Medarbejdere som ikke kommenterede på rangeringen og lægemiddelskift, som ikke blev rangeret er udeladt af tabellen.}
%\vspace{2mm}
%\label{table:resultat3}
%\centering
%\begin{tabular}{l|c|c|c|c|c|c|p{2cm}}
%\rowcolor[HTML]{C0C0C0}{\textbf{Lægemiddel}}& \multicolumn{6}{c}{\textbf{Testperson}} &  \\
%\rowcolor[HTML]{C0C0C0}\textbf{nummer}& \textbf{1} & \textbf{2} & \textbf{4} & \textbf{5} & \textbf{8} & \textbf{10}  & \textbf{I alt:}\\ \hline
%\cellcolor[HTML]{C0C0C0}\textbf{9} & Højere & & & & &  & \cellcolor[HTML]{EFEFEF} 1 \\ \hline
%\cellcolor[HTML]{C0C0C0}\textbf{11} & Lavere & Højere & & & Lavere &  & \cellcolor[HTML]{EFEFEF}3 \\\hline
%\cellcolor[HTML]{C0C0C0}\textbf{12} & & & Lavere & & & & \cellcolor[HTML]{EFEFEF}1 \\\hline
%\cellcolor[HTML]{C0C0C0}\textbf{13}& Lavere  & & & & & Lavere  & \cellcolor[HTML]{EFEFEF}2  \\ \hline
%\cellcolor[HTML]{C0C0C0}\textbf{14} &  & & Lavere  & & &  & \cellcolor[HTML]{EFEFEF}1 \\ \hline
%\cellcolor[HTML]{C0C0C0}\textbf{15} & & & & Lavere & & Lavere  & \cellcolor[HTML]{EFEFEF}2 \\\hline
%\cellcolor[HTML]{C0C0C0}\textbf{16} & & & &  & & Lavere & \cellcolor[HTML]{EFEFEF}1 \\\hline
%\cellcolor[HTML]{C0C0C0}\textbf{17}& & & Lavere & Lavere & & Lavere & \cellcolor[HTML]{EFEFEF}2 \\\hline
%\cellcolor[HTML]{C0C0C0}\textbf{18} & Lavere & & Lavere & Lavere & & Lavere & \cellcolor[HTML]{EFEFEF}4 \\\hline
%\cellcolor[HTML]{C0C0C0}\textbf{20} & & & Lavere &  & & Lavere & \cellcolor[HTML]{EFEFEF}2 \\\hline
%\cellcolor[HTML]{C0C0C0}\textbf{22} & Lavere & &  &  & & Lavere & \cellcolor[HTML]{EFEFEF} 2\\\hline
%\cellcolor[HTML]{C0C0C0}\textbf{23} & Lavere & Lavere & & Lavere & & Lavere  & \cellcolor[HTML]{EFEFEF} 4\\ \hline
%\cellcolor[HTML]{C0C0C0}\textbf{24} & & Lavere & & & & Lavere & \cellcolor[HTML]{EFEFEF} 2 \\\hline
%\cellcolor[HTML]{C0C0C0}\textbf{25} & & & Lavere & & & Lavere & \cellcolor[HTML]{EFEFEF} 2 \\\hline
%\cellcolor[HTML]{C0C0C0}\textbf{26} & Lavere & & Lavere & & & Lavere  & \cellcolor[HTML]{EFEFEF} 3\\\hline
%\cellcolor[HTML]{C0C0C0}\textbf{27} & & &  &  & & Lavere  & \cellcolor[HTML]{EFEFEF} 1\\\hline
%\cellcolor[HTML]{C0C0C0}\textbf{31} & & &  &  & & Lavere & \cellcolor[HTML]{EFEFEF} 1 \\\hline
%\cellcolor[HTML]{C0C0C0}\textbf{32} & & & & & & Lavere & \cellcolor[HTML]{EFEFEF}1 \\\hline
%\cellcolor[HTML]{C0C0C0}\textbf{33} & Lavere & & Lavere & & & & \cellcolor[HTML]{EFEFEF} 2\\\hline
%\rowcolor[HTML]{EFEFEF}\multicolumn{8}{r}{\textbf{Gennemsnit}}\\
%\rowcolor[HTML]{EFEFEF} \textbf{Antal}: & 8 & 3 & 8 & 4 & 1 & 14  & 6, 33\\
%\end{tabular}
%\end{table}

\subsection{Sensitivitet og specificitet} \label{App:ROC}
Til visualisering af ROC-kurven udregnes koordinater for sensitivitet og 1-specificitet, hvilket fremgår af Tabel  \ref{table:app_ROC}.

\begin{table}[H]
\caption{Koordinater til udarbejdelse af ROC-kurve. 
\scriptsize{*Den mindste cut-off værdi er den minimale observerede test værdi minus 1, og den største cutoff værdi er den maksimale observerede test værdi plus 1. Alle de andre cut-off værdier er gennemsnittet af to sammenhængende observerede test værdier.}}
\vspace{2mm}
\label{table:app_ROC}
\centering
\begin{tabular}{|p{3cm}|p{2.5cm}|p{2.5cm}|}
\rowcolor[HTML]{C0C0C0}\textbf{Positiv, hvis} & \textbf{Sensitivitet} & \textbf{1-Specificitet} \\ 
\rowcolor[HTML]{C0C0C0} \textbf{større end eller lig med*} &  &    \\ \hline
-1,00 & 1,000 & 1,000 \\ \hline
2,50	& 1,000 & 0,905 \\ \hline
7,50 & 1,000 & 0,619 \\ \hline
12,50  & 0,909 & 0,524 \\ \hline
17,50 & 0,818 & 0,429 \\ \hline
22,50  & 0,818 & 0,286 \\ \hline
27,50  & 0,636 & 0,238 \\ \hline
32,50  & 0,545 & 0,143 \\ \hline
37,50  & 0,545 & 0,000 \\ \hline
42,50  & 0,364 & 0,000 \\ \hline
47,50 & 0,273 & 0,000 \\ \hline
52,50 & 0,091 & 0,000 \\ \hline
56,00 & 0,000 & 0,000  \\ \hline
\end{tabular}
\end{table}

%\begin{table}[H]
%\caption{Koordinater til ROC-kurve. *Den mindste cut-off værdi er den minimale observerede test værdi minus 1, og den største cutoff værdi er den maksimale observerede test værdi plus 1. Alle de andre cut-off værdier er gennemsnittet af to sammenhængende observerede test værdier.}
%\vspace{2mm}
%\label{table:app_ROC}
%\centering
%\begin{tabular}{p{3cm}|p{2.5cm}|p{2.5cm}|p{2.5cm}|p{2.5cm}}
%\rowcolor[HTML]{C0C0C0}\textbf{Positiv, hvis} & \multicolumn{2}{|c}{\textbf{Lægemiddel Nyt}} & \multicolumn{2}{|c}{\textbf{Golden Standard}} \\
%\rowcolor[HTML]{C0C0C0}{\textbf{større end eller lig med*}}& \textbf{Sensitivitet} & \textbf{1-Specificitet}  & \textbf{Sensitivitet} & \textbf{1-Specificitet}  \\ \hline
%-1,00 & 1,000 & 1,000 & 1,000 & 1,000 \\ \hline
%2,50 & 1,000 & 0,913 & 1,000 & 0,905 \\ \hline
%7,50 & 1,000 & 0,652 & 1,000 & 0,619 \\ \hline
%12,50 & 0,889 & 0,565 & 0,909 & 0,524 \\ \hline
%17,50 & 0,889 & 0,435 & 0,818 & 0,429 \\ \hline
%22,50 & 0,778 & 0,348 & 0,818 & 0,286 \\ \hline
%27,50 & 0,667 & 0,261 & 0,636 & 0,238 \\ \hline
%32,50 & 0,556 & 0,174 & 0,545 & 0,143 \\ \hline
%37,50 & 0,556 & 0,043 & 0,545 & 0,000 \\ \hline
%42,50 & 0,333 & 0,043 & 0,364 & 0,000 \\ \hline
%47,50 & 0,222 & 0,043 & 0,273 & 0,000 \\ \hline
%52,50 & 0,111 & 0,000 & 0,091 & 0,000 \\ \hline
%56,00 & 0,000 & 0,000 & 0,000 & 0,000  \\ \hline
%\end{tabular}
%\end{table}


\newpage
\section{Referat af feedback og forbedringer til systemet} \label{App:Referat}
Medarbejderne blev inden diskussion inddelt i grupper af henholdsvis to og tre i forhold til at diskutere anvendeligheden af systemet, funktioner såsom look-a-like og Medicinrådet samt forbedringer til videreudvikling af systemet. Grupperne er illustreret af Tabel \ref{table:grupper} og den efterfølgende test angivet i kursiv er et referat af en diskussion som blev foretaget efterfølgende. Pointer som er anvendt til at understøtte resultater er markeret med fed skift.

\begin{table}[H]
\caption{Grupper til diskussion}
\vspace{2mm}
\label{table:grupper}
\centering
\begin{tabular}{l|l}
\rowcolor[HTML]{C0C0C0} \textbf{Gruppe} & \textbf{Afdeling} \\
\textbf{1} & To fra Lægemiddelinformation \\ \hline
\textbf{2} \multirow{2}{*}{} & En fra Lægemiddelinformation \\  & To fra  Klinisk Farmaci\\ \hline
\textbf{3} & To fra Lægemiddelinformation \\ \hline
\textbf{4}\multirow{2}{*}{} & En fra Lægemiddelinformation \\  & En fra Klinisk Farmaci \\ \hline
\textbf{5} & To fra Lægemiddelinformation \\ 
\end{tabular}
\end{table}

\textit{\textbf{Den første gruppe mente ikke at look-a-like skulle vægtes særligt højt, hvilket de også har fået af vide af Amgros, hvilket flere var enige i.} Dertil blev det tilføjet, at klinikken efterhånden er vant til, at lægemidler skifter navne og det der gør lægemiddelskift komplekse mere afhænger af om der er andre faktorer, som har betydning.\textbf{ I forhold til Medicinrådet blev der sagt at det i sig selv nødvendigvis ikke er årsag til at det bliver kompliceret, men at det er hvilke og hvor mange faktorer, der ændres i denne forbindelse såsom holdbarhed, emballage, konserveringsmiddel, som har betydning for kompleksiteten.}}

\textit{\textbf{En person fra den anden gruppe tilføjede til look-a-like, at det kunne være en idé at afprøve med flere ændrede bogstaver i vurderingen af look-a-like, da sammenhængene på nuværende tidspunkt er for lette.} Denne person synes yderligere, at det var rart, at oplysningerne kunne komme af sig selv og var begejstret for værktøjet. De andre fra gruppen informerede om at Medicinservice altid vil have forbrugende afdelinger i tankerne, når de skal vurdere kompleksiteten i lægemiddelskiftet. Da det kommer an på prisen i forhold til, hvad det koster, at skifte, hvor meget der forbruges samt hvor mange patienter der anvender lægemidlet. Det kunne derfor være en ide at koble data om forbrug i risikovurderingen.}

\textit{Den tredje gruppe synes, at hjælpeværktøjet var fint i forhold til at kunne bruge mindre tid på helt simple skift. Dertil blev der tilføjet, at hjernen selvfølgelig ikke skulle slås helt fra. \textbf{I forhold til Medicinrådet gav de den tidligere gruppe ret i, at hvis der ikke var sket en ændring i Medicinrådets behandlingsvejledninger, har det måske ikke den store betydning.}}

\textit{\textbf{I den fjerde gruppe var de enige i at look-a-like ikke har den store værdi i forhold til systemet.} I forhold til navneændring mente denne gruppe at det kunne variere i sværhedsgrad fra skift til skift, hvorved det måske kræver at der gås mere i dybden med at graduere scoren efter typen af navneskift, for eksempel, hvis det er skift fra original til generisk. \textbf{Dertil blev der tilføjet at dispenseringsform også kan differentieres eller gradueres i forhold til risikoscoren, da f.eks. skift mellem dispergible tabeletter og frysetørrede tabeletter ikke betyder noget for klinikken.}}

\textit{
Den sidste gruppe mente, at systemet var et godt udgangspunkt og kunne f.eks. anvendes inden udbuddet, ved at gøre opmærksom på, hvor der er kritiske områder. Hvis f.eks. at værktøjet får inputs fra flere problemstillinger, vil dette være rigtig godt.\textbf{Derudover blev der i forhold til styrke tilføjet at, hvis der skiftes i styrkeangivelse, men ikke i pakningsstørrelse vil dette ikke have den store betydning.} Der skal derfor skelnes mellem om der er tale om styrke eller styrkeangivelse.}