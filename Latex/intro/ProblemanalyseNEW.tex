\chapter{Problemanalyse}
\textit{I dette kapitel analyses problemstillinger, som opstår i forbindelse med lægemiddelskift. Disse problemstillinger vil sammenfattes i en opsummering og afsluttes med en problemformulering, der fremadrettet danner  grundlaget for rapporten.}

\section{Udbud af lægemidler}
Siden år 2007 har Amgros sendt lægemidler i udbud med henblik på at indkøbe lægemidler af højkvalitet med mest mulig besparelse til de offentlige danske hospitaler, hvilket har medført en besparelse på 3,1 millarder kroner i år 2017~\citep{Amgros2017b}. Et udbud kan forekomme hvis der findes mere end én leverandør af lægemidlet. I tilfælde af patent på et lægemiddel, hvormed der kun findes én leverandør, sættes lægemidlet sjældent i konkurrence, da prisen på allerede er fastsat.~\citep{Amgros2015}.

Størstedelen af udbud på lægemidler i ATC-grupper sker en gang årligt fra start september til midt november via Amgrosudbud\citep{Sygehusapoteket2017}. Hvor mindre udbud sker løbende hen over året og omfatter udbud på ATC-grupper som indgår i RADS behandlingsvejledninger.\fxnote{OBS! dette er medicinrådet nu}\citep{Sygehusapoteket2017}. 

\subsection{Årligt udbud via Amgrosudbud}
Ved udbud indsender ansøgende leverandører en omkostnings- og budgetkonsekvensanalyse for nye lægemidler og indikationer til Medicinrådet~\citep{Amgros2017, Amgros2017a}. Omkostningsanalysen omfatter samfundsomkostninger per patient for den nuværende og den ansøgte behandling.
Budgetkonsekvensanalysen omhandler de samlede økonomiske konsekvenser for regionerne ved at anvende det ansøgte lægemiddel.~\citep{Amgros2017a}

Analyserne vurderes på vegne af Medicinrådet af Amgros i forhold til relevans og valide oplysninger.~\citep{Amgros2017, Amgros2017a} Der vurderes, relevans i klinisk praksis, overholdelse af metodevejledning, kvalitet af omkostningsmodellen og overordnede usikkerheder samt evidensens kvalitet. Yderligere kategoriseres de nye lægemidler og indikationer i forhold til den nuværende behandling i stor, vigtig, lille eller ingen merværdi af Medicinrådet.~\citep{Amgros2017, Amgros2017a}

Den kliniske merværdi og analyserne danner grundlaget for prisforhandling~\citep{Amgros2017, Amgros2017a}. Amgros forhandler med den ansøgende leverandør for at opnå retfærdigt forhold mellem merværdi og meromkostninger i forhold til den nuværende og ansøgte behandling. Ud fra beslutningsgrundlag på forhandlingerne udarbejder Amgros en anbefaling til Medicinrådet.~\citep{Amgros2017, Amgros2017a}

På baggrund af den indsamlede evidens og resultatet af forhandlingerne foretaget af Amgros beslutter Medicinrådet hvorvidt lægemidlet skal anvendes som standardbehandling.~\citep{Amgros2017a}. 

\subsection{Løbende udbud via Medicinrådet}
*** SKAL OMSKRIVES *** \\
Medinrådet vurderer om  nye lægemidler og nye indikationer kan anbefales som standardbehandling og arbejder fælles regionale behandlingsvejledninger. Nye lægemidler vurderes i forhold til effekt, eksisterende behandling og omkostninger -> størst muligt gavn for patienter og lavere omkostninger for regionerne. De fælles regionale behandlingsvejledninger er vurderinger af hvilke lægemidler, der er mest hensigtsmæssige til behandlingne af patienter inden for et terapiområde og dermed grundlag for ensartet høj kvalitet på tværs af sygehuse og regioner. 


Siden år 2017 har medicinal virksomheder kunnet ansøge medinrådet om at få vurderet deres nye lægemidler eller indikationer kan anbefales som en standardbehandling. 5 ansøgninger i år 2017 -> anbefalinger udsendes til regionerne. 


Når en virksomhed henvender sig med et lægemiddel til vurdering på begynder sekretariatet sagsbehandling frem mod Rådets anbefaling. Forud for virksomhedens aflevering af den endelige ansøgning forbereder medicinrådet og virksomheden det faglige grundlag for ansøgningen. 

Dialog møder

\section{Skift af lægemiddel}
Et lægemiddelskift kan forekomme ved kontraktskift, hvor Amgros indgår en ny aftale med en leverandør om levering af lægemiddel via Amgrosudbud~\citep{Amgros2017a}. Når efterspørgslen på et lægemiddel overstiger den tilgængelige mængde opstår der restordre, hvilket ligeledes giver anledning til kontraktskift ~\citep{Amgros2015}. Restordre kan f.eks. opstå ved leveringesvigt fra leverandøren eller producenten på det ønskede lægemiddel~\citep{Amgros2017, Laegemiddelinformaion2017}. Leveringesvigt skyldes som ofte at producenten har mangel på råvarer eller produktionsvanskeligheder~\citep{Amgros2017, Laegemiddelinformaion2017}. I tilfælde  af restordre er det leverandørens ansvar at dække hospitalsapotekernes udgift ved indkøb af et erstatningslægemiddel\fxnote{Laegemiddelinformaion2017, Amgros2017}.

*** SKRIV NOGET OM MEDICINRÅDET

\section{Implementering af lægemiddelskift}
Implementering af et lægemiddel kan kategoriseres som simpel eller kompleks på baggrund af flere faktorer som f.eks. hvem skiftet har betydningen for, om der er nogle begrænsninger for hvornår et skift kan finde sted og risikovurdering af hvordan det påvirker afsnittet~\citep{Sygehusapoteket2017b}.

Et simpelt lægemiddelskift er vurderet til at påvirke klinikken i lav grad og varetages ofte af logistik-afdeling, hvorimod et kompleks lægemiddelskift påvirker klinikken i mellem til høj grad, hvorfor flere interessenter involveres ved disse skift.~\citep{Laegemiddelinformaion2017,Sygehusapoteket2017a}. 

Simple lægemiddelskift sker til dagligt i forbindelse med at et lægemiddel skiftes, på grund af restordre, til et simpel generisk lægemiddel~\citep{Laegemiddelinformaion2017}. De komplekse skift sker i forbindelse med ændringer af generiske lægemidler som f.eks. styrke, disponeringsform og ændring i hjælpestoffer. Ofte kontaktes interessenter som medicinansvarlige, kontraktsygeplejersker eller medicinservicefarmakonomerne i forhold til at undersøge lægemidlets anvendelighed for det pågældende hospitalsafsnit.~\citep{Laegemiddelinformaion2017,Sygehusapoteket2017a}
