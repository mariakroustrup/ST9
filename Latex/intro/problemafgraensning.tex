\section{Problemafgræsning}
På trods af at lægemidlerne sendes i udbud via Amgros med henblik på at opnå økonomiske besparelser er sundhedsudgifter til sygehusmedicin stigende. Udover det økonomiske aspekt medvirker implementeringen af lægemiddelskift til patientsikkerhedsmæssige konsekvenser på grund af medicineringsfejl. Størstedelen af fejlene forekommer ved forkert lægemiddel ved generisk substitution og de hyppigeste årsager til dette skyldes lignende og/eller svært navn på lægemidlet. Guidelines er udviklet med henblik på at undgå forveksling af navn samt emballage på lægemidlerne. Stregkode teknologier har påvist at have gavnlige effekter ved at reducere antallet af medicineringsfejl. Ligeledes er klinisk beslutningssøtte, som er anvendt i vid udstrækning inden for sundhedssektoren, påvist i flere studier at medvirke til reduceringen af fejl og anses samtidig som et vigtigt redskab til at øge kvaliteten ved medicinering.

\section{Problemformulering}
\textit{Hvilket potentiale har en algoritme som beslutningsstøttesystem til kategorisering af lægemidler med henblik på at forebygge fejl ved medicinering?}
