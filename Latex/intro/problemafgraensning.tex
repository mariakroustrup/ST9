\section{Problemafgræsning}
På trods af at lægemidlerne sendes i udbud via Amgros med henblik på at opnå økonomiske besparelser er sundhedsudgifter til sygehusmedicin stigende. Udover det økonomiske aspekt medvirker implementeringen af lægemiddelskift til patientsikkerhedsmæssige konsekvenser. Dette bekræftes i det stigende antal af rapporterede UTH'er omhandlende medicinering herunder ordination, dispensering og administration. Flere løsninger er blevet anvendt i forhold til at reducere antallet af fejl ved medicinering herunder elektronisk beslutningsstøtte, stregkode scanning samt klar-til-brug.

*** Mangler en kobling til afsnittet under. Tænker at det kunne gøres ved at lave et afsnit omkring hvilken betydning vejledning har i forhold til at mindske fejl ***

På nuværende tidspunkt vurderes sværhedsgraden af implementeringen af lægemidler manuelt af en ATC-ansvarlig på SRN ud fra en skabelon, som fremgår af Appendiks \ref{cha:AppD}. Dette gør metoden sårbar da den er personafhængig, hvorfor et hjælpemiddel til at kategorisere sværhedsgraden ved implementeringen ønskes. 

\section{Problemformulering}
\textit{Hvilket potentiale har en algoritme til kategorisering af sværhedsgraden ved implementering af et nyt lægemiddel ud fra en række parametre der indsamles i forbindelse med Amgrosudbud?}
