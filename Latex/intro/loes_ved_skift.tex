\section{Forebyggelse af problemstillinger ved lægemiddelskift}
De hyppigeste utilsigtede hændelser opstår i forbindelse med medicinering, administration og disponering~\citep{Jensen2014}. 
En måde at begrænse antallet af UTH'er er at synliggøre de problemstillinger der har medført rapporteringen af UTH'et. Ved at rapportere UTH'er kan de bagvedliggende årsager til fejl forebygges~\citep{StyrelsenforPatientsikkerhed2017}. Indbretningen af UTH'er er for sundhedsprofesionelle lovpligtigt~\citep{Jensen2014}. Yderligere blev det i år 2011 muligt for pårørende at rapportere UTH'er~\citep{Jensen2014}. På baggrund af rapporterede UTH'er er elektronisk patientmedicinering, stregkode scanning og klar-til-brug lægemidler implementeret med henblik på at nedbringe antallet af UTH'er.

\subsection{Elektronisk patientmedicinering}
Studier har påvist at implementering af elektronisk beslutningsstøtte reducerer antallet af UTH'er~\citep{DW1998,Bates2013,Cheng2011,Raboel2005} og beskrives som et vigtigt redskab til at øge kvaliteten af medicinering med henblik på at nedbringe fejl~\citep{Raboel2005}. I Danmark har indførsel af elektronisk patientmedicinering (EPM) medvirket til at reducere antallet af medicineringsfejl. Dette hjælper med at simplificere selve medicineringsprocessen, da dokumentation af ordination, dispensering og administration er samlet i ét system. Dette hjælpemiddel anvendes som et passivt beslutningsstøte for lægen ved ordination. Medicineringsfejl kan yderligere nedsættes hvis der udnyttes aktiv elektronisk beslutningsstøtte, hvor lægen vejledes f.eks. i forhold til lægemiddeldosis eller advares, hvis ordination kan være skadelig for patienten.~\citep{Raboel2005}. Til trods for at EPM har påvist at reducere antallet af UTH'er ved medicinering er der rapporteret UTH'er ved anvendelse af EPM~\citep{Syddanmark2008}, hvorfor der bør være fokus på forbedring og anvendelse af teknologier. 

%- Ny it-platform letter arbejdsgangene APOTO. APOTO skal levere data af høj kvalitet om lægemidler til sygehusenes EPJ systemer og på længere sigt kansystemet udvikles til også at under-støtte sygehusapotekernes produktion af lægemidler.  %[KILDE:Amgros_Årsmagasin2014, side 12]

\subsection{Stregkode scanning}
Internationale studier har påvist at stregkode scanning af lægemidler har en effekt på reducering af fejl i medicineringsprocessen~\citep{Poon2006,Bates2000,Levtzion-korach2010}. I Danmark anvender størstedelen af hospitalerne stregkode scanning, hvor lægemidlet kan spores fra leverandøren til patienten, hvilket medvirker til en effektiv lægemiddelforsyning og sikker medicinering ~\citep{Dzik2007,DPSD2008,Amgros2013}. Amgros har siden år 2010 stillet krav til stregkode på yderste og inderste emballage på lægemidler~\citep{Amgros2013}. Foruden at undgå alvorlige medicinrelaterede hændelser og reducere omkostninger samt tidsforbruget ved lagerstyring kan stregkoder medfører til andre positive resultater såsom, bedre sikring ved registering af lægemiddel i patienternes medicinjournal og effektiv tilbagekaldelse af medicin via it-systemer~\citep{Amgros2013}.
Implementering af stregkode scanning er påvirket af manglende stregkoder og dokumentation ved generisk lægemiddel. 

\subsection{Klar-til-brug lægemidler}
Internationale studier har påvist at automatisk dosisdispensering reducerer antallet af medicineringsfejl~\citep{Oren2003,Sygehusapotekerne2012}. I Danmark er der endnu ikke dokumenteret effekt på medicineringsfejl ved automatisk dosisdispensering, men påvist at reducering i fejl ved dispensering og administration~\citep{Sygehusapotekerne2012}. Dette er reduceret ved leveringen af klar-til-brug lægemidler såsom infusionsposer eller sprøjter, til de kliniske afdelinger~\citep{Sygehusapotekerne2012}. På denne måde skal personalet ikke tilberede lægemidlet, hvorved de skal håndtere farlige stoffer og det medvirker ligeledes til en sikre behandling for patienten, da risikoen for fejl er minimeret~\citep{Amgros2013}. 

\subsection{Vejledning}
*** Dette afsnit vil kunne lede godt op til det formål jeg har nemlig at kunne kategorisere sværhedsgraden af implementeringen med henblik på senere at kunne give vejledning til afdelingen på baggrund af denne ***