\section{Forebyggelse af problemstillinger ved lægemiddelskift}
Den europæiske lægemiddelstyrelse har udviklet guidelines med henblik på at forebygge problemer opstået ved forveksling af lægemidler og derved reducere antallet af medicineringsfejl~\citep{DanskSelskabforPatientsikkerhed2009}. Disse guidlines er ikke indført i Danmark, men der har været fokus på problemer med emballage forvekslinger, hvormed lovgivningen i Danmark er at pakninger ikke må kunne forveksles.   

Udover guidelines har internationle studier påvist at stregkode teknologi reducerer antallet af fejl i medicineringsprocessen~\citep{Poon2006, Levtzion-korach2010} Stregkode teknologi kan anvendes til at sikre at den rette medicin modtages af den rette patient~\citep{Amgros2013}.  
Amgros har siden år 2010 stillet krav til stregkode på yderste og inderste emballage på lægemidler~\citep{Amgros2013}. Foruden at undgå alvorlige medicinrelaterede fejl kan stregkoder medfører en mere enkel og sikker registrering af lægemiddel i patienternes medicinjournal og effektiv tilbagekaldelse af medicin via it-systemer~\citep{Amgros2013}. Implementering af stregkode teknologi er påvirket af manglende stregkoder og dokumentation ved generisk lægemiddel~\citep{Amgros2013}.

Det anbefales yderligere at anvende store og små bogstaver til navne på lægemidlerne i flere lande med henblik på at advare sundhedspersonalet om lignende lægemidler~\citep{ISMP2011,HQSC2013,ACSQ2011}. Sundhedspersonalet vurderede at det vil have en gavnlig effekt at blive advaret via Tall Man Lettering og at dette især vil have en betydning for lægemidler, hvor navneforveksling kan opstå~\citep{Campmans2018}.

Ligeledes har klinisk besluningstøtte i forhold til advarsler ved forveksling af navn og styrke vist sig at have en positiv effektiv i forhold til at forebygge antallet af medicineringsfejl~\citep{Campmans2018}. Flere studier har påvist at implementering af elektronisk beslutningsstøtte reducerer antallet af fejl~\citep{DW1998,Bates2013,Cheng2011,Raboel2005} og beskrives som et vigtigt redskab til at øge kvaliteten af medicinering med henblik på at nedbringe fejl~\citep{Raboel2005}. Klinisk beslutningsstøtte er anvendt i vid udstrækning inden for sundhedssektoren som f.eks. ved kontrol af lægemiddelallergier og interaktioner mellem to eller flere lægemidler~\citep{Raboel2005}. 