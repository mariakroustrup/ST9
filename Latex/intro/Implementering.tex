\section{Implementering af nyt lægemiddel}
Forinden et lægemiddel kan implementeres indkøbes lægemidlerne af Sygehusapoteket Region Nordjylland (SRN) som står for indkøb af lægemidler til distribution på de Nordjyske hospitaler ~\citep{SygehusapoteketRegionNordjylland2013}, som beskrevet i Appendiks \ref{cha:AppC}. Implementering af lægemidlet vurderes af ATC-ansvarlige, som er eksperter på området, som simpel eller kompleks på baggrund af flere faktorer~\citep{Sygehusapoteket2017}, som er beskrevet i Appendiks \ref{cha:AppD}. Dette indebærer f.eks. risikovurdering på afsnittet og lægemidlet samt type skift herunder bl.a. skift af dosis eller device og analoge lægemidler~\citep{Sygehusapoteket2017}. 

Et simpelt lægemiddelskift er vurderet til at påvirke klinikken i lav grad og varetages ofte af SRN's logistik afdeling~\citep{Laegemiddelinformaion2017, Sygehusapoteket2017a}. Disse skift sker dagligt i forbindelse med et simpel generisk lægemiddelskift. Hvorimod et kompleks lægemiddelskift påvirker klinikken i mellem til høj grad og kræver ofte involvering af flere interessenter som f.eks. medicinansvarlig, overlæger, kontaktsygeplejersker eller medicinservicefarmakonomer til at undersøge lægemidlets anvendelighed for det pågældende hospitalsafsnit. 
De komplekse skift sker ved generiske lægemidler, hvor flere faktorer som f.eks. styrke og disponeringsform afviger fra den nuværende behandling.~\citep{Laegemiddelinformaion2017,Sygehusapoteket2017a}
