\chapter{Problemanalyse}

\section{Lægemiddelskift}
\textit{Hvad er et lægemiddelskift, hvorfor sker det og hvad medfører det?} \\
Udgifterne til sygehusmedicin er stigende i flere europæiske lande, hvorfor flere lægemidler substitueres, med henblik på at opnå besparelser på medicin~\citep{Ess2003,Johnston2011}. Ved substitution udskiftes et lægemiddel til et andet lægemiddel enten ved generisk eller analog substitution.

Generisk substitution er substitution af lægemidler, der er  generisk ækvivalent med det forskrevne lægemiddel, herunder samme aktive stof, identiske styrke, koncentration og administrationsvej~\citep{DanskSelskabforPatientsikkerhed2009, Kairi2017}. Dette kan give anledning til at lægemidlets navn skifter eller varemærke ændres~\citep{Kairi2017}. Analog substitution er substitutionen af lægemidler, der afviger i sammensætningen, men anses for at have lignende bivirkninger og terapeutiske egenskaber~\citep{DanskSelskabforPatientsikkerhed2009, Kairi2017}. Analog substitution er derfor gældende for alle lægemidler, som ikke er generiske.~\citep{Kairi2017}.


\section{Konsekvenser ved lægemiddelskift} \label{sec:ProblemLaeg} %Problemer ved lægemiddelskift i klinikken} \label{sec:ProblemLaeg}
\textit{Hvilke problemer opstår ved lægemiddelskift og hvad skyldes disse problemer?} \\
Substitution her forskellige ulemper som kan lede til patientsikkerhedsmæssige konsekvenser~\citep{DanskSelskabforPatientsikkerhed2009}. Producenten af lægemidlet kan anvende forskellige hjælpestoffer, som på trods af kliniske forsøg, kan påvirke patienter forskelligt i forhold til optagelse og interaktion med andre lægemidler~\citep{Kairi2017}. Lægemidlet som erstattes kan have en anderledes form, størrelse eller farve end det forskrevne lægemiddel, hvormed patienten kan undlade at tage medicinen på grund af mistanke om fejl i ordination~\citep{Kairi2017}. Analog substitution kan alene medvirke til at lægemidlet inden for samme farmakologiske klasse afviger i forhold til biologiske virkning.~\citep{Kairi2017}. Det er endnu ikke påvist hvilken betydning dette har for den terapeutiske virkning~\citep{Kairi2017}. 

Et norsk studie har undersøgt konsekvenserne ved generisk substitution~\citep{Hakonsen2010}. Interview med 100 sygeplejersker påviste at der opstod fejlmedicinering ved generiske lægemidler~\citep{Hakonsen2010}. Ud af disse følte %92~\% af sygeplejerskerne at generiske lægemidler var tidskrævende og
91~\% at risikoen for fejl øges ved dispensering af disse, hvoraf 42~\% oplevede fejl som følge af generisk substitution~\citep{Hakonsen2010}.
Medicineringsfejl ved generisk substitution fremgår af Figur \ref{fig:GeneriskSubstitution}.

\begin{figure}[H]\centering	\includegraphics[width=0.8\textwidth]{billeder/GenSub.png} 
	\caption{Medicineringsfejl ved generisk substitution rapporteret (n=100)~\citep{Hakonsen2010}.}
	\label{fig:GeneriskSubstitution}  
\end{figure}

Det fremgår af Figur \ref{fig:GeneriskSubstitution} at størstedelen af fejlmedicinering ved generisk substitution skyldes forkert lægemiddel, hvoraf en mindre del skyldes forkert formulering og i sjældnere tilfælde forkert dosis, administrationsvej og udeladelse af dosis. Forkert lægemiddel forstås som at et andet lægemiddel end det forskrevne er givet. Formulering beskriver lægemidlet fysiske form som f.eks. tabelletform, dosis beskriver mængden af lægemidlet og administrationsvej beskriver hvordan indgiften af et lægemiddel tages f.eks. via munden.

Årsagerne til medicineringsfejl blev rapporteret af 42 sygeplejersker~\citep{Hakonsen2010}, og resultaterne heraf fremgår af Figur \ref{fig:GeneriskSubstitution1}.

\begin{figure}[H]\centering	\includegraphics[width=1\textwidth]{billeder/GenSub1.png} 
	\caption{Årsager til medicineringsfejl ved generisk substitution (n=42)~\citep{Hakonsen2010}.}	\label{fig:GeneriskSubstitution1}  
\end{figure}

Af Figur \ref{fig:GeneriskSubstitution1} fremgår det at størstedelen af årsagerne til medicineringsfejl ved generisk substitution skyldes lignende og/eller vanskeligt lægemiddelnavn, kraftig arbejdsbyrde og sløvhed eller fravær af dobbelttjek. En mindre del skyldes utilstrækkeligt journalførring og/eller ordination, usikkerhed eller andet. 

I flere lande opstår forkert lægemiddel ofte i forbindelse med forvirring ved forveksling af navn~\citep{DanskSelskabforPatientsikkerhed2009}, hvilket afspejles i det norske studie. Forveksling af lægemiddelnavne har i sjældnere tilfælde haft konsekvenser som har medført forlænget indlæggelse, forværret sygdom eller dødsfald~\citep{DanskSelskabforPatientsikkerhed2009}. Forveksling af navn kan for eksempel forekomme ved panodil, som er et smertestillende lægemiddel, og plendil, som anvendes til behandling af forhøjet blodtryk~\citep{DanskSelskabforPatientsikkerhed2009}. Derudover kan forskellige suffiks eller præfiks skabe forvirring og give anledning til fejl i dispensering, såsom Efexor kontra Efexor Depot~\citep{DanskSelskabforPatientsikkerhed2009} 

Udover lægemidlets navn kan lægemidler som har lignede navne, så kaldte look-a-like, prædisponeres til medicineringsfejl som kan have patientsikkerhedsmæssige konsekvenser~\citep{Wittich2014}. Eksempler på look-a-like lægemidler er  dopamin og dobutamin~\citep{Wittich2014}. Brugen af forkortelser i ordinationen medfører ligeledes øget risiko for medicineringsfejl~\citep{Wittich2014}.

Nogle af sygeplejerskerne i det norske studie mente at forvirringen over at finde den korrekte substitution kunne lede til at doseringen og formulationen var skyld i medicineringsfejl~\citep{Hakonsen2010}. Kognitive forstyrrelser, som fejl ved bekræftelse eller mangel på situationsfornemmelse, kan bidrage til medicineringsfejl. Et eksempel på dette kan være forvirring over at lægemidlets leverandør ændres hyppigt, hvormed det nuværende lægemidlet substitueres til et andet~\citep{Wittich2014}

Medicinering var den hyppigste årsag til rapportering af utilsigtede hændelser i Danmark i år 2013~\citep{Patientombuddet2013}. Antallet af rapporteringer i Region Nordjylland er steget med over 36~\% fra år 2012 til 2014~\citep{Jensen2014}. Ud af 824 rapporterede utilsigtede hændelser i år 2014 skyldes 97\% medicinering, 86\% administration af medicin og 41\% disponering~\citep{Jensen2014}, hvor mere end én rapporteret utilsigtede hændelser skyldes én eller flere grunde.  Håndteringen af medicin er kompleks, da det er en mangeartet proces som involverer flere personer og mange led~\citep{Barker2002,Sundhedsstyrelsen2005,Lisby2005, Tully2009}.Størstedelen af fejl forekommer i forbindelse med ordination og administration, hvor en mindre andel opstår i forbindelse med transskribering og dispensering~\citep{Agrawal2009, Anderson2002} En fælles årsag i disse studier var forkert dosis, forkert lægemiddel og udeladelse af henholdsvis ordination, dispensering og administration~\citep{Barker2002,Sundhedsstyrelsen2005,Lisby2005, Tully2009}.

\section{Implementering af lægemiddelskift} \label{sec:ImpLaeg}
\textit{Hvordan implementeres lægemiddelskiftet og hvilke ulemper er forbundet med denne proces?} \\
For at forebygge fejlmedicinering og forbedre patientsikkerheden ved lægemiddelskift formidler Sygehusapoteket Region Nordjylland (SRN) information om lægemiddelskift til de enkelte hospitalsafdelinger i regionen med henblik på at opnå en effektiv implementering. Hospitalsafdelingerne informeres om lægemiddelskift via Lægemiddel Nyt, som fremgår af Appendiks \ref{cha:AppD} punktnummer \ref{item:Laegemiddelnyt}. Lægemiddel Nyt udarbejdes på baggrund af vurdering af kompleksiteten af lægemiddelskift foretaget af ATC-ansvarlige medarbejdere fra SRN og skiftelister, som beskrives af Appendiks \ref{cha:AppD} punktnummer \ref{item:Skiftelister}. ATC-ansvarlige medarbejderes anvender skiftelisten samt en skabelon til vurdering, som fremgår af Appendiks \ref{cha:AppD} punktnummer \ref{item:ATC-ansvarlig} og oplysninger som ændret lægemidlets navn, dispenseringsform og styrke, tidligere erfaring, retningslinjer, problemstillinger vedrørende lægemiddelskift og indsamlet viden til at danne grundlag for vurdering af kompleksiteten af lægemiddelskift.

Denne proces stiller krav til den enkelte ATC-ansvarlige medarbejders erfaring inden for området, hvilket gør den personafhængig. Grundlaget for vurderingen er subjektiv, da den afhænger af én ATC-ansvarlig medarbejders viden inden for området og kan derfor være varierende mellem medarbejdere. I vurderingen sammenlignes data fra flere forskellige databaser for at finde nødvendige informationer omkring lægemiddelskiftet, hvormed risikofaktorer let kan overses. Samtidig udføres vurderingen af lægemiddelskift manuelt, hvilket gør processen sårbar.




%For at forebygge problemer som kan opstå i klinikken vurderer ATC-ansvarlige medarbejdere i SRN kompleksiteten af lægemiddelskift før dette implementeres i klinikken. Vurderingen af lægemiddelskift sker på baggrund af ATC-ansvarlige medarbejders tidligere erfaringer, retningslinjer, indsamlede problemstillinger vedrørerende lægemiddelskift, viden indsamlet samt ændringer ved lægemidlet som navn, dispenseringsform og styrke. Disse risikofaktorer vægtes af de ATC-ansvarlige medarbejdere, hvormed processen er sårbar, da den er personafhængig og manuel samt stiller krav til den enkelte medarbejders viden og erfaring inden for området. 

\section{Forebyggelse af problemstillinger ved lægemiddelskift}
\textit{Hvilke teknologier anvendes til at forbygge medicineringsfejl og hvordan kan den nuværende vurdering af lægemiddel optimeres?} \\
Flere studier har påvist at informationssystemer er anvendelige til forebyggelsen af fejlmedicinering ved ordination, dispensering og administration~~\citep{Agrawal2009, Kaushal2002, Stenner2010, Fischer2008, Simpson2008, Bates2000a}. Et eksempel er computerbaseret ordineringssystemer, som anvendes til at strukturere ordre, gøre disse letlæselige og fuldkommen samt gøre nødvendige oplysninger tilgængelige for klinikeren~\citep{Agrawal2009,Bates2000a}. I en kombination med beslutningsstøtte system, som f.eks. interaktion mellem lægemidler og automatisk beregning af styrke ved ændring i denne~\citep{Agrawal2009}, har computerbaseret ordineringssystemer påvist at være effektiv i forbedring af patientsikkerheden~\citep{Agrawal2009, Bates2000a}. Samme effekt er påvist for systemer med begrænset brug af beslutningsstøtte~\citep{Bates2000a}. Til dispensering anvendes forskellige teknologier som robotter og automatiserede skabe, som anvender stregkoder til at genkende medicin, hvilket medvirker til at reducere antallet af fejl relateret til emballage og dispensering.~\citep{Agrawal2009}

Fælles for de ovenstående informationssystemer er at de anvendes i klinikken når lægemiddelskiftet er implementeret. Ingen videnskabelig litteratur har undersøgt om informationssystemer kan anvendes til at opnå en effektiv implementering af lægemiddelskift i klinikken og derved gøre den nuværende vurdering mindre personafhængig og sårbar. Da informationssystemer, modsat den menneskelige evne, er i stand til at organisere og identificere sammenhænge mellem informationer fra en større mængde af data~\citep{Agrawal2009}, vil et system som dette medvirke til at processen kan blive mere ensartet og derved mindre personafhængig. Derudover vil flere risikofaktorer kunne vægtes i vurderingen af lægemiddelskift, hvilket vil understøtte beslutningsgrundlaget.  

Risikovurdering er bredt anvendt inden for sundhedssektoren som beslutningsstøtte~\citep{Geissert2018}. De fleste studier anvender statistisk, hvormed problemet målrettes og sammenhænge mellem input og output identificeres ved at vægte risikofaktorer. Denne tilgang kræver en stor mængde af data for at forudsige sammenhænge, da der både skal være data til at udvikle modellen samt afprøve denne. Da mængden af data ikke er stor nok til at målrette problemet til lægemiddelskift er det ikke muligt at anvende en statistisk tilgang. Givet at erfaringer og processerne for vurderingen af lægemiddelskift er kendte og tilgængeligheden af materiale kan en model med en deterministisk tilgang anvendes. Den eksisterende viden inden for området og fordelene ved informationssystemers evne til at sammenligne informationer fra en større mængde af data vil kunne anvendes til at gøre processen mindre sårbar og personafhængig. Regelbaseret systemer er anvendt i vid udstrækning ~\citep{}. Disse anvender en deterministisk metode, hvor regler er kendte og defineret ud fra en ekspert, hvilket gør dem hurtige at implementere og kræver minimal data. 

\section{Opsummering}
Lægemiddelskift sker som et led i at sænke udgifterne til sygehusmedicin, hvilket medfører substitution af lægemidler ~\citep{Ess2003, Johnston2011}. Substitution af lægemidler og hvor godt disse implementeres i klinikken har betydning for at forebygge medicineringsfejl og derved forbedre patientsikkerheden, jævnfør Afsnit \ref{sec:ProblemLaeg}. Den nuværende vurdering af kompleksiteten af lægemiddelskift inden implementering i klinikken foregår manuelt og er derfor sårbar, som beskrevet i Afsnit \ref{sec:ImpLaeg}. Ligeledes er den erfaringsbaseret og subjektiv, hvilket gør processen personafhængig.

Studier har påvist at informationssystemer kan anvendes til at reducere antallet af medicineringsfejl i klinikken~\citep{Agrawal2009, Stenner2010, Fischer2008, Simpson2008}, men ingen videnskabelig litteratur har undersøgt informationssystemer til at vurdere lægemiddelskift inden de implementeres i klinikken. 


Da flere studier anvender regelbaseret systemer til risikovurderingen forventes det at et system som dette vil kunne anvendes til at gøre den nuværende process mindre sårbar og personafhængig ved ....
Ud fra at proceduren for vurdering er kendt, erfaringerne er givet og materiale tilgængelig kan et regelbaseret system 

udvikles til at gøre den nuværende proces mindre sårbar og personafhængig.

. %Fordelene ved et system som dette er at det vil gøre processen mindre sårbar, da data er generelt af en computer og være mindre personafhængig, da den nødvendige data er tilgængelige for a


\section{Problemformulering}
\textit{Hvilket potentiale har et regelbaseret system til risikovurdering af lægemiddelskift med henblik på at gøre den nuværende proces mindre sårbar og personafhængig?}

\textit{risikovurdering af lægemiddelskift foretaget af ATC-ansvarlige medarbejdere på SRN? \textcolor{blue}{med henblik på at effektivisere implementering af kommende lægemiddelskift ved udarbejdelsen af LægemiddelNyt?} }



