\chapter{Indledning}
Den stigende andel af ældre, forekomsten og varigheden af kroniske sygdomme samt udviklingen i sundhedsforventninger og teknologier er skyld i stigende sundhedsudgifter i flere europæiske lande~\citep{Ess2003}. Danmark havde i år 2015 flere sundhedsudgifter end det gennemsnitlige land i Europa~\citep{EU2017}, hvor udgifterne siden år 2007 har været stigende med 7,8~\% i gennemsnit om året~\citep{Sundhed2016}.

For at begrænse udgifterne til sygehusmedicin foretager flere lande substitution af lægemidler~\citep{Ess2003,Johnston2011}, hvilket medfører lægemiddelskift af et lægemiddel til et andet lægemiddel~\citep{DanskSelskabforPatientsikkerhed2009, Kairi2017}. %\textcolor{blue}{ I Danmark sker substitution af lægemidler når Amgros, Regionernes Lægemiddelorganisation, sender lægemidler i udbud med henblik på at opnå bedre pris og kvalitet~\citep{Amgros2015}. %I år 2017 sparede Amgros de danske regioner for 3,1 milliarder kroner~\citep{Amgros2018b}.
%Udbud forekommer på lægemidler, hvor der findes mere end én leverandør~\citep{Amgros2015}. 
%Vinder en ny leverandør udbuddet giver dette anledning til lægemiddelskift og derved substitution af lægemidler~\citep{Amgros2015}.} %Vinder en ny leverandør udbuddet giver dette anledning til substitution af lægemidler~\citep{Amgros2015}.
%I Danmark har Amgros, Regionernes Lægemiddelorganisation, siden år 2007 sendt lægemidler i udbud hvert år med henblik på at indkøbe lægemidler af høj kvalitet til bedst mulig pris til de offentlige danske hospitaler~\citep{Amgros2018b}. I år 2017 sparede Amgros regionerne for 3,1 milliarder kroner~\citep{Amgros2017b}. Udbud forekommer på lægemidler, hvor der findes mere end én leverandør, hvormed lægemidler bringes i konkurrence~\citep{Amgros2005}. Vinder en ny leverandør udbuddet giver dette anledning til et lægemiddelskift~\citep{Amgros2015}.
Substitution af lægemidler kan kategoriseres som analog eller generisk~\citep{DanskSelskabforPatientsikkerhed2009}.  
Analog substitution er skift af lægemidler som indeholder forskelligt aktive stof, men forventes samme effekt og omtrent samme bivirkninger~\citep{DanskSelskabforPatientsikkerhed2009,Kairi2017}. 
Generisk substitution er skift af lægemidler, som indeholder samme aktive stof~\citep{DanskSelskabforPatientsikkerhed2009,Kairi2017}. 

Substitution af lægemidler kan medføre fejlmedicinering, hvilket kan have konsekvenser for patientsikkerheden~\citep{Hakonsen2010}. Den hyppigste fejl ved generisk substitution skyldes i 82,1~\% af tilfældene, at sygeplejerskerne ordinerede det forkerte lægemiddel~\citep{Hakonsen2010}. De typiske anledninger til forket lægemiddel er forveksling af navne, hvilket i nogle tilfælde har ført til forlænget indlæggelse, forværret sygdom og dødsfald~\citep{DanskSelskabforPatientsikkerhed2009}. Implementering af lægemiddelskift i klinikken har betydning for forebyggelsen af fejl ved medicinering.

I Region Nordjylland sender Sygehusapoteket Region Nordjylland (SRN) information via Lægemiddel Nyt til de enkelte hospitalsafdelinger for at gøre opmærksom på komplekse lægemiddelskift. Dette gøres med henblik på at forebygge fejl ved medicinering inden lægemiddelskift implementeres i klinikken. Risikovurdering af lægemiddelskift varetages af medarbejdere fra SRN ud fra ændrede egenskaber ved lægemidlet, erfaringer og indsamlet viden. Denne vurdering sker manuelt og er erfaringsbaseret, hvilket gør processen sårbar og personafhængig.

Der er ingen videnskabelig litteratur, som har undersøgt anvendeligheden af et informationssystem til risikovurdering af lægemiddelskift. 
Det er påvist i flere studier, at computerbaseret beslutningstøttesystemer kan anvendes i klinikken til forebyggelse af medicineringsfejl og derved forbedre patientsikkerheden~\citep{Agrawal2009, Stenner2010, Fischer2008, Simpson2008}, og at risikovurdering er anvendt som beslutningsstøtte inden for andre domæner i sundhedssektoren~\citep{Geissert2018, Rawshani2018}. 

På baggrund af dette er det relevant at analysere, hvorfor lægemiddelskift fører til patientsikkerhedsmæssige konsekvenser, hvordan risikovurderingen af lægemiddelskift udføres, samt undersøge hvordan informationssystemer anvendes til forebyggelse af medicineringsfejl og risikovurdering.
