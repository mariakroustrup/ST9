\chapter{Indledning}
Den stigende andel af ældre, forekomsten og varigheden af kroniske sygdomme samt udviklingen i sundhedsforventninger og teknologier er skyld i stigende sundhedsudgifter i flere europæiske lande~\citep{Ess2003}. Danmark havde i år 2015 flere sundhedsudgifter end det gennemsnitlige land i Europa~\citep{EU2017} og udgifterne har siden år 2007 været stigende med 7,8~\% i gennemsnit om året~\citep{Sundhed2016}.

For at begrænse udgifterne til sygehusmedicin foretager flere europæiske lande substitution af lægemidler~\citep{Ess2003,Johnston2011}, hvilket betyder at et lægemiddel udskiftes til et andet lægemiddel~\citep{DanskSelskabforPatientsikkerhed2009, Kairi2017}.  I Danmark sker substitution af lægemidler ved at Amgros, Regionernes Lægemiddelorganisation, sender lægemidler i udbud med henblik på at opnå bedre pris og kvalitet~\citep{Amgros2015}. %I år 2017 sparede Amgros de danske regioner for 3,1 milliarder kroner~\citep{Amgros2018b}.
%Udbud forekommer på lægemidler, hvor der findes mere end én leverandør~\citep{Amgros2015}. 
Vinder en ny leverandør udbuddet giver dette anledning til lægemiddelskift og derved substitution af lægemidler~\citep{Amgros2015}. %Vinder en ny leverandør udbuddet giver dette anledning til substitution af lægemidler~\citep{Amgros2015}.

%I Danmark har Amgros, Regionernes Lægemiddelorganisation, siden år 2007 sendt lægemidler i udbud hvert år med henblik på at indkøbe lægemidler af høj kvalitet til bedst mulig pris til de offentlige danske hospitaler~\citep{Amgros2018b}. I år 2017 sparede Amgros regionerne for 3,1 milliarder kroner~\citep{Amgros2017b}. Udbud forekommer på lægemidler, hvor der findes mere end én leverandør, hvormed lægemidler bringes i konkurrence~\citep{Amgros2005}. Vinder en ny leverandør udbuddet giver dette anledning til et lægemiddelskift~\citep{Amgros2015}.

Substitution af lægemidler sker enten ved analog eller generisk substitution~\citep{DanskSelskabforPatientsikkerhed2009}.  
Analog substitution omhandler lægemidler som indeholder forskelligt aktive stof, men forventes samme effekt og omtrent samme bivirkninger~\citep{DanskSelskabforPatientsikkerhed2009,Kairi2017}. 
Generisk substitution omhandler lægemidler, som indeholder samme aktive stof, altså er hinandens synonyme~\citep{DanskSelskabforPatientsikkerhed2009,Kairi2017}. 

Der er patientsikkerhedsmæssige konsekvenser forbundet med substitution, herunder fejlmedicinering~\citep{Hakonsen2010}. Den hyppigste fejl ved generisk substitution skyldes i 82,1~\% af tilfældene at sygeplejerskerne ordinerede det forkerte lægemiddel~\citep{Hakonsen2010}. De typiske anledninger til forket lægemiddel er forveksling af navne, hvilket i nogle tilfælde har medført til forlænget indlæggelse, forværret sygdom og dødsfald~\citep{DanskSelskabforPatientsikkerhed2009}. Implementering af lægemiddelskift i klinikken har betydning for forebyggelsen af medicineringsfejl.

I Region Nordjylland informerer Sygehusapoteket Region Nordjylland (SRN) via Lægemiddel Nyt de enkelte hospitalsafdelinger omkring komplekse lægemiddelskift med henblik på at forebygge fejlmedicinering og derved forbedre patientsikkerheden inden lægemiddelskift implementeres i klinikken. Kompleksiteten af lægemiddelskift vurderes af ATC-ansvarlige medarbejdere ud fra ændrede egenskaber ved lægemidlet, erfaringer og indsamlet viden. Denne vurdering sker manuelt og er erfaringsbaseret, hvilket gør processen både sårbar og personafhængig.

På baggrund af dette er det relevant at analysere, hvilke patientsikkerhedsmæssige konsekvenser lægemiddelskift medfører, hvordan lægemiddelskift vurderes inden implementering i klinikken samt hvordan problemstillinger relateret til klinikken og implementering af lægemiddelskift kan forebygges. 
