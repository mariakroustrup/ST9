\chapter{Teknologier}
Informationssystemer har bevist at være anvendeligt i forebyggelse af medicinfejl~\citep{Agrawal2009, Anderson2002}. Dette indebærer f.eks. computerbaseret lægeordreindgang, automatisk disperinsering skabe, bar-kodet medicin administration samt elektronisk afstemning af medicin. Flere hospitaler med automatiserede noter og journaler, ordreindgange og klinisk beslutningsstøte har påvist færre komplikationer, lavere dødelighed og lavere omkostninger.  

Håndteringen af medicin er kompleks, da det er en mangeartet operation som involvere flere personer og mange led, herunder ordination, transskribering, dispensering og administration.  Fejl blev påvist i alle led i processen, hvor størstedelen af fejl forekom i forbindelse med ordination og administration, hvor en mindre del af fejl forekom ved transskribering og dispensering~\citep{Agrawal2009,Anderson2002}.

\section{Ordination}
Størstedelen af medicineringsfejl sker ved ordination~\citep{Agrawal2009,Anderson2002,Kaushal2002}
Computerbaserede lægeordeindgange med patientspecifik beslutningsstøtte har påvist at have en potentielt stærk effekt for forbedring af patientsikkerheden~\citep{Agrawal2009, Bates2000a}. Dette gør sig også gældende ved mindre for systemer med en begrænset brug af beslutningsstøtte~\citep{Bates2000a}. Antallet at alvorlige fejl faldt med 55~\% i et studie og antallet af fejl generelt faldt med 83~\% i et andet studie ved brug af computerbaseret lægeordindgange~\citep{Bates2000a}. 

Gennemgående fejl ved ordinering inkluderer brug af forkert stof, doseringsform, styrkeberegning, manglende kontrol af allergier og manglende evne til at justere doseringen hos patienter ved nedsat nyre- eller leverfunktion.~\citep{Agrawal2009} 

Computerbaserede lægeordreindgange sikrer at alle ordre er struktureret, letlæselige og fuldkommen, hvor nødvendige oplysninger som styrke, rute og doseringsform~\citep{Agrawal2009,Bates2000a}. Ligeledes kan f.eks. allergier og interaktioner af lægemidler kontrolleres samt beregning ved justering af styrke baseret på vægt eller nyrefunktion~\citep{Agrawal2009}. Yderligere sker ajourføring af ordination med det seneste lægemiddelinformation for at undgå at lægemidler er trukket tilbage~\citep{Agrawal2009}.

\subsection{Beslutningsstøttesystem}
Beslutningsstøttesystemer kan blive implementeret som en isoleret applikation og kan være simpelt eller mere kompleks~\citep{Kaushal2002}. Et simpel kan f.eks. anvendes til at støtte ved valg af medicin, styrke og varighed, hvor mere avanceret systemer inkorporerer patient-specifikke eller patogene-specifikke informationer som f.eks. ved evaluering af antibiotika.~\citep{Kaushal2002}

Studier har påvist at brugen af beslutningsstøtte ved generisk substitution ved ordination har en positiv effekt i forhold til sikkerhed og kvalitet~\citep{Stenner2010, Fischer2008}. Opmærksom på tilgængeligheden af generisk substitution i det computerbaserede ordineringssystem medvirkede til et dramatisk og vedvarende forbedring i andelen af generiske substitutioner~\citep{Stenner2010}. Ligeledes var klinikere som anvendte ordination med beslutningsstøtte tilbøjelige til at ordinere generiske lægemidler, hvormed økonomiske besparelser var betydelige~\citep{Fischer2008}. %Inden for nogle tepariområder, herunder onkologi og neurologi, blev det valgt ikke at ordinere generisk substitution, grundet nylige data om potentiel fare ved generiske lægemidler.~\citep{Stenner2010}

\section{Transskribering}
Dokumentation og tiltag på transskribering området er begrænset~\citep{Kaushal2002}. Det formodes at en kombination af computerbaserede lægeordreindgange og computerbaserede medicin administration journal vil reducere antallet af fejl og derved forebygge fejl opstået ved transskribering~\citep{Kaushal2002}.

\section{Dispensering}
Der foretages et stort antal af dispenseringer på hospitalerne som kan lede til dispenseringsfejl~\citep{Agrawal2009}. Disse er ofte ikke opdaget og kan medfører alvorlige konsekvenser~\citep{Simpson2008}. Forskellige teknologier med henblik på at reducere antallet af fejl  er udviklet som f.eks. robotter til dispensering og automatiserede dispensering skabe. Disse teknologier reducerer antallet af dispeneringsfejl ved emballering, dispensering og anerkendelse af medicin ved stregkoder.~\citep{Agrawal2009}

Robotter anvendes til at automatisere dispenseringen af medicin hvor der er simple, rutine opgaver som f.eks. genkendelse af medicin ved brug af stregkoder~\citep{Kaushal2002}. Dispenseringsrobotter har påvist at mindske antallet af dispenseringsfejl med 50~\%, som f.eks. forkert lægemiddel, styrke og mængde~\citep{Stephen2013}.

Automatiserede dispensering anordninger kan reducere antallet af fejl væsentligt, hvis disse er forbundet med stregkodning og informationssystemer på hospitalet~\citep{Bates2000a}. Det er på denne måde muligt at holde lægemidler på et sted og udelade dem kun til en bestemt patient. Uden forbindelsen mellem systemer er effekten af automatiserede dispensering ukendt.~\citep{Bates2000a} 

\section{Administration}
For at reducere antallet af fejl ved administration er stregkodet medicin administration som bygger på fem rettigheder som rigtige patient, medicin, dosis, administrationsvej og tid~\citep{Agrawal2009}. Stregkoder kræver at sygeplejerskerne skal scanne patientens identifikationsarmbånd hvormed enhedsdosis af medicinen administreres. Hvis der er fejlpasning af patientidentitet eller navn, dosis eller administrationsvej af medicin advares sygeplejersken via systemet.~\citep{Agrawal2009} 

Stregkoder er anvendt i vid udstrækning i industrien til at forbedre nøjagtighed, men dette er ikke tilfældet inden for medicinal industrien~\citep{Kaushal2002}. Dette er grundet at virksomhederne som producerer medicinen ikke er enige om en fælles fremgangsmåde. Nogle individuelle hospitaler har indført stregkodet medicin, hvilket medvirker til en effektiv identificering af navn, styrke og administration tid af medicinen. Det er ligeledes muligt at tilknytte medicin til den rette patient og personale. Udover at være tidsbesparende for personalet er det påvist i et studie at stregkoder reducerede 80~\% af administrationsfejlene.~\citep{Kaushal2002}

En af de hyppigste årsager til medicineringsfejl er intravenøs administration~\citep{Kaushal2002}. Et intravenøs administration device, som anvender forenklet programmering og computerbaseret kontrol kan medvirke til at reducere fejl ved intravenøs medicinering. Dette medvirker blandt andet til at reducere sandsynligheden for tidobbelt overdoser.~\citep{Kaushal2002}


%\section{Elektronisk patient journal}
%God dokumentation understøter evidensbaseret sundhedsvæsen og letter revisonen og kvalitetsovervågning, som er vigtig parameter i mange sundhedsøkonomier. Brugen af EPJ er blevet universel og apoteksapotekspersonalet er bekendt med brugen af ​​edb-registrerede arkiver til støtte dispenseringsprocessen og rådgivning om medicin inden for deres praksis. Men i både primærpleje og sekundær pleje, nye apotekstjenester og innovative arbejdsmetoder udvikles, som kræver realtidsadgang til EPJ til klinisk beslutningstagning.

