\chapter{Beslutningsstøttesystemer}
Den hyppigste form for beslutning er anvendt i forbindelse med at stille en diagnose, som f.eks. at beslutte hvilket spørgsmål der ønskes besvares, hvilke test som skal bestilles eller hvilken procedure der skal udføres og bestemme værdien af resultatet relateret til den associerede risiko eller økonomiske udgift. I forhold til at stille en diagnose skal lægen beslutte hvad der er sandt i forhold til det patienten giver udtryk for samt finde den nødvendige data til at bestemme dette. Selvom diagnosen er kendt stilles der stadig krav til lægen viden og erfaring som f.eks. hvorvidt patienten skal behandles og hvis dette er tilfældet, hvilken behandling skal dermed foretages.

Inden for biomedicin er der også en række beslutningsopgaver, disse involverer i modsætning til diagnose ikke bestemte patienter eller sygdomme. Dette kan f.eks. være administration af hospitalet, hvor ledelsesdata anvendes til at styre beslutninger om ressourceallokering på hospitalet. Udover kliniske beslutninger kan begreberne generaliseres til andre problemområder f.eks. inden for finansområdet. 

Krav til at få den bedste beslutning er nøjagtige data, relevant viden og passende problemløsning færdigheder. Data for en given sag skal være tilstrækkelige, men ikke overdreven, til at det er muligt at træffe en velbegrundet beslutning. For mange oplysninger begrænser beslutningstageren i at behandle og syntetisere oplysninger intelligent og hurtigt og derved skabe forvirring i stedet for præcision. Et eksempel hvor dette kan opstå er operationsrum og intensiv, hvor der indsamles en masse data og beslutninger skal tages pludseligt. 

Kvaliteten af det tilgængelige data er også vigtigt, herunder præcision i terminologi, læsebart, tilgængelige optagelser. Brug af defekte data kan have alvorlige konsekvenser for patienten, hvis et forkert beslutning bliver truffet på baggrund af dette. Derfor skal klinisk data ofte valideres. Selv god data er ubrugelig, hvis der ikke er grundlæggende viden, der er nødvendig for at anvende dem korrekt. Beslutningstagerne skal have en bred viden om medicin, dybtgående fortrolighed med deres ekspertområde  og adgang til informationsressourcer, der giver relevante oplysninger. Deres viden skal være præcis og aktuelt i den hurtigt skiftende verden af medicin.

Det er ligeledes også vigtigt at beslutningstagere ved, hvordan man fastsætter passende mål for en opgave, hvordan der redegøres for hvert mål og hvordan afvejning mellem omkostninger og fordele ved diagnostiske processer tydeliggøres. Den dygtige kliniker trækker på person erfaring, hvor nye klinikere modsat skal bedømme baseret på at redegøre effektivt og hensigtsmæssigt om hvad der skal gøre i form af viden inden for området eller ved adgang til højkvalitet data om patienten. 

Programmerne skal have adgang til gode data, de skal have omfattende baggrundskendskab koder for det pågældende kliniske domæne, og de skal legemliggøre intelligent adgang til problemløsning, der er følsom over for krav til korrekt analyse, passende omkostningsfordele og effektivitet. 

\section{Computers rolle i beslutningsstøttesystemer}
Et klinisk beslutningsstøttesystem er ethvert computerprogram designet til at hjælpe sundhedspleje fagfolk til at træffe klinisk beslutninger. Der findes tre typer af beslutningsstøttefunktioner som værktøjer til informationsstyring, kliniske laboratoriesystemer og patientspecifikke anbefalinger. 

Informationsstyring omhandler sundhedsvæsnet informationssystemer og hvordan information lagres, deles og hentes fra disse systemer. Det hjælper klinikeren til at indhente relevant data og viden, men hjælper ikke til en bestemt beslutning. Fortolkningen af data er overladt til klinikeren, ligesom beslutningen om hvilke oplysninger der er nødvendige for at løse det kliniske problem.

Klinisk laboratoriesystemer som synliggøre værdier som ligger uden for normalen, lister over mulige forklaringer til de abnormiteter eller interaktioner af forskellige lægemidler. Disse systemer sørger for at gøre brugeren opmærksom på diagnoser eller problemer kan være overset. Disse systemer bruger ofte simple logikker, hvor der f.eks. vises fast lister eller standard svare på en bestemt eller potentiel abnormitet. 

Patientspecifikke anbefalinger giver skræddersyede vurderinger eller rådgivning baseret på patientspecifikke data. De kan følge simple logikker, kan være baseret på beslutning teori og omkostningsanalyse, eller kun anvende numeriske tilgange som en supplerende til symbolsk problemløsning. Nogle af disse systemer foreslår differentielle diagnoser eller indikere yderligere oplysninger, der kan medvirke til at begrænse udbuddet af ætiologiske muligheder~\fxnote{summen af de genetiske og miljømæssige faktorer, der igangsætter en sygdom, i modsætning til læren om sygdommenes patogenese, der beskriver mekanismerne ved sygdomsprocesserne.}. Andre systemet foreslår en enkelt forklaring på patientens symptomer. Andre systemer fortolker og opsummerer patientens journal over tid og på den måde følsom over der for den kliniske kontekst. Grænserne mellem disse er ikke skarpe, men det er nyttigt at skelne mellem for at definere den række muligheder, som computere kan hjælpe  med kliniske beslutninger. 

\section{Typer af biomedicinsk information}
% bog 
Biomedicinsk information kan opdele i to kategoriser, herunder patient-specifik information og vidensbaseret information. Patient-specifik information gælder for individuelle patienter og har til formål at oplyse sundhedspersonale om patientens sygdom og helbred. Vidensbaseret information er afledt og organiseret ud fra observation eller eksperimentel forskning som kan viden føres til den individuelle patient. Information er ofte hentet i form af bøger eller journaler.  

\section{Beslutningsstøttesystemer}
%% bog
Giver relevant viden og analyser som gør det muligt for beslutningstagerne at udføre mere velinformerede vurderinger. \textbf{Osheroff et al. 2004} beskriver beslutningsstøttesystemer således: Giver den rigtige information til den rigtige person, i det rigtige format, gennem den rigtige kanal, på det rigtige tidspunkt i arbejdsgangen med henblik på at forbedre sundhed og omsorgsomkostninger og resultater. Systemer som leverer klinisk beslutningsstøttesystemer kommer i tre grundlæggende former, de kan bruge oplysninger om nuværende klinisk kontekst til at hente relevante online dokumenter, de kan give patientspecifikke, situationsspecifikke advarsler, påmindelser eller andre anbefalinger til direkte handling. Organiserer og præsenterer oplysninger så problemløsning og beslutningstagning via grafiske displays, dokumentation skabeloner. 

\section{Motivation for beslutningsstøttesystemer}
% bog
Interessen i beslutningsstøttesystemer kommer på baggrund af ønsket om at forbedre sundhedsplejen og til bedre forståelse af processen med medicinsk beslutningstagning. Et system som forsøger at behandle data som klinikeren foretager og muliggør oprette formelle modeller af klinisk ræsonnement. På samme tid tilbyder disse systemer indlysende samfundsmæssige fordele i forhold til at opnå bedre kliniske resultater. Anerkendelsen af betydning af beslutningsstøttesystemer er steget markant i nyere tid som føle af den utrolige vækst i sundhedsplejen, kompleksiteten og omkostninger samt introduktion af sundhedslovgivningen rettet mod at imødegå disse tendenser. 

\section{Forskellige systemer}
% ikke bog
F.T. deDombal har udviklet et computer-baseret beslutningsstøttesystem til diagnosticering af mavesmerter ved brug af Bayesian sandsynlighedsteori. Til at udlede betingede  sandsynligheder anvendes kirurgiske eller patologiske diagnoser som den gyldne standard, som blev indsamlet for tusindvis af patienter. Systemet anvendte data som følsomhed, specificitet og sygdomsprævalens for forskellige tegn, symptomer og testresultater til at beregne sandsynligheden for syv mulige forklaringer til akut mavesmerte. Den bayesiske formulering antager, at hver patient havde en af de syv tilstande og valgt den mest sandsynlige en på baggrund af de indspillede observationer. Hvis systemet og klinikerens diagnose blev sammenlignet var klinikerens diagnose kun korrekte i 65 til 80~\% baseret på 304 tilfælde, hvor systemets diagnoser var korrekt i 91,8~\% af tilfældene. Ligeledes var systemet bedre til at tildele patienterne i den rigtige sygdomskategori end en erfaren kliniker. Systemet viste dog ikke lignende resultater da det blev implementeret i andre klinikker. Denne uoverensstemmelse kan skyldes variationen i den måde en kliniker fortolker data som indtastes i computeren som f.eks. uenighed i kriterier for identifikation af visse patientresultater. En anden kan være at der er forskellige sandsynligheder mellem fund og diagnoser i forskellige patientpopulationer. 

Et andet eksempel på et computerbaseret beslutningsstøttesystem er MYCIN, som et regel-baseret konsultationssystem som kombinere diagnose med passende ledelse af patienter med infektioner. Udviklerne af MYCIN mente at ligefrem algoritmer eller statistiske metoder var utilstrækkelige, da underliggende viden var dårlig forstået og eksperterne var ofte uenige om, hvordan forskellige patienter skulle styres. På baggrund af dette anvendes et regel-baseret system som med sin brug af interagerede regler til at repræsentere viden om organismer som kan medvirke til infektion og den mulige behandling med antibiotika. 

Viden om smitsomme sygdomme i MYCIN blev repræsenteret som produktionsregler, som er en betinget erklæring der relaterer sig til observationer og tilhørende logik der kan trækkes. Beslutningen kan betegnes som input observation af andre regler når et system af andre regler når et system af regler benyttes til at redegøre.

Hjælp-systemet er et system som kan genere automatiserede advarsler når unormaler i patientjournaler blev noteret. Det er et overvågningsprogram som var bygget op omkring regler som vedrører værdierne af data i patientdatabasen til handlinger som sundhedspersonale kan blive mindet om at tage. Hver beslutnings regler et et medicinsk logik modul, som er en specialiseret form for begivenhed-betingelse-aktion regel, som anvendes i evalueringen af situationsspecifikke betingelser ekspressions logik udløses af en ekstern hændelse, hvis tilstanden vurderes at være sand, så udføres handlingen. 

\section{Nyere beslutningsstøttesystemer}
Nyere systemer opnår typiske deres resultater ved brug af bayesian sandsynlighedsterori, regler og medicin logik modul.

\subsection{Info-knapper}
Den mest simple form for beslutningsstøttesystem er info-knapper i elektronisk patientjournal, som kan give yderligere information via tilgængelige informationsressource for klinikkeren. Dette er for flere ikke anset som et beslutningsstøtte, da den viser relevant information for klinikere, men ikke hjælper til nogen form for beslutning. 

\subsection{Forgreningslogik}
%bog
Talrige af beslutningsstøttesystemer bygger på problemspecifikkke flowcharts designet af klinikere og kodet dem til brug af en computer. På trods af at disse algoritmer er nyttige til formålet er disse afvist af læger, da de har virket forsimplet eller generisk til rutinemæssigt brug. Fordelen ved deres implementering på computere har ikke være klart. Brugen af disse har generelt vist sig tilstrækkelig til klinisk pleje. Selvom flowchart alene ofte ikke er tilstrækkelig repræsentation af beslutningstagning, er algoritmiske repræsentation af kliniske procedurer yderst nyttig til klinikere når de tænker på repræsentation af foretrukne kliniske arbejdsgang. Det er derfor hyppigt set at forgreningslogik er repræsenteret i kliniske protokoller og retningslinjer som en komponenter i kompleksiteten, heterogene repræsentation af viden som er nødvendig for at køre sofistikerede beslutningsstøttesystemer.

\subsection{Sandsynglighedssystemer}
Stort antal af bayesiske diagnoseprogrammet er udviklet og mange af disse har vist sig at være præcise i at vælge blandt konkurrerende forklaringer på en patients sygdomsstadie. Selvom naive bayesian model kan have begrænsninger i nøjagtigt modellering af en diagnostisk problem er en stor styrke i denne tilgang beregningsevnen. Overveje en given finding, den forudgående sandsynlighed for hver mulig diagnose under overvejelse og de betingede sandsynligheder for finding givet hver diagnose. Efter dette anvendes Bayes-regler til at beregne den efterfølgende diagnose værdi af findingen. 

- belief networks - populære fordi formalisme gør sandsynligheden forhold perspektiv over vinder antagelsen om betinget uafhængighed og muliggør ledsagerens sandsynligheder at blive lært ved analyse af passende datasæt. 

Fordi de fleste beslutninger inden for medicin kræver vejning af omkostninger og fordele af handlinger, der kunne tages ved diagnosticering eller styre en patients sygdom. Forskere har også udviklede værktøjer som trækker på metoderne til beslutningsanalyse f.eks. beslutningstræ.  

Der er en del supervised learning teknikker som kan bestemme, hvordan data er forbundet via hypoteser og dermed kan trænes for at komme frem til en konklusion  baseret på input data. Regression analyse er mere sofistikeret teknik, som anvender kunstige neurale netwærk og støtte vektormaskine. Når disse anvendes på rutinemæssigt indsamlede patientdata kan dette anvendes som beslutningsstøtte til at forudsigelse.

\subsection{Regel-baseret}
Brugen af metoder som ligger vægt på symbolsk forbindelser snarere end rent sandsynlighedsberegninger til at drive beslutningsstøtte, hvilket har medført til opbygningen af viden-baserede systemer. Viden-baserede systemer er programmer som symbolisere koncepter afledt af eksperter inden for en bestemt viden og anvender denne viden til at levere den slags problemanalyse og rådgivning som eksperten kan give. If-then regler er ofte brugt til at opbygge vidensbaserede systemer, som har nyere tilgange, der kode eksplicit modeller af applikations området. Viden i videnbaseret systemer kan omfatte sandsynligheds relationer, som f.eks. mellem sygdomme. Typisk er disse relationer kvalitative relationer, såsom kausalitet og tidsmæssige forhold. Når et vidensbaseret system er kodet ved hjælp af produktion regler, det er benævnt som et regel-baseret system.

Regel-baserede systemer giver den dominerede mekanisme for udviklere til at opbygge beslutningsstøtte systemer kapaciteter in i moderne informationssystemer. Fra beslutningsstøttesystemer der fortolker EKG-signaler til anbefaling af retningslinjebaseret handling, regler giver et yderst bekvent middel til at kode den nødvendige viden. 
Regel-baserede systemer kræver et formelt sprog til kodning af reglerne, og en inference engine, som abrjder på reglerne for at genere den nødvendige adfærd. Et eksempel på en regel engine er JESS som er en populær java-baseret regel engine og er baseret på programmeringssproget C. 

I de fleste implementerede regel-baserede beslutningssøtte er meget enklere men også mere begrænset. 

\section{Klassificering og forudsigelse}
En af de simpleste metoder til klassificering er k-nearest-neighbor eller KNN. KNN bruger klassificering af de nærmeste forekomster til en given input, som et sæt af stemmer som beskriver, hvordan inputtet skal klassificeres.
KNN har desværre tendens til ikke at være nyttig for omics-baserede\fxnote{data i offentlige tilgængelige databaser er afgørende for design af eksperimenter og fortolkning af de opnåede resultater. Objektiv undersøgelse} klassificeringer, fordi det har tendens til at bryde ned i højdimensionele plads. For højdimensionele data har KNN vanskeligheder kultivere med at finde nok naboer til at forudsige, hvilket vil føre til stor variation i klassificering. 

En mere generel statistisk tilgang til supervised learning
som omfatter antallet af populære metoder er funktionen af tilnærmelse. I denne tilgang forsøger man at
finde en nyttig tilnærmelse af funktionen f (x)
der ligger til grund for den faktiske relation mellem
indgange og udgange. I det tilfælde vælges en metric som bedømmer nøjagtigheden af tilnærmelsen, f.eks. rest summen af kvadrater og bruger denne beregning til at optimere modellen til at oplyse træningsdatene. Bayesiske modellering, logistisk regression og Support Vector Machines bruger alle variationer ved denne tilgang.

Regelbaseret klassificering kan tænkes som
en række regler, som hver især sætter sæt af
forekomster baseret på en given karakteristik. Detaljer,
som hvilke kriterier der bruges til at vælge funktionen
hvorpå en regel skal baseres, og om
algoritmen bruger forbedringer ved flere modeller sammen bestemmer specifikationen af klassifikations typen, for eksempel beslutningstræer, random forest eller covering
regler.

Hvilken tilgang der tages i brug afhænger både af
karakteren af data og spørgsmålet. Spørgsmålet kan prioritere følsomhed over specificitet eller omvendt. For eksempel for en test til at opdage en livstruende infektion, der er let at behandle med let tilgængelige antibiotika vil man måske hellere have fejl i forhold  følsomhed. Derudover kan data være numeriske eller kategoriske eller har forskellige grader af støj, manglende værdier, korrelerede træk eller ikke-lineære interaktioner blandt funktioner. Disse forskellige kvaliteter er bedre håndteres af forskellige metoder. I mange tilfælde den bedste tilgang er faktisk at prøve en række forskellige metoder og at sammenligne resultaterne.


