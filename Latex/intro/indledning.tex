\chapter{Initierende problem}
Den stigende andel af ældre, forekomsten og varigheden af kroniske sygdomme samt udviklingen i sundhedsforventninger og teknologier er skyld i stigende sundhedsudgifter i flere europæiske lande~\citep{Ess2003}. I år 2015 brugte Danmark flere sundhedsudgifter end det gennemsnitlige land i Europa~\citep{EU2017}. Siden år 2007 til 2015 har udgifterne til sygehusmedicin steget 7,8~\% i gennemsnit om året~\citep{Sundhed2016}.

For at begrænse udgifterne har Amgros, Regionernes lægemiddelorganisation, siden år 2007 sendt lægemidler i udbud årligt med henblik på at indkøbe lægemidler af høj kvalitet til bedst mulige pris til de offentlige danske hospitaler~\citep{Sygehusapoteket2017}. Udbud forekommer på lægemidler, hvor der findes mere én leverandør, hvormed lægemidler bringes i konkurrence, hvilket kan give anledning til et kontraktskift~\citep{Amgros2015}. Foruden kontraktskift kan lægemiddelskift forekomme ved restordre, hvor efterspørgselen på et lægemiddel overstiger den tilgængelige mængde~\citep{Amgros2015}. Restordre kan skyldes leveringesvigt fra leverandøren eller producenten~\citep{Laegemiddelinformaion2017, Amgros2017}. 

Et lægemiddelskift kan defineres som simpel eller kompleks i forhold til, hvordan det påvirker afdelingen~\citep{Laegemiddelinformaion2017, Sygehusapoteket2017a}. Et simpel lægemiddelskift påvirker klinikken i lav grad og sker dagligt i forbindelse med skift til et simpel generisk lægemiddel. %Generisk lægemiddel indeholder samme aktive stof som det nuværende lægemiddel, men indeholder andre hjælpemidler. 
Komplekse lægemiddelskift påvirker klinikken i større grad og omhandler skift af  generiske lægemiddel, hvor flere faktorer afviger fra det nuværende lægemiddel som f.eks. styrke og disponeringsform.~\citep{Laegemiddelinformaion2017, Sygehusapoteket2017a}

Implementering af lægemiddelskift i klinikken skaber økonomiske og patientsikkerhedsmæssige udfordringer, der kan lede til utilsigtede hændelser (UTH'er)~\citep{Laegemiddelinformaion2017, Sygehusapoteket2017a}. De hyppigste årsager til UTH'er skyldes i år 2013 medicinering~\citep{Patientombuddet2013}. Af 824 rapporterede UTH'er i Region Nordjylland skyldes 97~\% af disse medicinering, 86~\% administration af medicin og 41~\% at intet medicin var givet til patienten~\citep{Jensen2014}.\fxnote{hvorfor giver det ikke 100 \%, fordi flere af de rapporterede indeholder flere årsager.}

Rapportering af UTH'er har til formål at forbedre patientsikkerheden ved at synliggøre problemstillinger og derved forbedre samt forebygge lignende rapportering af UTH'er~\citep{StyrelsenforPatientsikkerhed2017}. Udover rapportering har elektronisk beslutningsstøttesystem, stregkode scanning og klar-til-brug lægemidler  medført en reduktion i antallet af medicineringsfejl både internationalt og nationalt~\citep{Bates2013, Levtzion-korach2010, Amgros2012}.\fxnote{Lav en stærkere begrundelse der leder bedre frem.}  

På baggrund af dette er det relevant at analysere, hvilke problemstillinger der opstår i forbindelse med lægemiddelskift med henblik på at undersøge, hvorledes disse problemstillinger kan anvendes til at forebygge antallet af UTH'er.

%\citep{Bates2013, Cheng2011, Raboel2005, Poon2006, Bates2000,Levtzion-korach2010,Oren2003, Amgros2012}. 

%I periode fra 2007 til 2015 har udgifterne til sygehusmedicin steget i gennemsnit med 7,8~\% om året, hvilket svarer til en stigning på 3,5 milliarder kroner over en periode på 8 år ~\citep{Sundhed2016}.Dette er til trods for at Amgros, Regionernes lægemiddelorganisation, årligt sender lægemidler i udbud med henblik på at indkøbe de rigtige lægemidler til den bedst mulige pris til de offentlige danske hospitaler~\citep{Sygehusapoteket2017}. Udbuddene forekommer på lægemidler hvor der findes mere én leverandør, på denne måde bringes lægemidlerne i konkurrence, hvilket kan give anledning til kontraktskift~\citep{Amgros2015}.

%Foruden kontraktskift kan lægemiddelskift forekomme ved restordre, hvor efterspørgselen på et lægemiddel overstiger den tilgængelige mængde~\citep{Amgros2015}. Dette kan skyldes leveringesvigt fra leverandøren eller producenten og det er i disse tilfælde leverandørens ansvar, grundet kontrakten, at finde et erstatningslægemiddel~\citep{Laegemiddelinformaion2017, Amgros2017}. 

%Ved implementering af lægemiddelskift er der både økonomisk og patientsikkerhedsmæssige problematikker som kan påvirke afdelingen fra lav til mellem eller høj grad~\citep{Laegemiddelinformaion2017, Sygehusapoteket2017a}. Studie har vist at de hyppigste utilsigtede hændelser ved lægemidelskift omhandler fejlmedicinering~\citep{Hakonsen2010}. Yderligere er antallet af rapporterede utilsigtede hændelser i Region Nordjylland steget med over 36~\% fra år 2012 til 2014~\citep{Jensen2014}.
