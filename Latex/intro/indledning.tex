\chapter{Initierende problem}
Årligt sender Amgros, Regionernes lægemiddelorganisation, lægemidler i udbud med henblik på at indkøbe de rigtige lægemidler til den bedst mulige pris til de offentlige danske hospitaler~\citep{Sygehusapoteket2017}. Udbuddene forekommer på lægemidler hvor der findes mere én leverandør, på denne måde bringes lægemidlerne i konkurrence, hvilket kan give anledning til kontraktskift~\citep{Amgros2015}.

Foruden kontraktskift kan lægemiddelskift forekomme ved restordre, hvor efterspørgselen på et lægemiddel overstiger den tilgængelige mængde~\citep{Amgros2015}. Dette kan skyldes leveringesvigt fra leverandøren eller producenten. I disse tilfælde er det leverandørens ansvar, grundet kontrakten, at finde et erstatningslægemiddel~\citep{Laegemiddelinformaion2017, Amgros2017}. 

Ved implementering af lægemiddelskift er der både økonomisk og patientsikkerhedsmæssige problematikker som kan påvirke afdelingen fra lav til mellem eller høj grad~\citep{Laegemiddelinformaion2017, Sygehusapoteket2017a}. Studie har vist at de hyppigste utilsigtede hændelser ved kontraktskift var forkert lægemiddel samt forkert formulering~\citep{Hakonsen2010}. Yderligere er antallet af rapporterede utilsigtede hændelser i Region Nordjylland steget med over 36~\% fra år 2012 til 2014~\citep{Jensen2014}.

