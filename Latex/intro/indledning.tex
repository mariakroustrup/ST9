\chapter{Initierende problem}
Den stigende andel af ældre, forekomsten og varigheden af kroniske sygdomme samt udviklingen i sundhedsforventninger og teknologier er skyld i stigende sundhedsudgifter i flere europæiske lande~\citep{Ess2003}. Danmark brugte i år 2015 flere sundhedsudgifter end det gennemsnitlige land i Europa~\citep{EU2017}. Siden år 2007 til 2015 er udgifterne til sygehusmedicin steget med 7,8~\% i gennemsnit om året~\citep{Sundhed2016}.

For at begrænse udgifterne til sygehusmedicin har Amgros, Regionernes lægemiddelorganisation, siden år 2007 sendt lægemidler i udbud hvert år med henblik på at indkøbe lægemidler af høj kvalitet til bedst mulige priser til de offentlige danske hospitaler~\citep{Amgros2018b}. I år 2017 sparede Amgros regionerne for 3,1 milliarder kroner~\citep{Amgros2017b}. Udbud forekommer på lægemidler, hvor der findes mere end én leverandør, hvormed lægemidler bringes i konkurrence, hvilket kan give anledning til et kontraktskift~\citep{Amgros2015}.

Kontraktskift medfører substitution af lægemidler, hvilket betyder udskiftning at et lægemiddel til et andet lægemiddel~\citep{DanskSelskabforPatientsikkerhed2009}. %Det udskiftede lægemiddel forventes at have lignende effekt og omtrent ens bivirkninger.
%Der er to typer af substitution af lægemidler herunder analog og generisk. 
Analog substitution omhandler lægemidler som indeholder forskelligt aktivt stof, men forventes samme effekt og omtrent samme bivirkninger~\citep{Kairi2017}. Analog substitution kræver at lægen er involveret i ordinationen af lægemidlet~\citep{DanskSelskabforPatientsikkerhed2009}.
Generisk substitution omhandler lægemidler, som indeholder samme aktive stof, altså er hinandens synonyme~\citep{Kairi2017}. Generisk substitution foretages ofte og kræver modsat analog substitution ikke at en læge er involveret i ordinationen, hvormed dette kan varetages af en sygeplejerske~\citep{DanskSelskabforPatientsikkerhed2009}. 

Der er patientsikkerhedsmæssige konsekvenser forbundet med substitution, herunder fejlmedicinering~\citep{Hakonsen2010}. %Udover at 92~\% af sygeplejerskerne følte generisk substitution var tidskrævende, mente 91~\% at det øgede risikoen for fejl ved disponering~\citep{Hakonsen2010}. 
Den hyppigste fejl ved generisk substitution skyldes i 82,1~\% af tilfældene at sygeplejersken ordinerer det forkerte lægemiddel~\citep{Hakonsen2010}. De typiske anledninger til forket lægemiddel er forveksling af emballager eller navne, hvilket i nogle tilfælde har medført forlænget indlæggelse, forværret sygdom og dødsfald~\citep{DanskSelskabforPatientsikkerhed2009}. Implementering af lægemiddelskift i klinikken er derfor essentielt i forhold til forebyggelse af medicineringsfejl.

Flere studier har påvist at informationssystemer er anvendeligt i forebyggelsen af medicineringsfejl ved at støtte den kliniske beslutningstagning~\citep{Agrawal2009,Anderson2002}. 
Informationsteknologisystemer kan forbedre adgangen til en stor del af informationer, organisere disse og identificere sammenhænge mellem disse, hvilket er begrænset for et menneske~\citep{Agrawal2009}. Informationsteknologisystemer kan på denne måde vise klinikeren relevant information, hvormed beslutningen forbedres.~\citep{Agrawal2009}

De nuværende teknologier anvendes i forbindelse med bedre beslutningstagning i selve klinikken og derved forebyggelse af fejl~\citep{Agrawal2009, Kaushal2002, Stenner2010, Fischer2008, Simpson2008}. Det kunne derfor være essentielt at undersøge om en teknologi kan anvendes inden implementering af lægemiddelskift, med henblik på at synliggøre de risici der kan forekomme ved de enkelte skift og på denne måde forebygge medicineringsfejl og forbedre patientsikkerheden. Computer-baseret systemer er anvendeligt i forbindelse med forebyggelse af medicineringsfejl og forbedring af patientsikkerheden~\citep{Agrawal2009,Masys2006}. Herunder er regel-baseret systemer og machine learning anvendt at forudsige sandsynligheden for fejl ved risikovurdering~\citep{Geissert2018}. %Regel-baseret systemer er anvendt til simple problemstillinger, er hurtige at implementere og kræver minimalt data, men er begrænset når problemstilling bliver mere kompleks. I disse situationer anvendes machine learning, der er langsommere at implementere og kræver mere data.

På baggrund af dette er det relevant at analysere, hvordan et lægemiddelskift håndteres samt hvilke problemstillinger der opstår og hvordan disse kan forebygges. 


%*** HER MANGLER NOGET TEKST OG BEGYNDELSE FOR VALG AF TEKNOLGOI ***
%Beslutningsstøttesystemer har påvist at være anvendeligt i forebyggelsen af medicineringsfejl.

%Informationssystemer har påvist at være anvendeligt i forebyggelsen af medicineringsfejl~\citep{Agrawal2009}. Dette indebærer f.eks. computerbaseret lægeordreindgang, automatisk dispenseringsskabe, bar-kodet medicin administration samt elektronisk afstemning af medicin.~\citep{Agrawal2009} Klinisk beslutningsstøtte, som f.eks. i form af advarsler, har påvist at forbygge forveksling af navn og styrke og ligeledes at reducere antallet af medicineringsfejl~\citep{Campmans2018}.
%Den europæiske lægemiddelstyrelse har udviklet guidelines med henblik på at mindske antallet af navneforvekslinger~\citep{DanskSelskabforPatientsikkerhed2009}.  Stregkodescanning har ligeledes påvist at kunne reducere antallet af fejl i medicineringsprocessen~\citep{Poon2006,Levtzion-korach2010}, ved forebyggelse af emballageforveksling~\citep{DanskSelskabforPatientsikkerhed2009}.  Tall Man Lettering anvendes til skelne lignende lægemidler ved brug af store og små bogstaver~\citep{Campmans2018}. Yderligere er det påvist at klinisk beslutningsstøtte i form af advarsler ved mulig forveksling af navn og styrke bidrog til forebyggelse af medicineringsfejl~\citep{Campmans2018}.

%hvilke problemstillinger der opstår i forbindelse med lægemiddelskift med henblik på at undersøge, hvorledes disse problemstillinger kan forebygges for at undgå medicineringsfejl.

%%%% OLD OLD OLD OLD OLD %%%%%%%%%

%Den stigende andel af ældre, forekomsten og varigheden af kroniske sygdomme samt udviklingen i sundhedsforventninger og teknologier er skyld i stigende sundhedsudgifter i flere europæiske lande~\citep{Ess2003}. I år 2015 brugte Danmark flere sundhedsudgifter end det gennemsnitlige land i Europa~\citep{EU2017}. Siden år 2007 til 2015 har udgifterne til sygehusmedicin steget 7,8~\% i gennemsnit om året~\citep{Sundhed2016}.
%
%
%For at begrænse udgifterne har Amgros, Regionernes lægemiddelorganisation, siden år 2007 sendt lægemidler i udbud årligt med henblik på at indkøbe lægemidler af høj kvalitet til bedst mulige pris til de offentlige danske hospitaler~\citep{Sygehusapoteket2017}. Udbud forekommer på lægemidler, hvor der findes mere én leverandør, hvormed lægemidler bringes i konkurrence, hvilket kan give anledning til et kontraktskift~\citep{Amgros2015}. Foruden kontraktskift kan lægemiddelskift forekomme ved restordre, hvor efterspørgselen på et lægemiddel overstiger den tilgængelige mængde~\citep{Amgros2015}. Restordre kan skyldes leveringesvigt fra leverandøren eller producenten~\citep{Laegemiddelinformaion2017, Amgros2017}. 
%
%Et lægemiddelskift kan defineres som simpel eller kompleks i forhold til, hvordan det påvirker afdelingen~\citep{Laegemiddelinformaion2017, Sygehusapoteket2017a}. Et simpel lægemiddelskift påvirker klinikken i lav grad og sker dagligt i forbindelse med skift til et simpel generisk lægemiddel. %Generisk lægemiddel indeholder samme aktive stof som det nuværende lægemiddel, men indeholder andre hjælpemidler. 
%Komplekse lægemiddelskift påvirker klinikken i større grad og omhandler skift af  generiske lægemiddel, hvor flere faktorer afviger fra det nuværende lægemiddel som f.eks. styrke og disponeringsform.~\citep{Laegemiddelinformaion2017, Sygehusapoteket2017a}
%
%Implementering af lægemiddelskift i klinikken skaber økonomiske og patientsikkerhedsmæssige udfordringer, der kan lede til utilsigtede hændelser (UTH'er)~\citep{Laegemiddelinformaion2017, Sygehusapoteket2017a}. De hyppigste årsager til UTH'er skyldes i år 2013 medicinering~\citep{Patientombuddet2013}. Af 824 rapporterede UTH'er i Region Nordjylland skyldes 97~\% af disse medicinering, 86~\% administration af medicin og 41~\% at intet medicin var givet til patienten~\citep{Jensen2014}.\fxnote{hvorfor giver det ikke 100 \%, fordi flere af de rapporterede indeholder flere årsager.}
%
%Rapportering af UTH'er har til formål at forbedre patientsikkerheden ved at synliggøre problemstillinger og derved forbedre samt forebygge lignende rapportering af UTH'er~\citep{StyrelsenforPatientsikkerhed2017}. Udover rapportering har elektronisk beslutningsstøttesystem, stregkode scanning og klar-til-brug lægemidler  medført en reduktion i antallet af medicineringsfejl både internationalt og nationalt~\citep{Bates2013, Levtzion-korach2010, Amgros2012}.\fxnote{Lav en stærkere begrundelse der leder bedre frem.}  
%
%På baggrund af dette er det relevant at analysere, hvilke problemstillinger der opstår i forbindelse med lægemiddelskift med henblik på at undersøge, hvorledes disse problemstillinger kan anvendes til at forebygge antallet af UTH'er.
%
%%\citep{Bates2013, Cheng2011, Raboel2005, Poon2006, Bates2000,Levtzion-korach2010,Oren2003, Amgros2012}. 
%
%%I periode fra 2007 til 2015 har udgifterne til sygehusmedicin steget i gennemsnit med 7,8~\% om året, hvilket svarer til en stigning på 3,5 milliarder kroner over en periode på 8 år ~\citep{Sundhed2016}.Dette er til trods for at Amgros, Regionernes lægemiddelorganisation, årligt sender lægemidler i udbud med henblik på at indkøbe de rigtige lægemidler til den bedst mulige pris til de offentlige danske hospitaler~\citep{Sygehusapoteket2017}. Udbuddene forekommer på lægemidler hvor der findes mere én leverandør, på denne måde bringes lægemidlerne i konkurrence, hvilket kan give anledning til kontraktskift~\citep{Amgros2015}.
%
%%Foruden kontraktskift kan lægemiddelskift forekomme ved restordre, hvor efterspørgselen på et lægemiddel overstiger den tilgængelige mængde~\citep{Amgros2015}. Dette kan skyldes leveringesvigt fra leverandøren eller producenten og det er i disse tilfælde leverandørens ansvar, grundet kontrakten, at finde et erstatningslægemiddel~\citep{Laegemiddelinformaion2017, Amgros2017}. 
%
%%Ved implementering af lægemiddelskift er der både økonomisk og patientsikkerhedsmæssige problematikker som kan påvirke afdelingen fra lav til mellem eller høj grad~\citep{Laegemiddelinformaion2017, Sygehusapoteket2017a}. Studie har vist at de hyppigste utilsigtede hændelser ved lægemidelskift omhandler fejlmedicinering~\citep{Hakonsen2010}. Yderligere er antallet af rapporterede utilsigtede hændelser i Region Nordjylland steget med over 36~\% fra år 2012 til 2014~\citep{Jensen2014}.
