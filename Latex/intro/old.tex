%Medicinrådet beslutter efter prisforhandlingen, hvorvidt det nye lægemiddel skal anvendes som standardbehandling. Hvis dette er tilfældet foretages et kontraktskift.~\citep{Amgros2017, Amgros2017a}

%og udbud på lægemidler sker har Amgros en dialog med Medicinrådet~\citep{Amgros2017, Amgros2017a}. Lægemidlerne vurderes i forhold til effekt, eksisterende behandling og pris med det formål at stræbe efter laveste priser samt bedst mulig behandling for patienterne.~\citep{Amgros2017, Amgros2017a}

%Medicinrådet kategoriserer nye lægemidler og indikationer i forhold til den nuværende standardbehandling ud fra merværdi i stor, vigtig, lille eller ingen merværdi~\citep{Amgros2017, Amgros2017a}. Dette sammenstilles med standardbehandlingen af Amgros over en omkostningsanalyse der er udarbejdet af leverandøransøgeren for lægemidlet.
%Amgros vurderer, hvorvidt de tilsendte  oplysninger i omkostningsanalysen er relevante og valide.~\citep{Amgros2017, Amgros2017a} 

%Den kliniske merværdi, omkostningsanalysen og estimeringen af budget konsekvenser danner grundlaget for prisforhandling~\citep{Amgros2017, Amgros2017a}. Medicinrådet beslutter efter prisforhandlingen, hvorvidt det nye lægemiddel skal anvendes som standardbehandling. Hvis dette er tilfældet foretages et kontraktskift.~\citep{Amgros2017, Amgros2017a} 



%20,86~\% kliniske processor, 17,98~\% kommunikation og dokumentation samt 17,23~\% administrative processer~\citep{Patientombuddet2013}.


%Restordre kan kategoriseres som simpel eller kompleks i forhold til hvordan de påvirker klinikken~\citep{Laegemiddelinformaion2017}. En simpel restordre er vurderet til at påvirke klinikken i lav grad. Disse opstår dagligt når et lægemiddel skiftes til et simpel generisk lægemiddel og varetages ofte af sygehusapotekets logistik-afdeling. ~\citep{Laegemiddelinformaion2017}
%
%En kompleks restordre vurderes til at påvirke klinikken i mellem til høj grad ~\citep{Laegemiddelinformaion2017}. Disse sker i forbindelse med mere komplekse skift til generiske lægemidler i forbindelse med ændringer af f.eks. styrke, disponeringsform  og andre hjælpestoffer. I tilfælde af kompleks restordre henvendes der ofte kontakt til den medicinansvarlige, kontraktsygeplejersker eller medicinservicefarmakonomerne i forhold til at undersøge om lægemidlet er anvendeligt for det pågældende hospitalsafsnit. Hensigten i tilfælde af komplekse restordre er at finde en erstatning i god tid, at erstatning ligner det lægemiddel der er i restordre samt mindske ændringer ved klinikkens arbejdsgang.~\citep{Laegemiddelinformaion2017}
%
%Der udføres en faglig risikovurdering i samarbejde med sygehusapoteket og eventuelt i samarbejde klinikken ved restordre ~\citep{Laegemiddelinformaion2017}. Dette gøres med henblik på optimal lægemiddelbehandling i forhold til patientsikkerhed, ændringer i håndtering og opbevaringsbetingelser af lægemidlet. Erstatningslægemidlet vurderes ud fra prioriteringen i kategorierne registeret specialitet (RS), ikke-registeret specialist (IRS) eller magistrelt lægemiddel~\citep{Laegemiddelinformaion2017}.
