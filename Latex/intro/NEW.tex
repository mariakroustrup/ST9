\chapter{Initierende problem}
Udgifterne til medicin er steget i flere europæiske lande grundet den stigende andel af ændre, forekomsten og varigheden af kroniske sygdomme, udviklingen af sundhedsteknologier og de sundhedsforventninger der er af patienter og samfund~\citep{Ess2003}. I år 2015 brugte Danmark flere sundhedsudgifter end EU-gennemsnittet~\citep{EU2017}. I periode fra 2007 til 2015 har udgifterne til sygehusmedicin steget i gennemsnit med 7,8~\% om året, hvilket svarer til en stigning på 3,5 milliarder kroner over en periode på 8 år ~\citep{Sundhed2016}.



Dette er til trods for at Amgros, Regionernes lægemiddelorganisation, årligt sender lægemidler i udbud med henblik på at indkøbe de rigtige lægemidler til den bedst mulige pris til de offentlige danske hospitaler~\citep{Sygehusapoteket2017}. Udbuddene forekommer på lægemidler hvor der findes mere én leverandør, på denne måde bringes lægemidlerne i konkurrence, hvilket kan give anledning til kontraktskift~\citep{Amgros2015}.

Foruden kontraktskift kan lægemiddelskift forekomme ved restordre, hvor efterspørgselen på et lægemiddel overstiger den tilgængelige mængde~\citep{Amgros2015}. Dette kan skyldes leveringesvigt fra leverandøren eller producenten og det er i disse tilfælde leverandørens ansvar, grundet kontrakten, at finde et erstatningslægemiddel~\citep{Laegemiddelinformaion2017, Amgros2017}. 

Ved implementering af lægemiddelskift er der både økonomisk og patientsikkerhedsmæssige problematikker som kan påvirke afdelingen fra lav til mellem eller høj grad~\citep{Laegemiddelinformaion2017, Sygehusapoteket2017a}. Studie har vist at de hyppigste utilsigtede hændelser ved lægemidelskift omhandler fejlmedicinering~\citep{Hakonsen2010}. Yderligere er antallet af rapporterede utilsigtede hændelser i Region Nordjylland steget med over 36~\% fra år 2012 til 2014~\citep{Jensen2014}.
