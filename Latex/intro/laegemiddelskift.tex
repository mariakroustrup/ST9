\section{Lægemiddelskift}
Lægemiddelskift forekommer når en ny lægemiddelvirksomhed vinder leverancen af nyt lægemiddel som standardbehandling, hvormed der er indgås et kontraktskift~\citep{Amgros2015}. 

Forinden et kontraktskift kan forekomme analyseres og vurderes lægemidlet i samarbejde med Medicinrådet og Amgros~\citep{DanskeRegioner2016}, som beskrevet i Appendiks \ref{cha:AppA}. Efter analyseringen og vurdering sendes lægemidlerne i udbud via Amgros med henblik på at indkøbe lægemidler af høj kvalitet til bedst mulige pris~\citep{Sygehusapoteket2017}.

Størstedelen af lægemidler i ATC-grupper sendes i udbud en gang årligt fra start september til midt november, hvor udbud på ATC-grupper som indgår i Medicinrådets behandlingsvejledninger sker løbende hen over året~\citep{Sygehusapoteket2017}, som beskrevet i Appendiks \ref{cha:AppD}.
Forinden udbuddet defineres antallet af vindere samt, hvorvidt udbuddet skal bygge på laveste pris eller være mest økonomisk fordelagtigt~\citep{Amgros2018a}, som beskrevet i Appendiks \ref{cha:AppB}.

På baggrund af de foregående analyser og vurderinger fastsættes et prisniveau der anvendes som beslutningsgrundlag for Medicinrådet om, hvorvidt lægemidlet skal anvendes som standardbehandling~\citep{DanskeRegioner2016}. Hvis standardbehandlingen er det eneste lægemiddel inden for terapiområdet kan dette implementeres direkte på hospitalet. I tilfælde af flere lægemidler inden for samme terapiområde, skal lægemidlernes ligeværdighed vurderes af Medicinrådet, som beskrevet i Appendiks \ref{cha:AppA}, med henblik på at udarbejde behandlingsvejledninger og rekommendationer for lægemidlerne. Disse videresendes til de danske hospitaler, som står for implementeringen.~\citep{DanskeRegioner2016}

Foruden kontraktskift kan lægemiddelskift forekomme, hvis efterspørgslen på et lægemiddel overstiger den tilgængelige mængde, hvormed der opstår restordre~\citep{Amgros2015}. Restordre kan f.eks. opstå ved leveringesvigt fra leverandøren eller producenten på det ønskede lægemiddel~\citep{Amgros2017, Laegemiddelinformaion2017}. Leveringesvigt skyldes som ofte at producenten har mangel på råvarer eller produktionsvanskeligheder~\citep{Amgros2017, Laegemiddelinformaion2017}. I tilfælde  af restordre er det leverandørens ansvar at dække sygehusapotekernes udgift ved indkøb af erstatningslægemiddel.~\citep{Laegemiddelinformaion2017, Amgros2017}