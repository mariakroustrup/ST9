\section{Lægemiddelskift}
Lægemiddelskift forekommer når en ny lægemiddelvirksomhed vinder leverancen af nyt lægemiddel som standardbehandling, hvormed der er indgås et kontraktskift~\citep{Amgros2015}. 

Forinden et kontraktskift kan forekomme analyseres og vurderes lægemidlet i samarbejde med Medicinrådet og Amgros~\citep{DanskeRegioner2016}, som beskrevet i Appendiks \ref{cha:AppA}. Efter analyseringen og vurdering sendes lægemidlerne i udbud via Amgros med henblik på at indkøbe lægemidler af høj kvalitet til bedst mulige pris~\citep{Sygehusapoteket2017}.

Størstedelen af lægemidler i ATC-grupper sendes i udbud en gang årligt fra start september til midt november, hvor udbud på ATC-grupper som indgår i Medicinrådets behandlingsvejledninger sker løbende hen over året~\citep{Sygehusapoteket2017}, som beskrevet i Appendiks \ref{cha:AppD}.
Forinden udbuddet defineres antallet af vindere samt, hvorvidt udbuddet skal bygge på laveste pris eller være mest økonomisk fordelagtigt~\citep{Amgros2018a}, som beskrevet i Appendiks \ref{cha:AppB}.

På baggrund af de foregående analyser og vurderinger fastsættes et prisniveau der anvendes som beslutningsgrundlag for Medicinrådet om, hvorvidt lægemidlet skal anvendes som standardbehandling~\citep{DanskeRegioner2016}. Hvis standardbehandlingen er det eneste lægemiddel inden for terapiområdet kan dette implementeres direkte på hospitalet. I tilfælde af flere lægemidler inden for samme terapiområde, skal lægemidlernes ligeværdighed vurderes af Medicinrådet, som beskrevet i Appendiks \ref{cha:AppA}, med henblik på at udarbejde behandlingsvejledninger og rekommendationer for lægemidlerne. Disse videresendes til de danske hospitaler, som står for implementeringen.~\citep{DanskeRegioner2016}

Kontraktskift medfører substitution af lægemidler, hvor et lægemiddel udskiftes til et andet~\citep{DanskSelskabforPatientsikkerhed2009}. Der findes to typer af substitution herunder analog og generisk.
Analog substitution omhandler lægemidler der indeholder forskelligt aktivt lægemiddelstof, har nogenlunde ens effekt og bivirkninger. Disse kræver ændring i ordination og skal derfor ordineret af en læge. Generisk substitution omhandler lægemidler som indeholder det samme virksomme lægemiddelstof og fungere derfor som hinandens synonyme. Dette skift kræver ikke en ændring i recept og kan derfor varetages af en sygeplejerske.~\citep{DanskSelskabforPatientsikkerhed2009}

Et simpelt lægemiddelskift er vurderet til at påvirke hospitalsafdelingen i lav grad og varetages logistik afdeling~\citep{Laegemiddelinformaion2017, Sygehusapoteket2017a}. Disse skift sker dagligt i forbindelse med et simpel generisk lægemiddelskift. Hvorimod et kompleks lægemiddelskift påvirker klinikken i mellem til høj grad og kræver ofte involvering af flere interessenter som f.eks. medicinansvarlig, overlæger, kontaktsygeplejersker eller medicinservicefarmakonomer til at undersøge lægemidlets anvendelighed for det pågældende hospitalsafsnit. 
De komplekse skift sker ved generiske lægemidler, hvor flere faktorer som f.eks. styrke og disponeringsform afviger fra den nuværende behandling.~\citep{Laegemiddelinformaion2017,Sygehusapoteket2017a}
