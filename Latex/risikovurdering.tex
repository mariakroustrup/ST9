\chapter{Riskovurdering}

\section{Riskofaktor}
Risiko er en sandsynlighed for skade som er forårsaget af sårbarhed og som forventes at kunne ændres gennem forbyggende handlinger~\citep{Alam2016}. De to definitioner af risiko faktorer er at de kan ses som en type af sammenhænge, som er associeret med en stigende sandsynlig af et resultat og anvendes til at dele populationen i en høj eller lav risiko gruppe~\citep{Offord2000}. Der findes tre typer af risikofaktorer, fastlagt markør, variable risikofaktor og tilfældig risikofaktor. Fastlagte markører er f.eks. køn, etnicitet, og fødselsdag. Hvis faktorerne kan ændres spontant er det en variable risikofaktorer, men hvis denne faktor ikke ændre resultatet kaldes dette en variable markør. Hvis risikofaktoren ændrer resultatet når denne faktor ændres kaldes dette en tilfældig risikofaktor.~\citep{Offord2000}

Udvalgte faktorer er valgt på baggrund af litteratur og tidligere studier.

I forhold til at bestemme risikofaktorer anvendes statistik til at se om der er en statistisk sammenhæng mellem faktorerne. Multivariate logistisk regression/standard afvigelse, fordeling af data, eller korrelation analyse. 



\section{Risiko model}
En risiko model er et statistisk procedure for 

\section{Ekspertsystem}