\chapter{Metode}
\textit{I dette kapitel beskrives formålet med udviklingen af et system til risikovurdering af lægemiddelskift, hvordan data er indsamlet samt processen der gennemgås ved udviklingen af et system til risikovurdering af lægemiddelskift.}

\section{Formål}
Den nuværende risikovurdering af kompleksiteten af implementering af lægemiddelskift varetages af ATC-ansvarlige på baggrund af tidligere erfaringer, retningslinjer, indsamlede problemstillinger vedrørende lægemiddelskift samt viden omkring lægemidler indsamlet via f.eks. pro.medicin. ATC-områderne er uddelt på de forskellige ATC-ansvarlige, så hver ATC-ansvarlig har hvert deres område. I vurderingen af lægemiddelskiftet vægtes flere faktorer, som f.eks. navn, styrke og dispenseringsform, af én ATC-ansvarlig for området, hvormed denne proces er personafhængig og stiller visse krav til den ATC-ansvarliges viden og erfaring inden for området. Vægtningen udføres i forhold til at vurdere risikoen ved implementering af lægemiddelskift på hospitalsafdelingerne. Ved lægemidler af høj risiko, hvilket vil sige at det er et kompleks skift, udarbejdes et Lægemiddel Nyt Tema omkring skift, hvilket fremgår af Appendiks \ref{cha:AppD} nummer \ref{item:Laegemiddelnyt}. Dette udarbejdes med henblik på at synliggøre i hvilke tilfælde hospitalsafdelingerne skal være særlig opmærksomme på lægemiddelskiftet i forhold til at undgå fejlmedicinering som f.eks. kan forårsages af forveksling ved ændringer i lægemiddelnavn.
For at imødekomme disse problemstillinger ønskes det at udvikle et computerbaseret system til risikovurdering af lægemiddelskift som gør den nuværende proces mindre personafhængig og derved danner et bedre grundlag for risikovurdering af lægemiddelskiftet. 


%På nuværende tidspunkt vurderes kompleksiteten af implementering af lægemiddelskift af ATC-ansvarlige ud fra tidligere erfaringer og viden indsamlet via bl.a. pro.medicin. Flere faktorer vægtes ved vurderingen, hvilket gør denne proces meget personafhængig og stiller visse krav til den ATC-ansvarliges viden og erfaring inden for området. Vurderingen der foretages har betydning for implementeringen af lægemiddelskiftet i klinikken og det er derfor vigtigt at den rette vurdering foretages i forhold til at undgå medicineringsfejl som f.eks. forveksling af navnet på lægemidlet grundet ændringer ved lægemiddelskift. For at imødekomme disse problemstillinger ønskes det at udvikle et beslutningsstøttesystem til den ATC-ansvarlige som foretager risikovurderingen af lægemiddelskift med henblik på at synliggøre de ændringer som sker ved et eventuelt lægemiddelskift, hvormed den ATC-ansvarlige har et bedre beslutningsgrundlag.

\section{Dataindsamling}
Data er indsamlet af Amgros og omhandler de aftaler som er indgået i forbindelse med Amgrosudbud. Dette omfatter både forlængelser af allerede eksisterende kontrakter samt ændringer forårsaget af kontraktskift. Sygehusapoteket Region Nordjylland (SRN) udtrækker denne data fra sygehusapoteksportalen og sorterer i forhold til relevans. Da den data der udtrækkes fra portalen indeholder alle de aftaler som er indgået er det nødvendigt at sortere dette for relevant indhold for den pågældende udbudsperiode. Denne sortering varetages af en ansat på SRN som udarbejder skiftelister ud fra proceduren, der er beskrevet i Appendiks \ref{cha:AppD} nummer \ref{item:Skiftelister}, og herefter sendes disse til de forskellige ATC-ansvarlige.

Der tages udgangspunkt i data fra disse skiftelister, som bygger på skiftelister udarbejdet for lægemiddelskift i år 2015, 2016, 2017 og 2018. Dataen indeholder oplysninger om lægemiddelnavn, dispenseringsform og styrke for det fortgående år samt året hvor skiftet er forgået. 
Skiftelisterne kombineres med udtræk fra sygehusapoteksportalen omkring andre forhold som sygehusapotekets indkøbspris per enhed (SAIP/enhed) samt hvorvidt lægemiddelskiftet skyldes Medicinrådets behandlingsvejledning. Dette kombineres yderligere med data fra Amgros omhandlende risikolægemidler og indsamlet data fra SRN omkring kritiske ATC-koder.



%Data er indsamlet af Amgros og Sygehusapoteket Region Nordjylland (SRN) og omhandler lægemiddelskift i år 2018. Data fra Amgros omhandler oplysninger om lægemiddelskiftet foretaget fra år 2017 til 2018. Yderligere har Amgros udarbejdet et dokument over risikolægemidler, som fremgår af Appendiks \ref{cha:AppD}. Risikolægemidler er lægemidler som er særligt risikofyldte hvis disse ender i restordre, da erstatninger er svære at finde samt risikoen for fejl øges ved erstatning. Data fra SRN omhandler de udbud af lægemidler Amgros foretog inden et eventuelt kontraktskift for år 2018. 
%Udover dette har SRN ligeledes indsamlet de problemstillinger der har været vedrørende Amgrosskift siden år 2012.  


\section{Udviklingsproces}
Udviklingen af system til risikovurderingen af lægemiddelskift bygger på forskellige processer herunder valg af features, præprocessering, design, implementering og test. Udviklingstrinene for denne proces fremgår af Figur \ref{fig:metode}.

\begin{figure}[H]\centering	\includegraphics[width=1\textwidth]{billeder/udviklingstrin.png} 
	\caption{Udviklingstrin for beslutningsstøttesystem}
	\label{fig:metode}  
\end{figure}
\vspace{-0.5cm}

Af Figur \ref{fig:metode} illustreres de forskellige processer som der gennemgås ved udviklingen af et system til risikovurdering af lægemiddelskift. Udvælgelsen af features danner grundlag for de inputs som vægtes i forhold til risikovurderingen. Da data ikke er homogen ønskes det at foretage præprocessering af data for at gøre data sammenligneligt. Designfasen bygger på design af algoritmen som danner grundlag for risikovurderingen. Designet verificeres i løbet af processen for at opnå det bedst mulige design. Herefter implementeres systemet, hvilket omfatter implementering af designet herunder kodning af algoritmen. Til sidst foretages en test af systemet i forhold til evaluere hvorvidt systemet performer efter formålet, hvis dette ikke er tilfældet vendes der tilbage til de fortgående udviklingstrin i forhold til at omformulere i  præprocessingsfasen, redesigne i algoritmen i designfasen eller tilpasse koden i implementeringensfasen. 



%Første trin omhandler identificering af problemer som systemet skal løse, herunder data som systemet skal arbejde på og tilgængelige ressourcer~\citep{Ligeza2006}. Det andet trin er omhandler identificering af nøglekoncepter samt relationer mellem disse som f.eks. typer af data, informationsstrøm og underliggende stukturer. Tredje trin involverer forståelse, beskrivelse og formalisering af problemet og hvordan løsninger findes. Denne proces bør omfatte verificering af systemet. Det fjerde trin har til formål at implementere den formaliseret viden i et program. Det sidste trin omfatter test ved validering af regler og implementeringen.~\citep{Ligeza2006}


\chapter{Systemudvikling}
\section{Valg af features}
Features er udvalgt på baggrund af retningslinjer for den ATC-ansvarlige, som fremgår af Appendiks~\ref{cha:AppD}, og litteratur som beskriver ændringer i forskellige faktorer som har ledt til medicineringsfejl i klinikken som følge af lægemiddelskift. Derudover er information omkring risikolægemidler og kritiske ATC-koder anvendt. Disse er indsamlet af Amgros eller dokumenteret af SRN i forbindelse med Amgrosudbud. Features og begrundelse for valg af features fremgår af Tabel \ref{table:features}.

\begin{longtable}{p{3.5cm}| p{10.5cm}}
	\caption{Valg af features}
	\label{table:features} \\
\cellcolor[HTML]{C0C0C0} {\textbf{Feature og deres vægtning}} & \cellcolor[HTML]{C0C0C0} {\textbf{Begrundelse}}  \\ \hline
\textbf{Navn  \newline Vægtes med 1} &  Forveksling ved ændringer i navn på lægemidler er en af de hyppigste årsager til medicineringsfejl jævnfør afsnit \ref{sec:ProblemLaeg} [X]. Forvirring over ikke at kunne finde lægemidlet fordi navnet har ændret sig \\  \hline 
\textbf{Lignende navne} &  Forveksling ved ændringer i navn på lægemidler er en af de hyppigste årsager til medicineringsfejl jævnfør afsnit \ref{sec:ProblemLaeg}. [X] Disse forekommer grundet allerede eksisterende lægemiddel jævnfør afsnit \ref{sec:ProblemLaeg}. [X]. \\  \hline 
\textbf{Dispenseringsform} &  FIND BEGRUNDELSE FOR DETTE  \\ \hline 
\textbf{Styrke} &  Forkert dosis er dokumenteret se kilde 18,19,20 og 21  \\ \hline
\textbf{Risikolægemidler} & Disse lægemidler kan være problematiske, hvis de ender i restordre, altså at lægemidlet ikke kan leveres og det derfor er nødvendigt at finde et erstatningslægemiddel. Disse lægemidler bør være på lager, da det er svært at finde erstatninger for disse samt risikoen for fejl øges ved erstatning. Beskrevet i Appendiks ~\ref{cha:AppD}[FIND KILDE, eller indsæt i appendiks]  \\ \hline 
\textbf{ATC-koder} &  Der er nogle ATC-koder som er mere komplekse end andre. Dette er f.eks. på områder som indgår i produktionen på Sygehusapoteket Region Nordjylland og væsker, hvor en ændring i leverandør typisk vil forårsage ændringer af device, hvilket kan skabe problemer i klinikken. [KILDE] \\ \hline 
\textbf{Medicinråd} & Nogle lægemidler omhandler medicinrådets behandlingsvejledninger. Disse lægemidler omfatter enkelte hospitalsafdelinger og kræver et tæt samarbejde med disse. Det ønskes ligeledes at lægemidlet implementeres hurtigt, da det er muligt at opnå store besparelser. [KILDE + APPENDIKS]\\ \hline 
\textbf{Pris} &  Pris er vigtigt i forhold til at opnå besparelser. Det skal ligeledes vægtes om det kan betale sig at skifte et lægemiddel med mindre besparelse, da udskiftningen kan få store betydninger for klinikken i forhold til arbejdsgangen og i værste tilfælde medføre patientsikkerhedsmæssige konsekvenser. [KILDE, AMGROS] \textcolor{blue}{Det er endnu ikke besluttet, hvordan denne faktor skal indgå i algoritmen.} \\ \hline
    \end{longtable}

\section{Præprocessering}
Data er manuelt indskrevet og indeholder tekst og er derfor ikke sammenligneligt, hvorfor data er præprocesseret før anvendelsen. Da det er forskelligt om data er skrevet med majuskel eller minuskel er det valgt at ændre alt data til minuskel. Noget data er skrevet med forkortelser hvorfor det er valgt at fjerne forkortelser og udskrive ordene med henblik på at gøre data generaliserbar. 

Der er derudover lavet antagelser for navn og dispenseringformer for lægemidler. Det er valgt kun at kigge på præfiks for alle lægemidler, hvorfor suffiks er fjernet fra alle lægemidler. Der er dog nogle lægemidler  hvor ændring i præfiks betyder ændringer i styrke og dispenseringsform. Det er dog antaget at denne tages højde for i sammenligning af ændringer i styrke og dispenseringsform. Ligledes antages det for dispenseringsformer som tabletter at disse er ens hvis dispenseringsform er en af følgende; filmovertrukne, overtrukne eller  tabelletter. \textcolor{blue}{Om der er tale om væske og opløselig væske. Tilføj også for styrke når dette er undersøgt. Derudover bliver der også fjernet forlængelser som det første - skriv dette til. Der er fjernet punktummer og komma og skråstreger...}

\textcolor{red}{Ved ikke om det giver mening at gøre dette mere konkret her ved at komme med eksempler eller om jeg godt kan vente med dette til implementering og vise det der.Synes at dette vil give mere mening.}

\section{Design}

\begin{equation}
Riskoscore = \frac{\mbox{\textit{Summen af vægten af ikke matchende features}}}{\mbox{\textit{Summen af alle vægtede features}}} * 100
\end{equation}


%\begin{equation} \label{eq1}
%\begin{split}
%Risikoscore = lægemiddelnavn + ligende lægemiddelnavn + dispenseringsform \\ 
%+ styrke + medicnråd + risikolægemiddel + kritisk ATC-kode \\
%+ pris
%\end{split}
%\end{equation}

\section{Implementering}
Det er valgt at implementere i NetBeans som er et integreret udviklingsmiljø (IDE) til Java. Der er tilføjet biblioteker som JExcelApi og Apache POI til håndtering af Microsoft dokumenter herunder excel.

\section{Test}
Evaluering af systemet foretages i forhold til at vurdere brugbarheden af systemet. Dette vurderes af brugeren for systemet, hvilket er de ATC-ansvarlige på SRN. \textcolor{red}{Tænker at disse skal kigge på outputtet og vurdere om dette kan anvendes til dem og om det lettere noget af det ansvar de sidder med}

\chapter{Resultat}



