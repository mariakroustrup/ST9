\chapter{Resultat}
\vspace{-0.3cm}
\textit{I dette kapitel beskrives resultatet for evaluering af systemets anvendelighed i forhold til at besvare problemformuleringen. I denne forbindelse er systemets risikovurdering, risikoscore og risikofaktorer og vægtning vurderet.}

%\section{Systemets anvendelighed}
%Til evaluering af systemets anvendelighed i forhold til risikovurdering af lægemiddelskift var der repræsenteret 11 medarbejdere fordelt på forskellige afdelinger fra SRN, 8 fra Lægemiddelinformation, 2 fra Medicinservice og 1 fra Klinisk Farmaci. % Systemet blev vurderet af medarbejderne, til at være et godt udgangspunkt for et hjælpeværktøj til risikovurdering af lægemiddelskift, jævnfør Appendiks \ref{App:Referat}. 
%Resultatet af evalueringen er opdelt i forhold til risikovurderingen, risikoscore og risikofaktorer samt deres vægtning.

\section{Risikovurdering}
Ud fra vurderingerne foretaget af 11 medarbejdere fra SRN er der defineret en Golden Standard. Denne er bestemt ud fra, at over 60~\% af medarbejderne var enige i at lægemiddelskiftet krævede uddybende information. Medarbejdernes vurdering sammenlignes med Lægemiddel Nyt i forhold til teste systemets sensitivitet og specificitet. Vurderingerne af, hvorvidt lægemiddelskift krævede uddybende information via Lægemiddel Nyt, ved sammenligning af Lægemiddel Nyt med Golden Standard for medarbejdernes vurdering fremgår af Tabel \ref{table:test1}, hvor de individuelle vurderinger af medarbejderne fremgår af Tabel \ref{table:resultat} i Appendiks \ref{App:Resultat}.

\begin{table}[H]
\caption{Vurderinger fra Lægemiddel Nyt og Golden Standard for medarbejderne i forhold til lægemiddelskift, som kræver uddybende information. Uenighed er markeret med gult og en vurdering under 60~\% enighed mellem medarbejderne er markeret med rødt.}
\label{table:test1}
\centering
\begin{tabular}{l|c|c|c|c|c|c|c|c|c|c|c|c|c|c|c|c|c}
\rowcolor[HTML]{C0C0C0} \textbf{} & \multicolumn{11}{|c|}{\textbf{Lægemiddelskift nummer}} \\
\rowcolor[HTML]{C0C0C0} & \textbf{1} & \textbf{2} & \textbf{3} & \textbf{4} & \textbf{5} & \textbf{6} & \textbf{7} &  \textbf{8} & \textbf{9} & \textbf{10} & \textbf{11}   \\ \hline
%\textbf{Risikoscore} & 0 & 0 & 5 & 5 & 5 & 5 & 5 & 5 & 10 & 10 & 10  \\ \hline
\textbf{Lægemiddel Nyt} & nej & nej & nej & nej & nej & nej & nej & nej & ja &\cellcolor[HTML]{F6F6C3} ja & nej \\ \hline
\textbf{Golden Standard} & nej & nej & nej& nej & nej &nej & nej & nej& ja & \cellcolor[HTML]{F6F6C3}nej & nej \\ \hline
\rowcolor[HTML]{C0C0C0} & \multicolumn{11}{|c|}{\textbf{Lægemiddelskift nummer}} \\
\rowcolor[HTML]{C0C0C0} & \textbf{12} & \textbf{13} & \textbf{14} &  \textbf{15} & \textbf{16} & \textbf{17} & \textbf{18} & \textbf{19} & \textbf{20} & \textbf{21} & \textbf{22}  \\ \hline
%\textbf{Risikoscore} & 15 & 15 & 15 & 15 & 20 & 20 & 20 & 25 & 25 & 25 & 30 \\ \hline
\textbf{Lægemiddel Nyt} & nej & nej & \cellcolor[HTML]{F6F6C3}nej & nej & nej & nej & \cellcolor[HTML]{F6F6C3}ja & \cellcolor[HTML]{F6F6C3}nej & \cellcolor[HTML]{F6F6C3}ja & \cellcolor[HTML]{F6F6C3}nej & \cellcolor[HTML]{F6F6C3}nej\\ \hline
\textbf{Golden Standard} & \cellcolor[HTML]{F6E6E5} - & nej & \cellcolor[HTML]{F6F6C3}ja & nej & nej & nej & \cellcolor[HTML]{F6F6C3}nej & \cellcolor[HTML]{F6F6C3}ja & \cellcolor[HTML]{F6F6C3}nej & \cellcolor[HTML]{F6F6C3}ja & \cellcolor[HTML]{F6F6C3}ja \\ \hline
\rowcolor[HTML]{C0C0C0} & \multicolumn{11}{|c|}{\textbf{Lægemiddelskift nummer}} \\ 
\rowcolor[HTML]{C0C0C0} & \textbf{23} & \textbf{24} & \textbf{25} & \textbf{26} & \textbf{27} & \textbf{28} &  \textbf{29} & \textbf{30} & \textbf{31} & \textbf{32} & \textbf{33}  \\ \cline{1-10}
\textbf{Lægemiddel Nyt} & nej & \cellcolor[HTML]{F6F6C3}ja & nej & nej & nej & ja & ja & ja & ja & \cellcolor[HTML]{F6F6C3}nej & ja\\ \hline
\textbf{Golden Standard} & nej & \cellcolor[HTML]{F6F6C3}nej & nej & nej & nej & ja & ja& ja & ja& \cellcolor[HTML]{F6F6C3}ja & ja \\\hline
\end{tabular}
\end{table}

Det fremgår af Tabel \ref{table:test1}, at der ikke er en 60~\% enighed mellem medarbejderne i forhold til lægemiddelskift nummer 12, hvorfor dette lægemiddelskift undlades i den efterfølgende databehandling. En medarbejder kommenterede i forhold til dette lægemiddelskift, at begrundelsen for lægemiddelskiftet ikke var dækkende, hvilket fremgår af Tabel \ref{table:resultat2} i Appendiks \ref{App:Resultat}. Af Tabel \ref{table:test1} fremgår det, at der er uenighed mellem Lægemiddel Nyt og medarbejdernes vurdering for 9 lægemiddelskift svarende til 27~\% og enighed for 23, svarende til 69,7~\%. 
For at undersøge systemets nøjagtighed i forhold til Golden Standard beregnes sensitivitet og specificitet, hvilket fremgår af Tabel \ref{table:PositivNegativ}.

\begin{table}[H]
\caption{Krydstabel for Lægemiddel Nyt og Golden Standard for medarbejderne. Sensitiviteten er markeret med blåt og specificiteten er angivet med grønt.}
\label{table:PositivNegativ}
\centering
\begin{tabular}{ll|l|l|l|l}
 \rowcolor[HTML]{C0C0C0}                          &                          &                            & \multicolumn{2}{c|}{\textbf{Golden Standard}} &        \\ 
  \rowcolor[HTML]{C0C0C0}                                                         &                          &                            & Positiv       & Negativ          & Antal  \\ \hline
\cellcolor[HTML]{C0C0C0} \textbf{Lægemiddel}                                                \hspace{-0.5cm} \multirow{4}{*}{} &  \cellcolor[HTML]{C0C0C0}                                                      Positiv \multirow{2}{*}{} & Antal & 6                & 4                & 10     \\ \cline{3-6} 
\cellcolor[HTML]{C0C0C0}                                                                \textbf{Nyt} &   \cellcolor[HTML]{C0C0C0}                                               & \% indenfor Golden Standard & \cellcolor[HTML]{ECF4FF} 54,5 \%           & 19,0 \%           & 31,3 \% \\\cline{2-6}
\cellcolor[HTML]{C0C0C0}                                                                                 & \cellcolor[HTML]{C0C0C0}                                                                               Negativ \multirow{2}{*}{} & Talte & 5                & 17               & 22     \\ \cline{3-6}
          \cellcolor[HTML]{C0C0C0}                                                                             &                         \cellcolor[HTML]{C0C0C0}                                                        & \% indenfor Golden Standard & 45,5 \%           & \cellcolor[HTML]{D4EED3}81,0 \%            & 68,8 \% \\ \hline
 \cellcolor[HTML]{C0C0C0}                                                                            \textbf{Total}                         &                          \cellcolor[HTML]{C0C0C0}                                                                             & Antal                     & 11               & 21               & 32     \\ \cline{3-6}
                          \cellcolor[HTML]{C0C0C0}                                                                                   &                           \cellcolor[HTML]{C0C0C0}                                                                            & \% indenfor Golden Standard & 100 \%            & 100\%            & 100 \% 
\end{tabular}
\end{table}

Af Tabel \ref{table:PositivNegativ} fremgår det, at lægemiddelskiftet blev vurderet positiv, hvilket vil sige, at der var enighed mellem Lægemiddel Nyt og medarbejderne vurdering i 6 tilfælde. Ligeledes  blev lægemiddelskiftet i 21 tilfælde vurderet negativt af både Lægemiddel Nyt og medarbejderne, hvilket vil sige at der var enighed om at lægemiddelskiftet ikke krævede uddybende informationer. Lægemiddelskiftet blev i 5 tilfælde vurderet negativt af Lægemiddel Nyt og positivt af medarbejderne, mens det i 4 tilfælde blev vurderet positivt af Lægemiddel Nyt, men negativt af medarbejderne. 

Ud fra Tabel \ref{table:PositivNegativ} fremgår det ligeledes at systemets sensitivitet er 54,5~\% og sensitivitet er 81~\%.  Dette vil sige, at systemet i 54,5~\% af tilfældene vil foretage en korrekt vurdering af at et lægemiddelskift kræver uddybende information, og i 81~\% af tilfældene vil foretage en korrekt vurdering af at et lægemiddelskift ikke kræver uddybende information. Derudover vil systemt i 45,5~\% af tilfældene give anledning til type 2 fejl, hvilket vil sige, at systemet vil vurdere at et lægemiddelskift ikke kræver uddybende information, hvor dette var tilfældet. Yderligere vil systemet i 19~\% af tilfældene give anledning til type 1 fejl, hvor systemet vil vurdere at et lægemiddelskift som ikke kræver uddybende information vil blive vurderet til at kræve dette.

For at undersøge overensstemmelsen mellem Lægemiddel Nyt og Golden Standard bestemmes Cohen's kappa-koefficienten, hvilket fremgår af Tabel \ref{table:Kappa}.

\begin{table}[H]
\caption{Cohen's kappa analyse}
\label{table:Kappa}
\centering
\begin{tabular}{l|l|l|l}
 \rowcolor[HTML]{C0C0C0}  &  \textbf{Værdi} &   \textbf{Standard Error} & \textbf{Signifikant} \\ 
  \cellcolor[HTML]{C0C0C0}                                                       \textbf{Kappa} & 0,363 & 0,174 & 0,040 \\
\end{tabular}
\end{table}

Af Tabel \ref{table:Kappa} fremgår det at kappa-koefficienten er 0,363, hvilket vil sige at overensstemmelsen mellem Lægemiddel Nyt og Golden Standard er begrænset, hvis det antages at en perfekt overensstemmelse er svarende til 1. Derudover fremgår det at resultatet er signifikant (p=0,040).

\section{Risikoscore}
Ud af 33 lægemidler blev 19 lægemiddelskift  vurderet til at skulle have enten lavere eller højere risikoscore end systemet havde udregnet af én eller flere medarbejdere, hvilket fremgår af Tabel \ref{table:resultat3} i Appendiks \ref{App:Rang}. Ud af vurderinger fra 6 medarbejdere var der enighed mellem 4 i forhold til  at 2 ud af 33 lægemiddelskift krævede lavere risikoscore. For at evaluere risikoscoren sammenlignes vurderingerne af Lægemiddel Nyt og medarbejderne angivet som Golden Standard med den beregnede risikoscore for lægemiddelskiftene, hvilket fremgår af Tabel \ref{table:test2}.

\begin{table}[H]
\caption{Risikoscore og vurdering af lægemiddelskift for Lægemiddel Nyt og medarbejderne fra SRN, angivet som Golden Standard. Vurdering under 60~\% enighed mellem medarbejderne er markeret med rødt.}
\label{table:test2}
\centering
\begin{tabular}{l|c|c|c|c|c|c|c|c|c|c|c|c|c|c|c|c|c}
\rowcolor[HTML]{C0C0C0} \textbf{} & \multicolumn{11}{|c}{\textbf{Lægemiddelskift nummer}} \\
\rowcolor[HTML]{C0C0C0} & \textbf{1} & \textbf{2} & \textbf{3} & \textbf{4} & \textbf{5} & \textbf{6} & \textbf{7} &  \textbf{8} & \textbf{9} & \textbf{10} & \textbf{11}   \\ \hline
\textbf{Risikoscore [\%]} & 0  & 0 & 5 & 5 & 5 & 5 & 5 & 5 & 10 & 10 & 10  \\ \hline
\textbf{Lægemiddel Nyt} & nej & nej & nej & nej & nej & nej & nej & nej & ja & ja & nej \\ \hline
\textbf{Golden Standard} & nej & nej & nej& nej & nej &nej & nej & nej& ja & nej & nej \\ \hline
\rowcolor[HTML]{C0C0C0} & \multicolumn{11}{|c}{\textbf{Lægemiddelskift nummer}} \\
\rowcolor[HTML]{C0C0C0} & \textbf{12} & \textbf{13} & \textbf{14} &  \textbf{15} & \textbf{16} & \textbf{17} & \textbf{18} & \textbf{19} & \textbf{20} & \textbf{21} & \textbf{22}  \\ \hline
\textbf{Risikoscore [\%]} & \cellcolor[HTML]{F6E6E5} 15 & 15 & 15 & 15 & 20 & 20 & 20 & 25 & 25 & 25 & 30 \\ \hline
\textbf{Lægemiddel Nyt} & \cellcolor[HTML]{F6E6E5}nej & nej & nej & nej & nej & nej & ja & nej & ja & nej & nej\\ \hline
\textbf{Golden Standard} & \cellcolor[HTML]{F6E6E5} - & nej & ja & nej & nej & nej & nej & ja & nej & ja & ja \\ \hline
\rowcolor[HTML]{C0C0C0} & \multicolumn{11}{|c}{\textbf{Lægemiddelskift nummer}} \\ 
\rowcolor[HTML]{C0C0C0} & \textbf{23} & \textbf{24} & \textbf{25} & \textbf{26} & \textbf{27} & \textbf{28} &  \textbf{29} & \textbf{30} & \textbf{31} & \textbf{32} & \textbf{33}  \\ \cline{1-10}
\textbf{Risikoscore [\%]} & 30 & 30 & 35 & 35 & 35 & 40 & 40 & 45 & 50 & 50 & 55 \\ \hline 
\textbf{Lægemiddel Nyt} & nej & ja & nej & nej & nej & ja & ja & ja & ja & nej & ja\\ \hline
\textbf{Golden Standard} & nej & nej & nej & nej & nej & ja & ja& ja & ja&ja & ja \\\hline
\end{tabular}
\end{table}

Ud fra Tabel \ref{table:test2} fremgår det, at der ved lægemiddelskift nummer 12 var under 60~\% enighed mellem medarbejderne, hvormed dette skift ikke tages med i den videre databehandling.  
Derudover fremgår det, at lægemiddelskift med en risikoscore på 10, 20, 25, 30 og 50 \% er vurderet forskelligt af Lægemiddel Nyt i forhold til uddybende information, hvor en risikoscore på 10, 15, 25 og 30 \% er vurderet forskelligt af medarbejderne fra SRN. For lægemiddelskift med en risikoscore mellem 0 og 5~\% er der enighed mellem medarbejderne, hvilket fremgår af Tabel \ref{table:resultat}. Denne enighed gør sig også gældende mellem Lægemiddel Nyt og medarbejderne. 

Lægemiddelskift nummer 9 og 14, med en risikoscore på henholdsvis 10 og 15 \%, er vurderet anderledes end de resterende med samme risikoscore. For disse lægemiddelskift var der angivet kommentarer i forhold til uenighed om, hvorvidt styrken eller styrkeangivelsen var ændret, hvilket fremgår af Tabel \ref{table:resultat2} i Appendiks \ref{App:Resultat}. Ligeledes blev lægemiddelskift nummer 22, med en risikoscore på 30~\%, vurderet anderledes end de andre med samme risikoscore. En medarbejder kommenterede i forhold til dette, at ATC-koden i dette tilfælde vil være mindre kritisk sammenlignet med andre.

For at undersøge systemets diskriminationsgrænse i forhold til at definere den bedste grænse for, hvornår et lægemiddelskift kræver uddybende information ud fra risikoscoren er sensitiviteten som funktion af 1-specificiteten for hver risikoscore udregnet, hvilket fremgår af Tabel \ref{table:app_ROC} i Appendiks \ref{App:ROC}, og visualiseret via ROC-kurven, som fremgår af Figur \ref{fig:ROC}. 

\begin{figure} [H]
	\centering
	\begin{subfigure}{0.49\textwidth} % width of left subfigure
		\includegraphics[width=\textwidth]{billeder/ROC.png}
\subcaption{Lægemiddel Nyt. \label{fig:ROC1}}
	\end{subfigure}
	\vspace{1em}
	\begin{subfigure}{0.49\textwidth} 
		\includegraphics[width=\textwidth]{billeder/ROC1.png}
\subcaption{Golden Standard. \label{fig:ROC2}}
	\end{subfigure}
	\vspace{-0.5cm}
\caption{ROC-kurve for Lægemiddel Nyt på Figur \ref{fig:ROC1} og medarbejderne, angivet som Golden Standard, på Figur \ref{fig:ROC2}. På y-aksen er sensitivitet angivet og x-aksen 1-specificitet. Den røde linje indikerer identitetslinjen og den blå linje ROC-kurven. Diskriminationsgrænserne er indikeret med en grøn prik.}
\label{fig:ROC}
\end{figure} 

På Figur \ref{fig:ROC} fremgår det, at ROC-kurven for Lægemiddel Nyt og medarbejderne, som er illustreret af den blå linje, er over identitetslinjen, som er illustreret af den røde. Derudover er diskriminationsgrænsen markeret med grøn, der hvor sensitiviteten og specificiteten er ligeværdige. For Lægemiddel Nyt er denne indikeret men en sensitivitet på 0,8 og 1-specificitet på 0,455, svarende til en specificitet på 0,545. Medarbejdernes diskriminationsgrænse er indikeret ved en sensitivitet på 0,818 og en 1-specificitet på 0,286, svarende til en specificitet på 0,714. Ved at sammenligne disse værdier med Tabel \ref{table:app_ROC} i Appendiks \ref{App:ROC} indikerer dette et cut-off på henholdsvis 17,5 for Lægemiddel Nyt og 22,5 for medarbejderne. Dette vil sige, at i tilfælde af, at scoren er under denne cut-off vil dette betyde, at der ikke skal være uddybende information til klinikken, mens en cut-off over denne værdi vil betyde at lægemiddelskiftet kræver uddybende information til klinikken. \textcolor{red}{Er lidt i tvivl om, hvordan jeg skal definere balancen mellem sensitivitet og specificitet. Synes det er vigtigt at alle lægemidler som kræver uddybende information bliver uddybet, så en høj sensitivitet, men at lægemidler som ikke kræver uddybende information bliver uddybet har ikke så stor betydning, så lav specificitet.}

For at kunne vurdere nøjagtigheden af risikoscoren for både Lægemiddel Nyt og medarbejderne beregnes arealet under kurven, hvilket fremgår af Tabel \ref{table:AUC}.

\begin{table}[H]
\caption{Arealet under kurven for Lægemiddel Nyt og medarbejderne, angivet som Golden Standard.}
\label{table:AUC}
\centering
\begin{tabular}{l|l|p{1.8cm}|l|l|l}
 \rowcolor[HTML]{C0C0C0}  &  &                         & \textbf{Signifikant} &
 \multicolumn{2}{c}{\textbf{95 \% Konfidensinterval}}  \\ 
  \rowcolor[HTML]{C0C0C0}                                                         & \textbf{Areal} &  \textbf{Standard error}                       &  & Nedre grænse      & Øvre grænse    \\ \hline
 \cellcolor[HTML]{C0C0C0}\textbf{ Lægemiddel Nyt }& 0,766 & 0,090 & 0,017 & 0,589 & 0,942  \\ \hline
 \cellcolor[HTML]{C0C0C0} \textbf{Golden Standard} & 0,840 & 0,074 & 0,002 & 0,696 & 0,984 \\
\end{tabular}
\end{table}

Af Tabel \ref{table:AUC} fremgår det at arealet under kurven er 0,766 for Lægemiddel Nyt og 0,840 for medarbejderne. I forhold til at antage at 1 er en perfekt test, vurderes risikoscoren for Lægemiddel Nyt til at være rimelig og medarbejderne til at være god. Derudover er ROC-kurven statistisk signifikant for Lægemiddel Nyt (p=0,017) og Golden Standard (p=0,002). Lægemiddel Nyt har en nedre grænse på 0,589 og en øvre grænse på 0,942 af arealet under kurven, hvilket vil sige, at ved en vurdering af 100 lægemiddelskiftet vil 95~\% af disse være mellem 0,589 og 0,942. For medarbejderne er denne nedre grænse på 0,696 og den øvre grænse på 0,984 af arealet under kurven, dette vil sige, at 95~\% af konfidensintervallet ligger mellem 0,696 og 0,984.

\section{Risikofaktorer og vægtning}
I forhold til risikofaktorer er det vurderet af flere medarbejdere fra SRN, jævnfør Appendiks 
\ref{App:Referat}, at look-a-like ikke skal vægtes højt i forhold til beregningen af risikoscore, da kompleksiteten af lægemiddelskift afhænger af om der er andre faktorer som har betydning for lægemiddelskiftet. I forhold til Medicinrådet blev der kommenteret fra flere, at Medicinrådet nødvendigvis ikke er årsagen til at et lægemiddelskift er kompleks, men mere, hvilke og hvor mange ændringer, der foretages i forbindelse med Medicinrådets behandlingsvejledning. Derudover blev der i forbindelse med risikovurderingen kommenteret i forhold til ATC-kritiske lægemiddelskift, at nogle ATC-koder er mere kritiske end andre, hvilket fremgår af Tabel \ref{table:resultat3} i Appendiks \ref{App:Resultat}. 

Derudover mente flere at var nødvendigt at variere sværhedsgraden af navn og dispenseringsform i forhold til vægtningen af disse, da nogle ændringer kan være ligestillet og derfor har en mindre betydning, jævnfør Appendiks \ref{App:Referat}. Samtidig blev der kommenteret af en gruppe, at der skal skelnes mellem om styrken er ændret i forhold til angivelsen eller pakningsstørrelse, da det ene er mere kompleks end det andet og vil derfor skulle vægtes forskelligt. Dette blev ligeledes bekræftet ved kommentarer i forbindelse med risikovurderingen, som fremgår af Tabel \ref{table:resultat3} i Appendiks \ref{App:Resultat}, hvor der opstod forvirring i forhold til at skelne mellem om styrken var ændret eller om det var styrkeangivelsen.




%\section{Systemets performance}
%Ud af lægemiddelskiftene var der for 13 lægemidler enighed mellem medarbejderne fra SRN. Ud af 33 blev 11 lægemidler vurderet til ikke at krævede uddybende samt at 2 lægemidler krævede uddybende information ved implementering af disse i klinikken. Størstedelen af lægemiddelskift blev vurderet af 9 ud af 11 til ikke at kræve uddybende information, mens 2 medarbejdere vurderede at størstedelen af lægemidlerne krævede uddybende information, hvilket er markeret med blåt i Tabel \ref{table:test1}. Gennemsnittet blev vurderet til at 12,27 lægemidler, svarende til 37,19~\%, krævede uddybende information, mens 20,63 lægemidler, svarende til 62,5~\%, ikke krævede uddybende information, hvilket ligeledes fremgår af Tabel \ref{table:test1}. 

%\begin{table}[H]
%\caption{Krydstabel for vurderingen ja eller nej i forhold til uddybende information om lægemiddelskiftet og testpersoner samt lægemiddel Nyt, som er indikeret som nummer 12. \textcolor{red}{Er i tvivl, hvorvidt det giver mening at have lægemiddel Nyt med i denne test}}
%\vspace{2mm}
%\label{table:test1}
%\centering
%\begin{tabular}{l|c|c|c|c|c|c|c|c|c|c|c|c|c}
%\rowcolor[HTML]{C0C0C0}
% & \multicolumn{11}{|c|}{\textbf{Testperson}} & \textbf{} & \textbf{Total}    \\
%\rowcolor[HTML]{C0C0C0}  & \textbf{1}  & \textbf{2}  & \textbf{3} & \textbf{4} & \textbf{5} & \textbf{6} & \textbf{7} & \textbf{8} & \textbf{9} & \textbf{10} & \textbf{11} & \textbf{12}  & \\ 
%\cellcolor[HTML]{C0C0C0}{Antal af Nej}            & 24 & 20 & 21  &  28 &  15 &  21 & 17   &  12  & 21  & 26   &  22  & 29 & 256      \\ \cline{2-13}
%\cellcolor[HTML]{C0C0C0}{\% af Nej }  & 9,4 & 7,8  & 8,2 & 10,9  & 5,9  &  8,2 & 6,6  & 4,7  & 8,2 & 10,2  &  8,6 & 11,3 &  100    \\ \cline{2-13}
%\cellcolor[HTML]{C0C0C0}{\% af Gruppe} & 72,7 &  60,6  & 63,6 & 84,8   & 45,5   & 63,6  & 51,5  & 36,4 & 63,6 & 78,8 & 66,7  & 87,9 & 64,6 \\ \cline{2-13}
%\cellcolor[HTML]{C0C0C0}{\% af Total}  & 6,1 & 5,1 & 5,3  & 7,1 & 3,8 & 5,3  & 4,3 & 3,0  & 5,3 & 6,6  & 5,6  & 7,3 &  64,6 \\ \hline
%\cellcolor[HTML]{C0C0C0}{Antal af Ja} & 9 & 13 & 12 & 5  &  \cellcolor{blue}{18} & 12  & 16  &  \cellcolor{blue}{21}  & 12 & 7  & 11 & 4 & 140    \\ \cline{2-13} \cellcolor[HTML]{C0C0C0}{\% af Ja}  & 6,4 & 9,3  & 8,6 & 3,6 & 12,9  & 8,6  & 11,4 & 15  & 8,6  & 5,0 & 7,9  & 2,9 & 100   \\ \cline{2-13}
% \cellcolor[HTML]{C0C0C0}{\% af Gruppe} & 27,3 & 39,4 & 36,4 & 15,2   & 54,5  & 36,4 & 48,5 & 63,6  & 36,4 & 21,2 & 33,3  & 12,1 & 35,4      \\ \cline{2-13}
%\cellcolor[HTML]{C0C0C0}{\% af Total}  & 2,3   & 3,3 & 3,0 & 1,3 & 4,5 & 3,0 & 4,0 & 5,3 & 3,0 & 1,8 &  2,8  & 1,0 &  35,4     \\ \hline  \cellcolor[HTML]{C0C0C0}{Total}  & 33 & 33 & 33  & 33  & 33  & 33  & 33  &  33 &  33 &   33 &  33  &  33 &  396     \\ \cline{2-13}
% \cellcolor[HTML]{C0C0C0}{\% af Nej og Ja}  & 8,3    & 8,3    &  8,3  & 8,3   &  8,3  &  8,3  &  8,3  &  8,3  & 8,3   & 8,3    & 8,3    &  8,3 &  100    \\ \cline{2-13}
% \cellcolor[HTML]{C0C0C0}{\% af Gruppe} & 100   & 100   &  100 &  100 &  100 &  100 &  100 & 100  & 100  & 100   &  100  & 100 &      100 \\ \cline{2-13}
%\cellcolor[HTML]{C0C0C0}{\% af Total} & 8,3    & 8,3    &  8,3  & 8,3   &  8,3  &  8,3  &  8,3  &  8,3  & 8,3   & 8,3    & 8,3    &  8,3 &  100   \\
%\end{tabular}
%\end{table}

%For alle lægemiddelskift med en risikoscore på enten 0 eller 5~\% blev det vurderet at uddybende information var unødvendigt, hvilket vil sige at ændring i navn alene kræver mindre opmærksomhed, hvilket er illustreret af Tabel \ref{table:risikofaktorer}. Derudover var 10 ud af 11 medarbejdere enige i at lægemidler, hvor styrken alene eller i kombination med ændring i navn krævede uddybende information til klinikken. De medarbejdere som var uenige kommenterede at styrkeangivelsen var ændret, men at selve styrken ikke var ændret. Derimod blev ændring i dispenseringsform alene ikke vurderet lige så nødvendig at give uddybende information som ved styrke. Vurderingen i forhold til hvorvidt der kræves uddybende information ved ændringer i navn, styrke og dispenseringsform eller en kombination af disse fremgår af Tabel \ref{table:risikofaktorer}.

%\begin{table}[H]
%\caption{Oversigt over risikofaktorer og kombination mellem disse i forhold til uddybende information omkring lægemiddelskift (Ja = uddybende information, nej = ikke uddybende information).}
%\vspace{2mm}
%\label{table:risikofaktorer}
%\centering
%\begin{tabular}{l c|c|c|c}
%\rowcolor[HTML]{C0C0C0} & & \multicolumn{3}{c}{\textbf{I kombination med}} \\
%\rowcolor[HTML]{C0C0C0} & & \textbf{Navn}& \textbf{Styrke} & \textbf{Dispenseringsform} \\
%\cellcolor[HTML]{C0C0C0} & \cellcolor[HTML]{C0C0C0}Ja & 0 & 9.5 & 7 \\ \cline{2-5}	   
%\multirow{-2}{*}{\cellcolor[HTML]{C0C0C0}\textbf{Navn}} & \cellcolor[HTML]{C0C0C0}Nej & 11 & 1.5 & 4\\ \hline 
%\cellcolor[HTML]{C0C0C0} &  \cellcolor[HTML]{C0C0C0}Ja & 9.5 & 10 & 9.5 \\ \cline{2-5}	   
%\multirow{-2}{*}{\cellcolor[HTML]{C0C0C0}\textbf{Styrke}} & \cellcolor[HTML]{C0C0C0}Nej & 1.5 & 1 & 1.5\\ \hline 
%\cellcolor[HTML]{C0C0C0} & \cellcolor[HTML]{C0C0C0}Ja & 7 & 9.5 & 3.5 \\ \cline{2-5}	   
%\multirow{-2}{*}{\cellcolor[HTML]{C0C0C0}\textbf{Dispenseringsform}} & \cellcolor[HTML]{C0C0C0}Nej & 4 & 1.5 & 7.5 \\ \hline 
%\end{tabular}
%\end{table}

%Af Tabel \ref{table:risikofaktorer} fremgår det at, hvis styrken er ændret vil dette som udgangspunkt være den risikofaktorer, der oftest kræver uddybende information, både alene eller i kombination med enten navn eller dispenseringsform. Ændring i dispenseringsform blev i sjældnere tilfælde end styrke, vurderet til at kræve uddybende information og denne vurdering steg hvis ændring i dispenseringsform blev kombineret med navn eller styrke. Ændring i navn kræver kun uddybende information i kombination med styrke eller dispenseringsform. 

%Ud af 11 medarbejdere var 10 enige i at lægemiddelskift, hvor ATC-koden var angivet som kritisk eller at lægemidlet indgår i Medicinrådets behandlingsvejledning alene ikke krævede uddybende information til klinikken. Kommentarer fra de som var uenige med størstedelen var at et lægemiddelskift, som indgår i Medicinrådets behandlingsvejledning ikke nødvendigvis er kritisk, men at ændringer i behandlingsvejledning kan gøre lægemiddelskiftet kritisk. Derudover blev det vurderet at look-a-like alene og i kombination med navn ikke have indflydelse på om lægemiddelskiftet krævede uddybende information. I forhold til dette blev der kommenteret at look-a-like som udgangspunkt ikke er et problem, men kan opstå ved variationer af lægemidlets navn, hvor dette får betydning for ændringer ved dispenseringsformer.

%Rækkefølgen på lægemiddelskiftene som er vurderet på baggrund af risikoscoren blev vurderet af 6 ud af de 11 medarbejdere. Ud af 33 lægemiddelskift blev 18 lægemidler vurderet af én eller flere til at skulle rangeres lavere end systemet. Ud af de 6 medarbejdere var 4 eller 3 enige i at henholdsvis 2 eller 3 lægemidler skulle rangeres lavere end systemet.

%\section{Systemets brugbarhed}
%Systemet blev vurderet som et brugbart hjælpeværktøj i forhold til at oplysninger om lægemiddelskiftet kan genereres automatisk, hvormed der kan bruges mindre tid på helt simple skift. Dog skal systemet ikke overtage, da erfaringen inden for området stadig har stor betydning for risikovurderingen af lægemiddelskift. 

%Det blev vurderet at funktioner, såsom Look-a-like og Medicinrådet, ikke nødvendigvis skulle vægtes så højt, da det afhænger af den enkelte situation. Look-a-like vægtes generelt ikke særligt højt i den nuværende vurdering af lægemiddelskift varetaget af medarbejdere fra SRN, hvoraf antallet af egenskaber der skifter er mere afgørende for kompleksiteten såsom holdbarhed, emballage og konserveringsmidler. I forhold til Medicinrådet blev det vurderet at det ikke var vigtigt, hvorvidt lægemidlet indgik i Medicinrådets behandlingsvejledning, men hvorvidt der var foretaget ændringer i denne havde større betydning. 

%I forhold til videreudvikling blev det vurderet at systemet skal have flere problemstillinger som input, såsom pris i forhold til hvor meget det koster og hvor meget der forbruges samt hvor mange patienter og afdelinger som anvender lægemidlet. Derudover bør granuleringen af look-a-like, navneændring og dispenseringsform i forhold til vægtningen af risikoscoren overvejes. Nogle ændringer i navn og dispenseringsform vil kunne ligestilles og derfor ikke have betydning for klinikken, hvis der skiftes mellem disse. Der skal ligeledes skelnes mellem styrke og styrkeangivelse, da ændringerne ved styrkeangivelse ikke vil have en betydning, hvis pakningsstørrelsen ikke er ændret. Systemets anvendelighed blev derudover perspektiveret i forhold til at kunne anvendes til andre processer i forbindelse med lægemiddelskift såsom at anvende et lignende systemet ved udbud på lægemidler inden de endelige lægemiddelskift er vedtaget. 



