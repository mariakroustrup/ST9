Studier har påvist, at langvarig mindfulness meditation kan forbedre en række kognitive funktioner for patienter med kroniske smerter. 
Imidlertid er antallet af studier, der undersøger effekten af kortvarig mindfulness meditation for patienter med kroniske nakkesmerter begrænset. 

Formålet med dette studie er at påvise om kortvarig mindfulness focused attention meditation kan påvirke smertesensitivitet. Smertetærsklen og smertetolerancen blev målt med et algometer på den øvre højre trapezius på raske forsøgspersoner over to måling sessioner med 5 dages mellemrum. 
Behandlingsgruppen udøvede 20 minutters mindfulness focused attention meditation over 5 sammenhængende dage, mens kontrolgruppen fortsatte deres normale rutiner mellem målingerne.

Resultater viser ingen signifikant forskel mellem smertetærsklen og smertetolerance mellem behandlings- og kontrolgruppen. 

På trods af dette bidrager studiet stadig til området inden for smertelindring ved brug af mindfulness meditation. 

Yderligere forskning er dog nødvendig for at undersøge effekten af mindfulness focused attention meditation over en længere periode, eftersom dette studie viser en tendens til at smertetærsklen og smertetolerancen øges.
